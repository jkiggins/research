\chapter{Adaptation of the Astrocyte Model To Multi-Synapse
  Configurations} \label{chapter:obj3}

In biology, Astrocytes have not been observed influencing a single synapse on a
single neuron. In general they encapsulate hundreds synapse across a single
digit number of neurons. In this chapter, single neuron, multi-synapse
configurations will be explored. Taking that small step closer to a bio-inspired
configuration, but remaining simple enough to explore fundamental properties.

\section{Multi-Synapse Cell-Level Response}
What is the ultimage goal here... to support some learning on some specific set
of inputs, that would not otherwise be achievable w/ STDP. What is the task?
what is an interesting task that involves spike timing?

What are some things I've got in my notes about this?

Gating of plasticity
  - Only allow plasticity on synchronous inputs
  - The opposite of ^
  
Orchestration between Synapse
  - Don't allow multiple synapse to learn from the same post-synaptic spike (WTA
  Pairing)
  - Global learning ``pool'' Each thr event on any synapse depletes the pool,
  which then refills
  - Relative strength of plasticity


\section{Summary}
Though there is a lack of general consensus, much of the Neuroscience literature
suggests that Astrocytes have a fast local response to synaptic activity, which
is then integrated to a cell-level response. Taking inspiration from this
multiple level integration approach, the existing signals and Astrocyte behavior
are duplicated for each synapse that an Astrocyte influences. The initial goal
then, is to determine how the introduction of additional synapses effects
general behavior. At a high level, the simple \ca threshold approach to driving
plasticity is no longer stable in a 1nNs1a configuration. This is somewhat
expected, and there is a need for additional constraints for weight values to
converge. To meet this need synaptic coupling is introduced. This coupling
drives plasticity of the coupled synapses based on activity across the coupled
domain. This mimics an effect explored in Neuroscience literature, which is that
ER bodies within an Astrocyte can form pathways between synapses that don't
immediately affect the main cell body, but do influence a group of synapses. The
two coupling rules explored in this chapter are AND and NAND. As their
names suggest, these coupling rules are tied to a particular logic function. In
general, the Astrocyte will drive weight values towards implementing the desired
function, detecting behavior that is not in alignment and updating weights
accordingly.

\section{Synaptic Coupling}
