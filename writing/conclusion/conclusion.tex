% Author       : Jacob Kiggins
% Contact      : jmk1154@g.rit.edu
% Date Created : 01/31/2021
%

\chapter{Future Work} \label{chapter:future-work}
This work covers a broad range of topics, and explores work in fields that are
still largely in their infancy. Plasticity, while well-defined in some areas of
machine learning, isn't as clear-cut in spiking neural networks. Further,
Astrocyte-Neuron interactions, aren't well understood in the Neuroscience
community, and the mathematical model representing them vary widely as a
result. The goal of this work was to extract some useful features, given what is
known about Astrocyte-Neuron interactions, using a simple model. This was
achieved, but the path taken leaves many interesting questions unanswered, and
opens the door to future work.

\section{Astrocyte Local Learning}
Local learning in this work was limited to two different types of inputs, and
the Astrocyte wasn't evaluated against any particular learning objective with
those input. Some interesting properties were teased out, such as the tendency of
the Astrocyte to change weights, seeking a state with low \ca response to
inputs. This is good, as that low \ca response configuration can be leveraged,
to direct an Astrocyte to a particular goal. The threshold function
\emph{Dw(\ca)} is fairly simple, more complicated functions should be explored
here, to determine what temporal features an Astrocyte can learn.

\section{Multi Synapse Plasticity}
This work just scratches the surface of what should be explored in a
mutli-synapse configuration. One of the best next steps however, is to explore
input and learning objective pairings that allow for local and global dynamics
to work together with plasticity. A harmonious relationship between global and
local dynamics would allow a global rule to operate on a longer time-scale, and
implement a higher-level function.

%% \section{Multi Neuron and Synapse Coordination}

\section{Supervised Learning}

Astrocytes offer an interesting opportunity to work in tandom with a supervisory
learning signal. In may cases, especially with a longer running task-based
learning objective, feedback from a supervisory signal is sparse in
time. Astrocytes are in a fantastic position to bridge this gap. They can
maintain \ca concentrations over longer time-scales, and use their connections
and internal states to direct plasticity intelligently, given a global and
non-specific learning signal.

\chapter{Conclusion} \label{chapter:conclusion}

