\chapter{Extension Astrocyte Model To Multi-Synapse
  Configurations} \label{chapter:obj3}

In biology, Astrocytes have not been observed influencing a single synapse on a
single neuron. In general they encapsulate hundreds synapse across a single
digit number of neurons. In this chapter, single neuron, multi-synapse
configurations will be explored. Taking that small step closer to a bio-inspired
configuration, but remaining simple enough to explore fundamental properties.

%% \section{Multi-Synapse Cell-Level Response}
%% What is the ultimage goal here... to support some learning on some specific set
%% of inputs, that would not otherwise be achievable w/ STDP. What is the task?
%% what is an interesting task that involves spike timing?

%% What are some things I've got in my notes about this?

%% Gating of plasticity
%%   - Only allow plasticity on synchronous inputs
%%   - The opposite of ^
  
%% Orchestration between Synapse
%%   - Don't allow multiple synapse to learn from the same post-synaptic spike (WTA
%%   Pairing)
%%   - Global learning ``pool'' Each thr event on any synapse depletes the pool,
%%   which then refills
%%   - Relative strength of plasticity


\section{Summary}
Though there is a lack of general consensus, much of the Neuroscience literature
suggests that Astrocytes have a fast local response to synaptic activity, which
is then integrated to a cell-level response. Taking inspiration from this
multi-level integration approach, the Astrocyte model is extended. At the
local level, \ipt, \kp, and \ca dynamics remain mostly the same compared to
previous chapters. The \ca signals do not remain local however, and instead
propagate up to a regional level. At this multi-synapse, regional level, \ca
responses from multiple synapses are integrated. A response is then generated at
the regional level, then travels back down to one or more synapses, affecting
local plasticity. The two regional functions are explored in this
chapter, AND and NAND. In general, the Astrocyte will drive weight values
towards implementing the desired function, detecting behavior that is not in
alignment and updating weights accordingly. For these functions in particular,
having a multi-synapse view of activity is critical.


\section{Synaptic Coupling}
In order to implement synaptic coupling in an efficient and bio-realistic way,
there are some changes to internal Astrocyte signaling, when compared to the
1N1S1A configuration. Figure \ref{fig:astro:syn_coupling} shows the progression
of the Astrocyte model to support coupling. Instead of each synapse having
separate \ipt, \kp, and \ca dynamics, the coupling rule defines the mapping of
these concentrations, and drives any resetting behavior.

\asvgf{figures/1nNs1a_diagram_obj3.svg}{Astrocyte Functional Diagram for
  Multi-Synapse Synaptic Coupling}{fig:astro:syn_coupling}{0.5}

\subsection{AND Coupling}

The AND coupling function is implemented across 2 or more synapses within a
single Astrocyte, and may be one of many coupling functions at work. Logically,
AND coupling can be defined by Figure \ref{fig:logical-and-coupling}.

\asvgf{figures/AstroAndCoupling.svg}{Logical Definition for AND
  Coupling}{fig:logical-and-coupling}{0.5}

In Figure \ref{fig:logical-and-coupling} each of the blue or orange boxes
represents a spiking event, consisting of one or more spikes. In the case of
multiple spikes, the synaptic weight may be tuned to be sensitive to a
particular spiking pattern, if that pattern is differentiable with a single
weight value and LIF neuron dynamics.

In order to test the Astrocyte response and tune parameters, some timing
constraints must be defined. Initially, a single timing constraint
$\delta_{ptp}=10ms$ will be used. This implies that the time between the
first pre-synaptic spike, to the first (possibly only) post-synaptic spike must
be no more than this value to ensure proper behavior. $\delta_{ptp}/2$ will be
the maximum time between any two pre-synaptic spikes. With these timing
constraints a variety of pre-pre-post spiking events can be generated in a
two-synapse configuration. This experiment will remove the influence of synaptic
weights and LIF neuron dynamics, focusing solely on Astrocyte response.

Using the timing constraints outlined above, the following simulations were run
\begin{itemize}
  \item Isolated 10ms simulations with randomly generated presynaptic and
    postsynaptic spikes, across two synapses. This simulation revealed a few
    parameter values that were incompatible with the 10ms time window. After
    correcting these, a simulation with 10000 10ms bouts reveled 0 mismatches
    from expected Astrocyte behavior. This easily covers the space of possible
    spike configurations, which has an upper bound of 1331.

  \item A continuous simulation, with inputs generated by concatenating a number
    of 10ms bouts. In this configuration, there are some situations where the
    Astrocyte does not respond as expected, in general this is due to a \ca
    deviation from zero, when bouts of activity intersect. These situations made
    up $\approx 12\%$ of possible inputs.

  \item An Astrocyte-LIF (1n2s1a) simulation, given continuous inputs generated
    by concatenating 10ms bouts of only presynaptic spikes. In this
    configurations, instead of randomly selecting postsynaptic spike times, an
    LIF neuron with synaptic weights drives post-synaptic spikes. Initially the
    default synaptic weight of $\approx 0.7$ wasn't sufficient to drive the LIF
    neuron to fire. As expected, the Astrocyte traces showed \dser signals
    arriving at both synapses for all inputs that satisfied an AND
    condition. Completing the link, and allowing \ca to drive plasticity, showed
    weight updates corresponding to the \dser signals and \ca
    concentration. Overall, with these simple inputs, 
\end{itemize}



