% introduction.tex
%
% Author       : James Mnatzaganian
% Contact      : http://techtorials.me
% Date Created : 08/27/15
%
% Description  : Introduction chapter used by "thesis.tex".
%
% Copyright (c) 2015 James Mnatzaganian
%
% Version 2 ::
% Author       : Andres Kwasinski
% Contact      : https://people.rit.edu/axkeec/
% Date Created : 01/09/2020
%

% NOTE: All filler text has "TODO" written. This must be removed in the final copy!

\chapter{Introduction}\label{section:introduction}
\section{Motivation}
Biological neurons and neuronal networks originally inspired traditional,
ANNs. Decades of research and hardware advancements later, ANNs have achieved
and even surpassed humans in specialized tasks. Currently neural networks
outperform humans in object recognition and classification, and a variety of
games (chess, go, some video games). This performance comes at a cost however,
huge networks with millions of parameters are needed. Computing their decision
in real-time is challenging, and if possible requires specialized, power-hungry
hardware. In addition their ability to handle time-series data is limited. The
human brain on the other hand, is able to handle a wide variety of complex tasks
with $\approx$ 20 W of power, and seamlessly handle time-series input. Over the
decades, research in the neuroscience field has advanced as well and it is known
that the perceptron neuron is poor approximation of a biological
neuron. Neuroscientists have also discovered a variety of temporal encoding
schemes beyond just rate-based coding, which initially inspired the
simplifications of the perceptron. This suggests traditional artificial neurons
can't approximate biological neuron communication and encoding. As the
functional gap between biological and artificial neurons is realized it is
important to consider the differences, and possible advantages to closing this
gap. Spiking neural networks, are a step in that direction and there is growing
interest in exploring their properties. Broadly, the goal of such research is to
tease out the elements of biological neural networks that facilitate spiking
neural networks processing of time-series data, in addition to real-time,
on-line training. Achieving these things may require a fundamental shift in the
models used, a shift towards spiking neural networks. Reducing the power
requirement for a given task would be extremely beneficial from an engineering
perspective, but there are further implications. SNNs lend themselves to more
efficient, and simpler hardware implementations compared to ANNs. Reducing the
number of transistors required for a single neuron could open doors for deeper
and larger networks then would be feasible today.
    
\section{Thesis Objectives}
The overarching goal of this thesis is to develop a bio-inspired Astrocyte
model, and more broadly an Astrocyte-Neuron interaction model. This model will
be computationally simple, scalable, and captures common themes within
Neuroscience literature. To direct this research, the developed Astrocyte
element will drive synaptic plasticity; First generalizing, then diverging from
the widely used STDP rule. Benefits of Astrocyte-like control of synaptic
plasticity will be demonstrated in the single synapse, single neuron
configuration (1S1N) as well as select multi-synapse configurations. Inputs will
consist of procedurally generated spiking inputs. This would include poisson
rate-coded spike trains, spiking inputs representing Boolean variables, and
temporal inputs that follow specific patterns. To evaluate each experiment
different intrinsic properties will be observed, such as convergence/divergence
of weights, speed of convergence, stable points across a parameter space, and
lastly performance on any task driven experiments. Statistical analysis of
various intermediate signals, and weights may be employed as well.

More specific objectives.
\begin{enumerate}
  %% Specify -> developing a novel astrocyte model. Stress what is new and value-added
\item To Develop an Astrocyte \Gls{lif}-Neuron model based on bio-chemical pathways
  common in background literature. This Astrocyte should respond internally to
  pre and post-synaptic spikes In a way that can mimic STDP, but is more
  generalized and flexible.

  \item Show, in the case of a single spiking neuron with one input, how the
    developed Astrocyte-\Gls{lif} model can drive synaptic plasticity in a way that
    generalizes, and extends STDP.

  \item Extend the Developed Astrocyte-Synapse Model to include coordination of
    plasticity rules across multiple synapses, and show how this coordination is
    critical to a learning task.

\end{enumerate}
