% background.tex
%
% Author       : Jacob Kiggins
% Contact      : jmk1154@g.rit.edu
% Date Created : 01/09/2020
%

% NOTE: All filler text has "TODO" written. This must be removed in the final copy!

\chapter{Background and Related Work}\label{chapter:background}
    \section{Spiking Neural Networks}
    \Glspl{snn} are a type of brain-inspired computing
    model. This class of \Gls{ann} is the most
    bio-realistic that is used in engineering applications. They are an
    important topic in the machine learning field specifically, and have gained
    interest in recent years. There are a variety of motivations for exploring
    bio-inspired compute models. As of 2015 most papers were concerned with the
    low-power and parallelism offered by \Glspl{snn}, but other interests include
    processing speed, and ability to handle time-series data
    \parencite{schuman_2017}. These bio-inspired models have distinct advantages when
    looking at hardware implementations, and can offer significantly better
    performance/watt when compared to traditional ANNs. In addition, \Glspl{snn} offer
    a method of exploring some of the important features of biological neural
    networks that aren't yet possible to replicate in traditional artificial
    networks. These include one-shot learning, short-term memory, and problems
    involving time-series data, where the mammalian brain excels
    \parencite{kasabov_2013}.

    %% \todo{Add figure of SNN topology}

    Traditional perceptron type neurons (ANs) are similar to spiking neurons,
    but differ in a few key ways. Both types of neurons have some number of
    synapses, on which inputs are received. These inputs are transformed based
    on per-synapse parameters (generally multiplied by a synaptic weight), then
    summed together. This is where the behavior diverges. Spiking Neurons (SNs)
    add to a leaky membrane voltage with each input. If this membrane voltage
    satisfies an activation function (generally a voltage threshold) then the SN
    fires, or generates an output spike on its post-synaptic terminal. For ANs,
    there is still an activation function, and that function may produce a
    spike-like output (1 or 0), but can output any real value. Similarly on the
    input side, SNs receive spikes on their synapses, where ANs can receive any
    real value. Another primary difference is related to time: A SNs membrane
    voltage persists over time, unlike ANs, where there isn't a per-neuron state
    maintained across inputs.
    
    \Glspl{snn}, formed by connections between spiking neurons, are a bio-realistic
    type of ANN. Spiking neurons instead emit voltage spikes. These spikes
    propagate across a synapse, are usually multiplied by a weight, and are
    sometimes transformed in other ways (delay, filtering, etc...). The output
    of this transformation is called a \Gls{psp}. \Glspl{psp} are
    added over space and time, and accumulate at the neuron downstream, in the
    form of membrane voltage, or more generically a state variable. An
    activation function of this state variable determines if and when a spiking
    neuron fires. Generally, once a certain threshold is reached the
    post-synaptic neuron emits a voltage spike (fires). Since there are many
    Spiking Neuron models, it is necessary to formalize what is considered a
    spiking neuron. In general they process information from one or more inputs,
    and produce a single spike-like output. The probability of firing is
    increased by excitatory inputs, and decreased by inhibitory inputs. There is
    at least one internal state variable, and depending on this variable output
    spikes are generated \parencite{ponulak_2011}.
    
    There are many potential models that fit the requirements of a spiking
    neuron. In addition, some models are more often paired with other
    bio-inspired components such as Astrocytes, which it is a goal of this work
    to explore. Relevant models are outlined below. These models each balance
    bio-realism with computational efficiency, with some excelling at
    both. Figure \ref{fig:sn_model_compare} shows where a number of models fall
    when measured by these criteria.
    
    \afigf{figures/sn-model-compare.png}{Comparison of Spiking Neuron
      Models}{fig:sn_model_compare}{1.0}

    %%%%%%%%%%%% Spiking Neuron Models %%%%%%%%%%%%
    \subsection{Leaky Integrate and Fire Neurons}
    The most popular Spiking Neuron model used in the literature is the
    Leaky Integrate and Fire or \Gls{lif} neuron. In this model the state variable is
    represented by an internal voltage, which is increased as current flows into
    the post-synaptic neuron, and charge accumulates. Current leaks from this
    internal reservoir lowering the voltage. Equation \ref{eq:lif} defines the
    behavior.

    \begin{align}
        C \frac{du}{dt}(t) = -\frac{1}{R}u(t)+(i_o(t) + \Sigma w_ji_j(t)) \label{eq:lif}
    \end{align}
    
    Once the state variable u (generally thought of as a voltage) reaches a
    certain threshold the neuron outputs a spike. After this time the state
    variable is reset, and is held at that value for a period known as the
    absolute refractory period. This mimics the behavior of biological neurons
    \parencite{ponulak_2011}.

    %% \todo{Add \Gls{lif} diagram}

    %% \todo{Add 2 tau model from Norse}

    \subsection{FitzHugh - Nagumo Model}
    The FitzHugh–Nagumo is a common \Gls{sn} model that is coupled with
    Astrocytes in a few works that will be explored later (\parencite{postnov_2009},
    \parencite{postnov_2007}). It is based on the highly (computationally) complex
    Hodgkin-Huxley model, with a variety of parameters fixed and others tuned to
    get different firing patterns. \parencite{postnov_2009} outlines the membrane
    voltage dynamics, with a $\tanh$ activation function forming the synaptic
    coupling between neurons (or neuron and Astrocyte).

    \begin{align}
      \epsilon_1 \frac{dv_1}{dt} = v1 - \frac{v_1^3}{3} -
      w_1 \label{eq:fn_neuron}  \\
      \frac{dw_1}{dt} = v_1 + I_i - I_{app} \\
      \tau_s \frac{d_z}{dt} = (1 + tanh(S_s(v_1 - h_s)))(1 - z) - \frac{z}{d_s} \label{eq:fitz_nn}
      \\
      I_{syn} = (K_s - \delta G_m)(z - z_0)
    \end{align}

    The constant $\epsilon_1 = 0.04$ is the time-separation parameter and
    $I_1 = 1.02$ defines the operating regime as excitatory. $H_s$ and $D_s$
    control activation and relaxation of the neuron. $I_{app}$ represents the
    input to the synapse at all sources. To control the post-synaptic behavior,
    dynamics of $z$ are defined, which represents a spike from the neuron
    $z \approx 0$ or $z \approx 1$. The spike $z$ induces a current at the
    post-synaptic terminal, given by $I_{syn}$.

    %% \todo{Add Image of simulation postnov 2009 pg 492}
    %% \todo{Add sections for SRM and Izh}
    
    \subsection{Spiking Neural Network Topologies}
    Spiking networks can take on a variety of topologies, similar to traditional
    ANNs. Feed forward, Recurrent, and a Hybrid of both. One, more specific
    hybrid is a Synfire chain, which is feed forward between sub-populations of
    recurrent networks. Within a standard feed-forward network lateral
    inhibition is often employed, implementing a Winner Take All (WTA)
    configuration. In this case a layer of neurons has feed forward connections
    from the previous layer, in addition to inhibitory connections within the
    layer. The result is the first neuron to emit a spike prevents all other
    neurons within that layer from doing so \parencite{ponulak_2011}.
    
    Synfire chains are of particular interest, as they show a coordination
    within the spiking network. From input to output, in a feed-forward pattern,
    populations of neurons coordinate, passing along a "packet" of neuronal
    activity. It was observed in monkey's, that the precise timing of spikes was
    correlated with behavior, and synfire activity was given as a possible
    explanation \parencite{aertsen_1996}.
    
    %%%%%%%%%%%% Coding Schemes %%%%%%%%%%%%
    \section{Coding Schemes}
    Spiking neural networks, receive and transmit voltage spikes, rather than
    real values as with other \Glspl{ann}. This leads to two important
    questions, how to encode real values as inputs to \Glspl{snn}, and how is
    information passed within an \Gls{snn} encoded. Generally, coding schemes
    used are based on observations in biology \parencite{ponulak_2011}.
    
    Rate-based coding is based on some of the earliest observations of sensory
    neuron activity. As pressure on a tactile nerve was increased researchers
    observed an increase in firing rate of that receptor neuron. This translated
    into rate-based coding in artificial \Glspl{snn} \parencite{ponulak_2011}. Rate-base
    coding couldn't explain certain observations in biology however. Some
    responses were too fast for the neurons involved to estimate the firing
    rate. In addition, dynamic responses have been observed in the primary
    auditory cortex without a changing of firing rate, but instead, firing
    pattern within a sub-population of neurons. More specifically, the relative
    timing of two or more spikes can encode information. This is very
    advantageous, since a spike lasts $\approx 10^-3s$, but relative timing
    between spikes can distinguished down to $10^-8s$ \parencite{ponulak_2011}. This
    greatly increases the bandwidth of communication, and theoretical minimum
    latency. The general trend, is that any neurological system where processing
    speed is especially important, will tend to rely on a spike timing based
    encoding for information.
    
    A variety of temporal coding schemes have been proposed, and
    investigated. Time to first spike, as the name suggests, encodes information
    in the latency between stimulus and the first spike. This is accomplished by
    using a neuron model that has an inhibitory feedback connection, suppressing
    additional spikes. Rank order coding (ROC) represents information in the
    order of spikes within a population of neurons, that each emit a single
    spike. This allows fast information processing, since the time between
    spikes isn't important, and can be very small. This scheme was proposed as
    an explanation for ultra-fast processing in the primate visual
    cortex. Similar to ROC, with latency coding, information is represented in
    the time between spikes of a population of spiking neurons. As with ROC,
    each neuron emits a single spike. This encoding scheme is supported by
    evidence in biological networks, where it has been observed that moving the
    timing of a  single spike by ~10ms can change downstream activity
    \parencite{ponulak_2011}.
        
    %%%%%%%%%%%% Encoding Real Values %%%%%%%%%%%%
    \subsection{Encoding Techniques}
    In general, data being presented to a \Gls{snn} will take the form of real
    values. This is true of images, sound, values from various integrated
    sensors, such as temperature or acceleration. A method is needed that can
    convert these real values into a spike, or spikes, as defined by a coding
    scheme.
    
    \subsection{Poisson Encoding/Decoding}
    A very popular method of encoding real values into spikes, or just to
    generate random spikes, is treating spikes as a Poisson process. In general,
    the rate parameter is either random, or equal to the real value being
    encoded. There are two main approaches for sampling spikes from a Poisson
    distribution. The first calculates the probability of a spike in a given
    time-step $P(k)$, with the value to be encoded as a firing rate. The
    resulting probability 
    is compared to the result of sampling a uniform distribution sampled once
    for each discrete time step. If the uniformly random value is below the
    Poisson probability a spike $spike_i$ is inserted, at time $i$. One potential issue
    with this, relates to numeric precision. If the simulation time step is
    small in comparison with the rate parameter, then the probability of an
    event may be quite small. Go small enough, and the uniformly random number
    won't have enough precision to ever dip below the threshold. In this case no
    spikes would be produced.
    
    %% \todo{ref  https://www.cns.nyu.edu/~david/handouts/poisson.pdf}
    
    \begin{align}
        P(k) = \frac{\lambda^k e^{-\lambda}}{k!} \\ P(1) = \lambda e^{-\lambda} \\
        spike_i = P(1) > X_i \\
    \end{align}
    
    The next, and more numerically stable method, is to use the exponential
    distribution, or Poisson waiting time to sample intervals between
    spikes. Interval times are randomly sampled from the exponential
    distribution.
    
    \begin{align}
      x = rand(0,1) \\
      interval = \lambda e^{-\lambda x} \\
    \end{align}

    $interval$ is computed repeatedly until the desired length spike train has
    been constructed, with $x$ being a uniformly distributed random variable.
    
    %This technique generates a spikes train as shown in the following graph.
    
%    \afig{figures/encode_poisson_sweep.png}{Poisson encoded spike train for real values between 0 and 1}{fig:enc_poisson_sweep}
    
    To determine how different operations process data within a \Gls{snn}, it is
    useful to be able to decode a spike train into real values. This involves
    finding the most likely $\lambda$ parameter for given spike train. Using
    maximum likelihood estimation this is found to be the inverse of the sample
    mean, where $N$ is the duration sampled, and $x_i$ is $0$ or ) $1$ if a
    spike occurred at time $i$.
    
    \begin{align}
        \lambda = \frac{N}{\Sigma x_i}
    \end{align}
    
    This decoding process is subject to error, which improves with sample
    time.
    
    \subsection{Temporal Coding}
    There are a variety of temporal coding schemes. The main distinction from a
    rate-based approach, is the reliance on precise spike timing to encode
    information. Beyond a qualification of these methods, a precise definition
    (or definitions) need to be presented and used during simulation.
    
    \subsubsection{Rank Order Coding}
    Information is encoded in the relative spike times within a population of
    neurons. Encoding itself is fairly straight-forward. The real values that
    make up a single input (such as an image) are transformed into a
    latency. This latency defines the time until a spike is presented to a
    single synapse. Generally larger values are assigned a smaller latency, and
    smaller values a larger one \parencite{delorme_2001}.
    
    Some changes to the network itself have shown success when paired with rank
    order coding. These Include a progressive desensitization, similar to the
    biological concept of shunting inhibition. Equation \ref{eq:roc_activation}
    shows the activation of a neuron, given a modulation factor
    $\alpha in (0,1)$ and a set of incoming connections $a_i$. Subsequent spikes along a
    given input connection have less and less of an effect on the overall
    activation of the neuron \parencite{delorme_2001}.
    
    \begin{align}
        Activation(i) = \Sigma_{j \in [1,m]}\alpha^{order(a_i)}W_{j,i} \label{eq:roc_activation}
    \end{align}

    Where $i$ is the synapse, $order(a_i)$ is the order of incoming spikes, and
    $\alpha$ is a modulation factor.
    
    
    %%%%%%%%%%%% Convolution in SNN %%%%%%%%%%%%
    %% \section{Convolutions With Spiking Networks}
    %% Convolution is not as straightforward to implement in the spiking domain,
    %% when compared to traditional ANNs. There are certainly many ways one could
    %% implement convolution. Including, in the same way they are handled in ANNs,
    %% where neurons aren't explicitly involved. However, the goal is to be
    %% biologically plausible. Observations in the human visual cortex inspired the
    %% tiling window approach to convolution \parencite{wang_2016}. Instead of applying
    %% an activation like ReLU, the activation is a spiking neuron. In addition,
    %% instead of defining convolution as a time-based operation, it is implemented
    %% in space, where the kernel is represented by same-valued sets of weights,
    %% connecting feature maps, represented by spiking neurons
    %% \parencite{mozafari_2018}. Training the weights of a convolution,
    
    %% \section{Reservoir Computing}
    %% A reservoir is a type of recurrent neural network architecture. There are
    %% many variants, but they all share a common theme. The reservoir is made of
    %% multiple non-linear units (neurons). These units are randomly connected with
    %% some probability. Inputs are fed into the reservoir, and a layer of neurons
    %% receiving output from the reservoir act as readouts. The shape and level of
    %% connectivity into and out of the reservoir, as well as the connectivity
    %% within the reservoir are tune-able parameters. There are to major benefit
    %% that have driven research into reservoir computing. One, is that SOTA
    %% results can be achieved with only a single trainable layer, the readout
    %% layer. The reservoir is able to extract features from input data in an
    %% unsupervised way, allowing a single layer to perform classification. In
    %% fact, multiple tasks can be performed on the same input, using the same
    %% reservoir with different readout layers. Second, is that reservoirs have
    %% memory, and can seamlessly handle time-series data \parencite{schrauwen_2007}.
    
    %% \subsection{Liquid State Machines}
    %% LSMs are a type of reservoir computing architecture, built from spiking
    %% neurons. In general they are formed from random between neurons in some a
    %% pool. Neurons of the LSM take in some input, which is then projected to some
    %% transient internal state. Readout neurons tap into a feature representation
    %% derived by the LSM, and produce a linear readout. Classification performed
    %% on this readout layer generally provides good results, and with only one
    %% trainable layer.  LSMs naturally have a fading memory, and lend themselves
    %% to time-series data. \parencite{wang_2016}
    
    %%%%%%%%%%%% SNN Learning Approaches %%%%%%%%%%%%
    \section{SNN Learning Approaches}
    In biology, the changing of synaptic weights is referred to as synaptic
    plasticity, and is considered to be one method that facilitates learning and
    memory. Changes in plasticity can be quick, such as with pulse-paired
    facilitation (STDP-like learning) or more gradual, such as with long-term
    potentiation \parencite{ponulak_2011}.
    
    %%%%%%%%%%%% \Gls{stdp} %%%%%%%%%%%%
    \section{Spike-Timing Dependent Plasticity}
    One of the most widely used learning approaches employed in \Glspl{snn} is
    \Gls{stdp} Traditionally this is an unsupervised approach, and an extension
    of Hebbian learning. Classic Hebbian learning can be summed up by ``if
    neurons fire together, they wire together'' meaning correlated firing is
    rewarded by increased connection (or weight). \Gls{stdp} gets a bit more
    specific, and rewards causation. This means a connection is strengthened if
    a pre-synaptic spike is followed by post-synaptic firing, and weakened if a
    pre-synaptic spike follows post-synaptic firing. Eq \ref{eq:classic_stdp}
    describes classic \Gls{stdp} \parencite{tavanaei_2019}.

    \begin{align}
        \Delta W =
        \begin{cases}
          Ae^{-\frac{|t_pre-t_post|}{\tau}} & t_{pre} - t_{post} \leq 0, A > 0
          \\ Be^{\frac{|t_pre-t_post|}{\tau}} & t_{pre} - t_{post} > 0, B < 0
       \end{cases} \label{eq:classic_stdp}
    \end{align}

    Swapping the conditions in Equation \ref{eq:classic_stdp} leads to
    ``anti-STDP'' (aSTDP) where correlated spikes result in reduced weight. This
    is useful in some learning rules extending \Gls{stdp}.

    \subsection{Properties of Classic STDP}
    Classic unsupervised \Gls{stdp} has seen much success in pattern recognition
    tasks, but does fall short in certain areas. In general \Gls{stdp} converges to
    recognize repeating spatio-temporal patterns, and complex patterns can be
    recognized by a single neuron \parencite{tavanaei_2019}. \Gls{stdp} does have trouble
    converging in order to differentiate similar (but distinct) inputs
    \parencite{vigneron_2020}. \Gls{stdp} also tends to drive weights to favor early
    spikes, using those early spikes as the main factor in pattern
    recognition. There has been extensive statistical analysis of \Gls{stdp}, both due
    to it's nature, and since it is known the the brain performs Bayesian
    analysis of sensory stimuli \parencite{tavanaei_2019}. As part of this analysis
    \parencite{nessler_2009}  determined that \Gls{stdp} coupled with a WTA
    configuration of neurons, was able to implement the Expectation-Maximization (EM)
    algorithm. In addition, it has been shown that classic \Gls{stdp} will generally
    drive weights to either maximally inhibitory, or maximally
    excitatory.

    %% \todo{Expand stat analysis}.

    %% \todo{cite https://arxiv.org/pdf/1611.03000.pdf}.

    \Gls{mlp} networks satisfy the Universal approximation theorem, meaning they
    can approximate any function between their inputs and outputs, arbitrarily
    well given the right topology. \Glspl{snn} share this property, through their
    ability to approximate Sigmoid neurons using rate-based coding. It isn't
    clear yet whether \Glspl{snn} have this property outside of this
    approximation, or using coding schemes other then rate-based coding \parencite{legenstein_2005}. When considering STDP, this raises a similar
    question, what subset of all the possible weight values can be learned using
    STDP, or STDP-like rules? \parencite{legenstein_2005} explores this
    question. For a mapping to be learned, it must be stable. For
    STDP, this means the weights are bi-modal, with values either being at their
    minimum or maximum. Researches found that on average, a supervised variant
    of \Gls{stdp} (with teaching signal to force output firing) can converge to
    a desired weight distribution, if this distribution is bi-model.

    \subsection{Weight Initialization}
    
    With \Gls{stdp}, weight initialization is an important step, which can
    impact \Gls{snn} performance, and ability to learn. It is common practice to
    use a normal distribution to initialize weights. There are some pittfalls to
    this approach. If the mean ($\mu$) is too small, some weights may result in
    ``dead'' synapses, which will never result in a neuron firing. Conversely
    choosing a variance ($\sigma$) that is too large, will result in some
    synapses (at random) overpowering others, and lead to poor convergence
    \parencite{vigneron_2020}. Generally, the best approach is to have a value for
    $\mu$ which allows neurons to fire, with $sigma$ being small, compared to
    $\mu$.

    \subsection{Reward-Modulated STDP}
    
    Recently, there has been some success with a supervised variant on \Gls{stdp},
    known as reward-modulated \Gls{stdp} (R-STDP). In this case a Deep Convolutional
    Spiking Neural Network was applied to the \Gls{mnist} digit recognition
    task. Early layers of the network were trained in an unsupervised manor,
    with normal STDP. Later layers were updated using reward-modulated
    STDP. That is, if the output was correct, \Gls{stdp} was applied. If not,
    anti-STDP was applied. This learning approach has basis in biology, with the
    reward modulation mimicking the activity Dopamine and Acetylcholine
    (ACh). As a proof of concept, researchers applied their learning method to a
    shallow network, with a single trainable layer. They used rank order coding,
    and at most one spike per neuron. The output class is determined by which
    neuron in the output layer exhibits a spike first. This architecture gave
    passable results, but wasn't well suited to multi-layer training.
    
    For their deep architecture, shown in figure \ref{fig:rstdp_dcnn} better
    results, an accuracy of 97.2\% which is on par with the state-of-the-art was
    achieved \parencite{mozafari_2018}.
    
    \afig{figures/rstdp_dcnn_arch.png}{Spiking DCNN
      Architecture}{fig:rstdp_dcnn}{0.6}

    R-STDP improves classification on similar samples, and trains the network to
    recognize discriminate features, instead of repeating ones. The result, is
    a learning rule capable of handling similar but distinct inputs.

    \subsection{ReSuMe}
    The ReSuMe learning rule is another variant of STDP, which is very similar
    to the R-STDP update rule defined by \parencite{mozafari_2018}, though with some
    important differences. With ReSuMe, a single neuron and set of synapses are
    initialized with some weight values randomly. There is a 2nd similar neuron
    which implements the desired behavior. Inputs are provided to both neurons,
    and at each time step weights are updated according to the difference
    between the teaching neuron, and neuron in training. The weight update can
    be described by \ref{eq:resume_stdp}.

    \begin{align}
      \Delta w = \Delta W_{STDP}(S_{in}, S_{d}) + \Delta W_{aSTDP}(S_{in}, S_{o}) \label{eq:resume_stdp}
    \end{align}

    Where $\Delta W_{STDP}$ and $\Delta W_{aSTDP}$ are the weight changes
    associated with classic \Gls{stdp} behavior, and anti- \Gls{stdp}
    behavior. In each term, the input spike is considered, but first with
    $S_{d}$ which is the desired output coming from the ideal neuron. Second,
    with the actual output \parencite{mozafari_2018}. \parencite{ponulak_2010} was
    the first to develop and explore this learning rule, and they were able to
    prove convergence, from an untrained neuron, to one of similar configuration
    with a desired behavior. The use of an additional synaptic input and the
    specifics of Equation \ref{eq:resume_stdp} creates a unique extension to
    classic STDP. Weights can be updated without one of the down-stream neurons
    firing, and activity on a third synapse is influencing plasticity. Beyond
    that, this is yet another learning rule, that has shown success when
    introducing a third factor to \Gls{stdp}.

    %%%%%%%%%%%% ReSuMe %%%%%%%%%%%%
%    \subsection{Teaching-Signal Rules}
%    \todo{populate this section} \parencite{ponulak_2011} - 417
    
    %%%%%%%%%%%% Back Propagation %%%%%%%%%%%%
    \subsection{Back Propagation}
    
    Back-propagation is traditionally difficult to implement in \Glspl{snn}, as the
    spike function is not differentiable. There are some shortcuts proposed,
    such as using the membrane voltage derivative instead of spike
    output. However applying backprop may not be the best approach. The brain
    doesn't (as far as is known) have a method of back-propagating errors. It
    has been proposed to adapt the network architecture to allow rewards at the
    output to affect all portions of the network. A hybrid reward-STDP approach
    shows promising results \parencite{tavanaei_2019}.
    
    %%%%%%%%%%%% Deeper Networks %%%%%%%%%%%%
    %% \section{Deeper Networks}
    %% Deep learning is still largely unexplored in \Glspl{snn}, as well as multi-layer
    %% learning in the deep architectures that have been explored. Motivating
    %% further exploration is both the success of deep Artifical Neural Networks,
    %% and the observation that the mammalian brain relies on a deep architecture
    %% for visual tasks such as detection and recognition \parencite{tavanaei_2019}.
    
    \section{Notable Experiments}
    A number of papers have shown promising results on \acrfull{sota} recognition tasks,
    including \Gls{mnist} written digit classification \parencite{mozafari_2018}. As well as
    facial recognition \parencite{delorme_2001}. These papers, among others use
    hand-crafted Difference of Gaussian's filters in early convolution
    layers. End-to-end trainable \Gls{snn} architectures, for image processing tasks
    appear to be lacking in literature.
    
    \parencite{mozafari_2018} has shown that a mixture of \Gls{dog} filters, STDP, and
    reward-modulated \Gls{stdp} can achieve \acrshort{sota} results on \Gls{mnist}. The
    network wasn't end-to-end trainable, but did have a neuron-based readout,
    requiring no external classifier, such as Support Vector Machines (SVM).
    
    \parencite{delorme_2001} uses \Gls{roc} along with some architectural and learning rule
    modifications, to better support the coding scheme. The goal of this paper
    is to show that features can be extracted from \Gls{roc} encoded images after \Gls{dog}
    filtering. 3x3 \Gls{dog} filters were used, followed by ROC encoding, and then 32
    maps generated using an 11x11 receptive field. Connections from \Gls{dog} maps to
    \Gls{lif} neurons are organized in such a way to implement convolution, but still
    be trainable. If weights are updated in one region, due to training, the
    weights are simultaneously updated in the other regions. In addition, there
    is inhibition between feature maps. If a neuron fires within one map, the
    corresponding neuron is inhibited in the others. The goal was to force maps
    to learn unique features. Results were promising, and activation of \Gls{lif}
    neuron maps showed selectivity for image features such as blobs and
    contours.
    
    %%%%%%%%%%%% Astrocytes Intro %%%%%%%%%%%%
    \section{Introduction to Astrocytes}
    Astrocytes are a type of glial cell found in mammalian brains. Their
    structure and function are still the topic of cutting edge research
    today. It is known that they are vital for normal brain functions, including
    cognition and behavior \parencite{mederos_2018}. In the human brain astrocytes
    are known to tightly wrap many synapses, as well as dendrites and cell
    bodies. Since the Astrocytes engulf the synapic cleft, they are ideally
    placed to control extracellular neurotransmitter and ion
    concentrations. There is substantial evidence that they regulate
    extracellular \kp, which is required for propagation of action potentials
    through neuron bodies. \Gls{gaba} transporters in astrocyte cell membranes
    serve to clean up neurotransmitters, and can help mitigate exitotoxcity. It
    should be noted that Astrocytes do not wrap, or even influence every
    synapse, and that their density varies widely depending on the region of the
    brain. More interesting, their morphology can change throughout life, and in
    response to other bodily functions, such as food intake \parencite{mederos_2018}.

    From a functional point of view, at a high level, Astrocytes listen and
    respond to activity at the synaptic clefts of multiple synapses. Activity on
    a given synapse leads to a local response, in the Astrocyte process that
    surrounds that synapse. Local responses from many synapses contribute to a
    global cell-body response. This global response may propagate to other
    Astrocytes, and the responses from other Astrocytes may influence, or even
    trigger a global response of an Astrocyte \parencite{min_2012}. At a given
    synapse, Astrocytes are sensitive to pre and post synaptic potentials, via
    neurotransmitter or ion uptake. In general, the neurotransmitter in question
    is \Gls{glu} for excitatatory transmission, and \Gls{gaba} for
    inhibitory. Depolarization via \kp at the post-synaptic terminal can be
    sensed as well, through the voltage change, and uptake of \kp by an
    Astrocyte. The Astrocyte will then respond with one or more
    Gliotransmitters. These include \Gls{glu}, D-serine, \Gls{atp} or
    \Gls{tnfa}. These gliotransmitters have different effects depending on where
    and when they are released.

    \begin{itemize}
      \item \Gls{glu}: When released to the pre-synaptic neuron \Gls{glu} generally
        results in an increase in Probability of Release (PR). This equates to
        an increase in the weight for a computational model. At the
        post-synaptic terminal, \Gls{glu} release results in depolarization, or
        a so called slow inward current (SIC) \parencite{pitta_2016}.
      \item D-Serine: this neurochemical gates the NDMA receptors, and
        subsequently gates \Gls{ltp}/\Gls{ltd} \parencite{mederos_2018}.
      \item \Gls{atp} has a depressive affect, opposite to \Gls{glu}, and can
        propagate beyond the synaptic cleft \parencite{mederos_2018}.
      \item \Gls{tnfa} results in an increase in the number of post-synaptic
        surface \Gls{glu} receptors, and a decrease in \Gls{gaba} receptors
        \parencite{chung_2015}.
    \end{itemize}

    Gliotransmitter release from an Astrocyte is dependent on the integration of
    synaptic activity over 100s of milliseconds (for a local response)
    \parencite{pitta_2016} to 10s of seconds for a whole cell response
    \parencite{mederos_2018}. This integration is mediated by a variety of input
    pathways, some of which are more thoroughly understood than others. These
    input pathways generally converge to provide an increase to intracellular
    \ca concentration, within an Astrocyte. Though there is some debate surround
    \ca signaling within astrocytes \parencite{mederos_2018}. Initial results of
    in-vitro experiments showed a slow \ca response in Astrocytes following
    intense neuronal activity. This showed that while Astrocytes were active,
    they could not exert rapid or granular control over information flow at the
    synapse. More recent research with more advanced methods has shown that
    there are synapse local \ca responses that are much quicker and respond to
    lower levels of activity \parencite{araque_2014}. The connection between local
    and cell level response is mediated by \Gls{cicr}. As local activity
    increases beyond a certain threshold, the \ca concentration causes
    exponential \ca release from the \Gls{er}. This \Gls{cicr} propagates like a
    wave, and can reach the main cell body, other astrocyte processes, and even
    other Astrocytes \parencite{manninen_2018}.

    %% TODO: Add in the \Gls{cicr} \Gls{er} picture, it shows how \ca waves can
    %% propagate to the cell soma
    %% TODO: More detail here

    \section{High Level Roles for Astrocytes In Biology}

    There are a variety of theories as to the functional role of Astrocytes.
    Some of these are speculation based on general themes of Astrocyte behavior,
    and others specific and highly grounded in experimental data.

    %% Astrocytes are Master Integrators
    Astrocytes have been shown to act as integrators of synaptic activity. Early
    studies discovered that Astrocytes become active in response to intense
    neuronal firing. Subsequent reasearch shows that this response, represented
    by \ca concentration, was highly complex in time
    \parencite{araque_2014}. This complex global response which happens on a seconds
    timescale is preceded by \ca activity in Astrocyte processes local to
    synapses. Though there isn't direct evince of this, it is thought that these
    local \ca increases (sometimes called ``puffs'' or ``sparks'') propagate
    their way to the main cell body, and induces the response observed at the
    cell body \parencite{araque_2014}. In this way Astrocytes can locally integrate
    neuronal activity temporally, on a 100s of milliseconds time-scale. These
    local events can then be temporally and spatially integrated on a seconds
    timescale. There is some evidence, when looking at Astrocyte \Gls{mglur},
    that along a common astrocye process, there are clusters of
    receptors which form a local region. Each of these regions is associated
    with synaptic terminals, and may integrate their activity at different
    spatial and temporal time-scales as their neighbors
    \parencite{pitta_2012}. Furthermore, experiments with computational models show
    that varying astrocyte modalities (as is the case in the brain) leads to
    different patterns of signal transmission. This could indicate that
    Astrocytes in different areas of the brain perform different functions
    \parencite{pitta_2012}.

    %% Astrocytes facilitate long-range spatial influence
    A single Astrocyte, through its end-foot processes can influence many
    synapses simultaneously. In the brain, Astrocytes are physically distributed
    via a mechanism called contact spacing, where their end-foot processes
    connect at the periphery of an Astrocyte's domain \parencite{pitta_2012}. This
    spacing is not always uniform, some micro-domains are formed favoring a
    neuron signal pathway, with adjacent astrocyte connections observed to be
    absent. One example is in the Ferret visual cortex, where astrocytes (like
    Neurons) form receptive fields on the visual input \parencite{pitta_2012}. This
    finding lines up well with the work of \parencite{gordleeva_2021}, where
    Astrocytes were used as working-memory units, forming receptive fields for
    visual information.

    Signals propagate in these Astrocyte networks in a few different
    ways. One pathway, involves \ipt diffusion through Astrocyte gap junctions. Once across, sufficiently high levels of \ipt cause
    \Gls{cicr}, and in turn Astrocyte waves. \ca may also diffuse across gap
    junctions in the same way. This propagation is dubbed \ca waves, due to
    the wave-like nature of the propagation. In addition to the gap-junction
    mediated effects, \gls{atp} released from an Astrocyte may diffuse
    extra-cellularly, and influence other Astrocytes \parencite{amiri_2013}.
    
    %% Astrocytes Modulate STP/STD
    \parencite{pitta_2012} explores the pre-synaptic \Gls{glu} mediated Astrocyte
    stimulation and response loop, as a mechanism for modulation of short term
    plasticity. This loop is characterized by a positive feedback loop, where
    \Gls{glu} release from the pre-synaptic neuron stimulates the Astrocyte, and
    leads to additional \Gls{glu} release. If the \acrshort{pr} at that synapse
    is low, then there should be sufficient neural resources to support
    transmission, and the Astrocyte is able to gain-up the response. If
    \acrshort{pr} is high, then the Astrocyte response leads to short-term
    depression, as neural resources are exhausted quickly.

    %% Astrocyte Modulation of \Gls{ltd}/\Gls{ltp}
    Astrocytes have been shown to play a key role in the bio-chemical pathways
    that lead to \Gls{stdp} at neuronal synapses, and can gate \Gls{ltd} and \Gls{ltp} via D-Serine
    release \parencite{manninen_2019}.

    Both \Gls{ltd} and \Gls{ltp} are controlled by the \Gls{ndma}
    receptor. Through gating factors this receptor is able to respond to the
    co-incidence of pre and post-synaptic action potentials. Figure
    \ref{fig:astro_plastic} shows Astrocyte involvement in this process. In the
    case of \Gls{ltd}, when a post-synaptic depolarization is followed closely
    by pre-synaptic depolarization a signaling pathway proceeds, leading to
    \Gls{glu} release by the astroctye, which in turn triggers \Gls{ndma}
    receptors in the pre-synaptic neuron. The influx of ions leads to a decrease
    in pre-synaptic neurotransmitter release probability. Beyond passively
    participating in the normal \Gls{stdp} process, astrocytes may gate it's
    activity, or reverse the polarity, implementing anti-STDP
    \parencite{min_2012}. This is achieved through gliotransmitter release, such as
    D-serine, which provides a necessary gating factor for \Gls{ndma} receptors
    to open. \Gls{glu} released by astrocytes in response to pre and
    post-synaptic activity may stretch or shift the traditional \Gls{stdp} curve
    in both experimental results, and computational models \parencite{pitta_2016}.

    Astrocytes can modulate the concentration of their surface glu transporters,
    this in turn modulates the level of glu spill-over, beyond the synapse. This
    increases the excitability (depolarizing neurons partially towards firing)
    at the post-synaptic terminal for a local region of synapses. Astrocytes may
    also respond to \Gls{glu} activity (as sensed by transporter uptake) by
    releasing more \Gls{glu}, or by releasing \Gls{atp}, based on signaling
    frequency. Within the Hippocampus, it has been observed that \Gls{gaba}
    release from a pre-synaptic neuron can be taken up by \Gls{gaba}
    transporters on an Astrocyte's surface. This leads to an increase in \na /
    \ca within the Astrocyte, and the eventual release of \Gls{atp}. \Gls{atp}
    acts as an ihibitor, down-regulating excitatory transmission
    \parencite{mederos_2018}.
        
    \afig{figures/astrocyte_ltd_ltp.png}{Astrocyte-MediatedPlasticity}{fig:astro_plastic}

    \section{Biochemistry Inspires Computational Models of Various Pathways}
    Over the last 30 years there have been experiments and studies surrounding the
    role of Astrocytes in the mammalian brain. This research has lead to the
    identification of a variety of signaling pathways, along with some
    quantitative data. These new insights in the neuroscience field, have lead
    to a wave of computational models, which attempt to either mimic
    Neuron-Astrocyte interactions, or extract some computational benefit by a
    simplified, but still very bio-plausible model. The first clue to Astrocyte
    involvement in computation was the observation of \Gls{glu} released from
    the pre-synaptic neuron, and subsequent increase in \ca concentration
    within astrocyte cell bodies and foot processes \parencite{manninen_2018}. Around
    2010, new experimental data emerged from in-vivo studies, which furthered
    the functional understanding of Astrocyte signaling
    \parencite{manninen_2018}. Some pathways are responsible for modulating
    plasticity while other lead to transient changes in neuron dynamics. A key
    step in developing a computational model, that captures the key features
    involved in information processing, is understanding the biological
    pathways.

    %% TODO: Add pathway figure

    Figure \ref{fig:astro_pathways} shows the possible signaling pathways at the
    Tripartied synapse. Each pathway has some underlying research into the
    behavior at that pathway, either in vitro or in vivo. These experimental data
    have led to a variety of computational models, which share a common theme but
    differ in complexity, level of bio-realism, and computational efficiency.

    \subsection{Neuron-Astrocyte Pathways - Presynaptic to Astrocyte}

    One of the first experimental observations of Astrocytes was a transient
    increase in \ca in response to high levels of neuron activity. One of the
    main pathways mediating this response is the Pre-synatic \Gls{glu}
    pathway. Excitatory cortical neurons release \Gls{glu} in response to input
    stimulus. \Gls{glu} binds to the G-couple \Glspl{mglur} on the surface
    of an astrocyte. This sets in motion a cascade involving the \ipt second
    messenger, which ultimately leads to \ca release in the astrocyte soma
    \parencite{pitta_2012}. \ipt induced \ca release from the \Gls{er} leads to a rapid
    breakdown of \ipt, creating a kind of local \ca spike. This \ca
    concentration can integrate within the cell soma, but does degrade due to
    the activity of pumps at \Gls{er} surface \parencite{pitta_2012}.

    To model this \parencite{pitta_2009} uses a three variable approach, which
    extends the Li–Rinzel model with bio-realistic \ipt dynamics. In general \ca
    concentration (when considering the \ipt mediated pre-synaptic pathway only)
    is dependant on two internal activites, and one external. Internally, there
    is a $J_{leak}$ factor, describing a differential based leak from the \Gls{er}
    into the Astrocyte Cytosol. To override this leak, and maintain the
    differential, SERCA pumps move \ca into the \Gls{er}. This is noted by
    $J_{pump}$. $J_{chan}$ Accounts for the flux of \ca from the \Gls{er} into the
    cytoplasm due to \ipt levels, or \Gls{cicr}, depending on the \ca concentration
    \parencite{pitta_2009}. In this case \ipt levels are a function of \Gls{glu}
    release from the pre-synaptic neuron. \parencite{pitta_2016} extends this model,
    including the Glio-transmitter release from the Astrocyte in response to
    \ca transients. In their work a spike of GT is released (similar to a
    neuron) when \ca concentration reaches some threshold. It should be noted,
    that the \ipt/\ca spiking response (\ipt -> \ca release -> \ipt cleanup + \ca
    cleanup) is independent of the glio-transmitter release, meaning there can be
    sub-threshold \ca spikes).

    The above outlines the common theme for presynaptic Astrocyte
    modulation. The general form of the pathway outlined above is shared by
    \parencite{postnov_2009}, and \parencite{wade_2011}. Though there are
    variations among modeling approach which will be covered in further detail.

    \subsection{Neuron-Astrocyte Pathways - Postsynaptic to Astrocyte}
    Another common pathway that is explored both in neuroscience and
    computationally is the ``fast'' post-synaptic pathway \parencite{bassam_2015}. In
    this pathway, the post-synaptic neuron fires, and subsequently releases \kp
    as part of the depolarization process. This \kp spillover is quickly shuttled
    into the Astrocyte, and causes depolarization across the Astrocyte's
    membrane. Voltage-gated channels on the \Gls{er} then lead to \ca release into
    the cytoplasm. It is considered fast pathway, because the effect of \kp is
    direct, vs. the pre-synaptic pathway involving a second messenger
    \parencite{bassam_2015}.

    The post-synaptic pathway is modeled differently across reviewed
    literature. \parencite{bassam_2015} uses the Gerstner Spiking Response
    Model, and models the \ca response to a the post-synaptic neuron firing
    using Equation \ref{eq:kp_path_srm}. This characterizes the response to a
    single post-synaptic spike as a portion of an exponential function, similar
    to the \Gls{psp} generated by a pre-synaptic spike. Other works, such as
    \parencite{postnov_2007} incorporate the \ca response into a FitzHugh–Nagumo
    neuron model, using equation \ref{eq:kp_path_fn}, and referencing Equation
    \ref{eq:fitz_nn}.

    \begin{align}
        PS_{mod} =
        \begin{cases}
          \alpha W + W * e^{\frac{-t - t^k_{i}}{\tau_s}} & t^k_{i} < t < t^k_{i}
          + d \\
          \alpha W + W * e^{-\alpha (-t - t^k_{i})} & otherwise
       \end{cases} \label{eq:kp_path_srm}      
    \end{align}

    Where $W$ is the sum of all synaptic weights $t^k_i$ is the time of the last
    spike on the $i$-th synapse. $\alpha$ is a constant, and $t$ is time.

    \begin{align}
      \tau_c \frac{dc}{dt} \alpha (r + \alpha w2 + \beta S_m) \label{eq:kp_path_fn}
    \end{align}

    Where $\tau_c$ is a time-constant, $r$ controls the initial state, and
    $\alpha W2$ represents the effect of post-synaptic depolarization on
    \ca. $W2$ is analogous to $W1$ from equation \ref{eq:fn_neuron}, which
    defines the FitzHugh - Nagumo model.

    Though the exact dynamics of the response are neuron dependent in the
    literature, the general form of the response is the same, and can be
    represented using an exponential function.
    
    \subsection{Calcium, and Other Messenger Dynamics}
    \ca, \ipt and \kp are the main substances involved in signaling pathways
    within the Astrocyte. What isn't immediately obvious unless one is paying
    close attention to the modeling (or more bio-realistiic) equations, is that
    there are spikes and thresholds within these internal Astrocyte
    concentrations. Neuroscience experiments involving monitoring of Astrocyes
    revealed oscillations, which are more intense and propagate further
    depending on Activity. These oscillations are the consequence of non-linear
    negative feedback, leading to fast cleanup, and spike-like behavior
    \parencite{postnov_2009}. The second messenger \ipt has a similar behavior, with a
    threshold and $tanh$ (with decay factor) resulting in spike-like behavior
    \parencite{postnov_2009}.

    \parencite{wade_2011} cites similar behavior, except it is the output GT release
    that exhibits spiking behavior, where there isn't spiking observed in the
    \ca. In this model, like with \parencite{postnov_2009} there is a threshold
    gating GT release.

    \parencite{pitta_2009} introduces the AM/FM variants on the $J_{pump} -
    J_{chan} - J_{leak}$ model, which exhibits spiking and non-spiking \ca
    dynamics, depending on choice of parameter. This is the same model outlined
    in \parencite{wade_2011}.

    It is also well established that the \ipt pathway operates on a slower
    time-scale, then more direct pathways such as \Gls{atp} or \kp
    \parencite{postnov_2009}, \parencite{bassam_2015}.

    \section{Evolution of Astrocyte Models}
    \parencite{manninen_2018} provides a very insightful review of astrocyte models
    from 1995 until about 2017, and, more importantly an analysis on the origin
    of models for many papers considered. In this work, four different core
    Neuron-Astrocyte models have been identified in the literature over the
    time-period considered. These astrocyte models are modified throughout the
    works, to show different effects.

    \subsection{Foundational Astrocyte Models}
    \parencite{manninen_2018} Grouped models from early works that appeared to
    precede other models (qualitatively) through 2017. These groups were formed by
    Lin and Rinzel-like models, DeYoung and Keizer, and Hofer. Lin-Rizel and
    DeYoung models share a common high-level definition, with the details
    differing from each-other and across derivative works. These models consist
    of three terms which define \ca dynamics, $J_{pump}$ to denote the flux of
    \ca to maintain a gradient between Astrocyte cytoplasm, and
    \Gls{er}. $J_leak$ denotes a constant leak factor from \Gls{er} to
    cytoplasm. Treating these as simultaneous differential equations, overall
    \ca response $\frac{dc}{dt}$ can be computed, see Equation
    \ref{eq:lr_dk_astro}.

    \begin{align}
      \frac{d[Ca^{2+}]}{dt} = J_{chan} + J_{leak} -
      J_{pump} \label{eq:lr_dk_astro} \\
    \end{align}

    Hofer on the other hand, defines an Astrocyte model in a more bio-realisitc
    way, modeling internal chemical signaling using second-order differential
    equations, and including messengers beyond just \ipt. There is also an
    emphasis placed on Astrocyte to Astrocyte communication through gap
    junctions, which is unique to this model. Overall Hofer-like models were the
    least prevalent in existing literature \parencite{manninen_2018}.

    
    \section{Astrocyte Mediated Effects In the Literature}
    \subsection{Glio-Transmitter Release From Astrocyte}
    To signal back to neurons, Astrocytes release various gliotransmitters,
    which were outlined above. In general, this release is dependent on the
    concentration of \ca local to the synapse. Input pathways converge with
    \ca concentration, and then divergent effects are observed via the release
    of multiple gliotransmitters.

    \Gls{atp} release from astrocytes can travel between itersiticial spaces and
    effect synaptic transmission at physically local
    synapses. \parencite{postnov_2009} explores the phenomenon of
    hetero-synaptic suppression, which is mediated by \Gls{glu} and \Gls{atp}
    release. Their experiment depends on an existing computational model, with
    some modifications to support \Gls{atp} release. A synapse local to an
    Astrocyte is potentiated in the short term via \Gls{glu} release. Over a
    longer time-scale, \Gls{atp} is released from the Astrocyte and diffuses to
    a neighboring synapse, decreasing synaptic PR. This signaling can bypass
    both synaptic connections, and Astrocyte gap junctions, effecting physically
    local synapses and neurons.

    \subsection{Facilitation of \Gls{ltp}/\Gls{ltd}}
    Astrocytes are believed to facilitate, or even implement \Gls{ltp}/\Gls{ltd} in
    biological synapses. Retrograde signals responsible for synaptic PR changes
    are picked up by Astrocytes, and result in the release of \Gls{glu} and D-Serine,
    which act on pre-synaptic NMDA receptors. Activation of these receptors
    leads to long-lasting changes in synaptic strength \parencite{min_2012}.

    \section{Propagation of Astrocyte Signals Internally}
    Early research into Astrocyte behavior showed cell-wide \ca transients in
    response to intense activity, which would in some cases propagate to
    neighboring cells. \parencite{manninen_2018} reports that it wasn't until around
    2010 that pharmacological tools became precise enough to discover smaller
    variations in \ca at Astrocyte processes, in response to lower levels of
    neuronal activity. These smaller variations are thought to be integrated,
    and lead to a larger cell-level \ca response, though there is not direct
    evidence of this \parencite{araque_2014}. In addition, the mechanism thought to
    trigger the switch from local to global responses is \ca-Induced \ca
    release (\Gls{cicr}) which is the mechanism responsible for spike-like behavior
    within Astrocyes.

    Astrocytes also form their own networks, separate from the connections of
    neurons. Gap junctions between cells can pass various molecules including
    ions and secondary messengers. Gap junctions are not evenly distributed, or
    random, meaning they form meaningful connections between specific
    astrocytes. These networks appear to form non-overlapping territories, where
    groups are interconnected, but distinct from other
    groups \parencite{mederos_2018}. Astrocytes don't have long-reaching
    projections like the axons of neurons, limiting \ca propagation to a few
    $\mu m$ \parencite{hofer_2002}. The shapes of these networks are varied,
    and in the visual cortex consist of between 2 and 10 Astrocytes. In
    addition, Astrocyes are generally not found un-coupled in the brain, further
    supporting the significance of Astrocyte networks \parencite{postnov_2009}. The
    specific molecules that generally diffuse across gap junctions are \ipt and
    \ca, with only \ipt being modeled in some cases \parencite{pitta_2012}. In any
    case the effect is generally described as excitable, regardless of the
    molecule that diffuses \parencite{gordleeva_2021}, \parencite{pitta_2012},
    \parencite{postnov_2009}.

    The functional role, and information processing implications of Astrocyte
    networks isn't yet well understood. There are a few consequences of
    Astrocyte networks that have been observed. First, these
    networks provide additional pathways for \ca waves to propagate, and in
    some cases form cycles, where astrocyte waves return to their place of
    origin. In addition, under intense neural activity, ANs support far reaching
    synaptic modulation, through wide reaching \ca wave propagation \parencite{postnov_2009}. There have been attempts at
    modeling astrocyte network, with variying degrees of
    complexity. \parencite{postnov_2009} developed a random procedural algorithm for
    generating a 2D bio-plausible Astrocyte network. The algorithm is as follows,

    \begin{enumerate}
        \item From the center of each astrocyte, produce a random number between
          m and n of branches pointing outward in uniformly random directions.
        \item If any branch either intersects another of the same astrocyte, or
          is closer to another astrocyte than its ancestor (could be its
          astrocyte, or another branch), then the branch is rejected, and
          another random branch is generated in an attempt to take its place.
        \item For any surviving branches generate a random number of next-level
          branches between 1 and k.
        \item repeat 2, 3 until a maximum branching level $N_{max}$ is reached.
    \end{enumerate}

    Using the novel Astrocyte connection algorithm developed,
    \parencite{postnov_2009} was able to reproduce in simulation some important
    observations of Astrocyte signaling. First, a low-level of synaptic
    activity elicited a strictly local response. Increasing the level of
    stimulation results in a \ca wave, which spans many astrocytes, and
    propagates out from its source. With sufficiently active neurons, the
    resulting \ca wave is shown to propagate out from an Astrocyte, and then
    eventually return to its source. The main goal here was to match
    experimental results outlined neuroscience research, and there was minimal
    discussion as to a functional role.

    Other works, such as \parencite{gordleeva_2021} do demonstrate functional
    benefits of inter Astrocyte communication leverage this same property to
    implement working memory, with Astrocytes maintaining \ca activity, and
    influencing neurons in response to a cue.

    %% \section{Related Work - Organized By Paper and Author}
    
    %% There have been a variety of works attempting to bring an Astrocyte model
    %% into ANNs. In general these models were either not biologically inspired
    %% enough, or too complex in their modeling of Astrocyte activity for real-time
    %% and scalable use \parencite{bassam_2015}. These models generally consisted of,
    %% in the simple case, a supervising element per neuron. If this neuron emits
    %% sufficiently many spikes in some time-frame, the Astrocyte will increase the
    %% downstream weighs for that neuron. Alternatively if a neuron is inactive,
    %% measured by fewer than some number of spikes within a time-frame, downstream
    %% weights will be decreased \parencite{mesejo_2015}. Models like this one
    %% however, have no concept of \ca signaling, and can't easily support
    %% Astrocyte networks.
    
    %% \parencite{bassam_2015} proposes a Spiking Response Model which uses \ca
    %% concentration as an internal state variable, and incorporate multiple
    %% pathways to affect a change in \ca. Figure \ref{fig:srm0} gives a
    %% high-level overview of this model. Spikes from some number of neurons enter
    %% the synapse. A key difference in this work is the emphasis placed on the
    %% difference in time-scales between the Pre-synaptic slower \Gls{glu} -> \ipt -> \ca
    %% pathway, vs the postsynaptic faster \kp -> \ca pathway.
    
    %% \begin{align}
    %%     V_j(t) = \eta(t - t_j) + \Sigma^{N_{i+1}}_{i=1}W_{ij}\Sigma^{K_i}_{k=1}
    %%     \epsilon(t - t_i^k - d_{ij})
    %% \end{align}
    
    %% Where $V_j(t)$ represents the membrane potential of a neuron, with $K_i$
    %% spikes coming from each of $N_{i+1}$ neurons. The voltage change associated
    %% with each spike is the product of the weight $W_{ij}$ and a response kernel
    %% $\epsilon$.
    
    %% \afigs{figures/snn_model.png}{Spiking Response Model}{fig:srm0}{0.6}
    
    %% The spike response kernel $\epsilon$ is defined by Equation \ref{eq:srm_eps}
    %% \begin{align}
    %%     \epsilon (s) = \frac{s}{\tau s} e^{\frac{s}{\tau s}}
    %%     H(s) \label{eq:srm_eps} \\ s = (t - t_i^k - d_{ij}) \\ H(s) =
    %%     \begin{cases} 
    %%       1 & s >= 0 \\ 0 & s < 0
    %%    \end{cases}
    %% \end{align}
    
    %% Where $H(s)$ can be recognized as the Heavy-side Step Function. After a
    %% spike, the neuron enters a refractory period.
        
    %% \begin{align}
    %%     ca^{2+} = r + S_{mod} + PS_{mod} \label{eq:srm_astro_ca} r = 0.31
    %% \end{align}    
    
    %% %%%%%%%%%%%%\Glspl{snn}on an FPGA %%%%%%%%%%%%
    %% \section{SNNs on an FPGA}
    %% \parencite{cassidy_2017} have shown the \Gls{lif} based\Glspl{snn}can be implemented on FPGA
    %% using adders to numerically integrate discrete spikes. \Gls{stdp} weight updating
    %% was also included in the implementation. Figure \ref{fig:fpga_lif} shows the
    %% hardware implementation of an \Gls{lif} neuron. The normal continues \Gls{lif} equations
    %% collapse to an adder with a negative bias.
    
    %% \afigw{figures/fpga_spike.png}{FPGA Implementation of \Gls{lif}
    %%   Neuron}{fig:fpga_lif}{2}
    
    %% Traditional ML tasks weren't performed on the hardware spiking network, but
    %% a few experiments were performed, which may provide insight. First network
    %% weights were optimized to provide a maximally informative Spacio-temporal
    %% receptive field (STRF) for English speech. This is a key step in voice
    %% recognition. Second the researchers exploited a well-known property of STDP
    %% to validate network and training behavior. Each input was driven by a
    %% Poisson spike train, each with the same $\lambda$. The expected result
    %% (after some time) is that the network will reach a steady state, with an
    %% equal number of exitatory and inhibitory weights. This result was observed,
    %% validating the FPGA implementation aligns with the expected result.
    
    \section{Theories on High-level Functions of Astrocytes Biology}
    
    It is important to speculate on some higher level functional roles of
    Astrocytes, even when the availability of concrete data isn't sufficent to
    analyze these speculations. In general astrocytes can modulate synapse
    excitability through a variety of mechanisms, and modulate synaptic
    plasticity through the release of \Gls{glu} and D-serine. There is evidence
    that astrocytes modulate \Gls{ltp}/\Gls{ltd} based on external inputs and internal
    state, this could provide a mechanism for reward-modulated STDP
    \parencite{min_2012}. In addition It has been shown that smaller learning rates
    coupled with \Gls{stdp} lead to better memory, when presented with test and
    challenge inputs \parencite{van-rossum_2012}. Astrocytes modulate learning at a
    regional level, providing an opportunity for one portion of the network to
    learn, while another retains what it learned previously. In addition to
    regional gating (or more fine-grained modulation) of learning, Astrocytes,
    through local differences in response (within the same cell) can bias a
    network of spiking neurons to learn a particular function, or class of
    functions.

    In addition to plasticity, Astroctes can modulate the synaptic tone, or
    basal level of excitation. It has been suggested that this effect, coupled
    with inter-astrocyte communication can "activate" a region of a network,
    possibly choosing it as most important for the given input. In essence,
    Astrocytes would implement a context switch \parencite{min_2012},
    \parencite{gordleeva_2021}. Another result of this regional excitation is
    synchronization. As Astrocytes lower the threshold for post-synaptic firing,
    pre-synaptic potentials that are disparate in time, or strength can result
    in synchronized post-synaptic firing. This effect could be important in a
    variety of situations, but one of particular note is STDP. Since \Gls{stdp} is
    very sensitive to relative spike timing, having synchronized inputs (and not
    penalizing an input for being ever so slightly late) would result in better
    stability.

    Given Astrocyte's level of connectedness with neurons, and the formation of
    Astrocyte networks, there is a distinct opportunity for information transfer
    across long distances without the need for supporting neural
    connections. It is thought that these pathways underlie some of the
    coordinated activity between neural circuits from different brain
    regions. Some researchers even claim Astrocytes for the basis of human
    consciousness.
