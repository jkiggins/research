% prologue.tex
%
% Author       : James Mnatzaganian
% Contact      : http://techtorials.me
% Date Created : 08/27/15
%
% Description  : The prologue used by "thesis.tex".
%
% Copyright (c) 2015 James Mnatzaganian

% NOTE: All filler text has "TODO" written. This must be removed in the final copy!

% Initialize starting pages to use Roman numerals
\frontmatter

% \begin{acknowledgments}
%%%%%%%%%%%%%%%%%%%%%%%%%%%%%%%%%%%%%%%%%%%%%%%%%%%%%%%%%%%%%%%%%%%%%%%%%%%%%%%
% Acknowledgments
%%%%%%%%%%%%%%%%%%%%%%%%%%%%%%%%%%%%%%%%%%%%%%%%%%%%%%%%%%%%%%%%%%%%%%%%%%%%%%%

\begin{acknowledgments}
\begin{itemize}
    \item Dr. Cory Merkel - For supporting and advising me
    \item Dr. David Schaffer - For supporting and advising me, and providing 
    \item Dr. Alexander Loui - For taking interest in my work, and taking on a role in my committee 
    \item Jay Cisco - my better half, For putting in many late nights with me
    \item My many colleagues at D3 Engineering, and the company as a whole for supporting me
    \item RIT Computer Engineering Dept. - For their continued support, and help funding my work
\end{itemize}
\end{acknowledgments} 
% \end{acknowledgments}

% \begin{dedication}
%%%%%%%%%%%%%%%%%%%%%%%%%%%%%%%%%%%%%%%%%%%%%%%%%%%%%%%%%%%%%%%%%%%%%%%%%%%%%%%
% Dedication
%%%%%%%%%%%%%%%%%%%%%%%%%%%%%%%%%%%%%%%%%%%%%%%%%%%%%%%%%%%%%%%%%%%%%%%%%%%%%%%

\begin{dedication}
\end{dedication}
% \end{dedication}

%\begin{abstract}
%%%%%%%%%%%%%%%%%%%%%%%%%%%%%%%%%%%%%%%%%%%%%%%%%%%%%%%%%%%%%%%%%%%%%%%%%%%%%%%
% Abstract
%%%%%%%%%%%%%%%%%%%%%%%%%%%%%%%%%%%%%%%%%%%%%%%%%%%%%%%%%%%%%%%%%%%%%%%%%%%%%%%

% Abstract
\begin{abstract}
The mammalian brain is the most capable and complex computing entity known
today. For many years there has been research focused on reproducing the brain's
processing capabilities. An early example of this endeavor was the perceptron
which has become the core building block of neural network models in the deep
learning era. Deep learning has had tremendous success in well-defined tasks
like object detection, games like go and chess, and automatic speech
recognition. In fact, some deep learning models can match and even outperform
humans in specific situations. However, in general, they require much more
training, have higher power consumption, are more susceptible to noise and
adversarial perturbations, and have very different behavior than their
biological counterparts. In contrast, spiking neural network models take a step
closer to biology, and in some cases behave identically to measurements of real
neurons. Though there has been advancement, spiking neural networks are far from
reaching their full potential, in part because the full picture of their
biological underpinnings is unclear. This work attempts to reduce that gap
further by exploring a bio-inspired configuration of spiking neurons coupled
with a computational astrocyte model. Astrocytes, initially thought to be
passive support cells in the brain are now known to actively participate in
neural processing. They are believed to be critical for some processes, such as
neural synchronization, self-repair, and learning. The developed astrocyte model
is geared towards synaptic plasticity and is shown to improve upon existing
local learning rules, as well as create a generalized approach to local
spike-timing-dependent plasticity. Beyond generalizing existing learning
approaches, the astrocyte is able to leverage temporal and spatial integration
to improve convergence, and tolerance to noise. The astrocyte model is expanded
to influence multiple synapses and configured for a specific learning task. A
single astrocyte paired with a single leaky integrate and fire neuron is shown
to converge on a solution in 2, 3, and 4 synapse configurations. Beyond the more
concrete improvements in plasticity, this work provides a foundation for
exploring supervisory astrocyte-like elements in spiking neural networks, and a
framework to implement and extend many three-factor learning rules. Overall,
this work brings the field a bit closer to leveraging some of the distinct
advantages of biological neural networks.

\end{abstract}
%\end{abstract}

% \begin{introductory lists and tabels
%%%%%%%%%%%%%%%%%%%%%%%%%%%%%%%%%%%%%%%%%%%%%%%%%%%%%%%%%%%%%%%%%%%%%%%%%%%%%%%
% Introductory Lists and Tables
%%%%%%%%%%%%%%%%%%%%%%%%%%%%%%%%%%%%%%%%%%%%%%%%%%%%%%%%%%%%%%%%%%%%%%%%%%%%%%%

% Add TOC, list of figures, list of tables in that order
\makealllists

% Add the acronyms
\glsaddall
\printglossary[type=\acronymtype]

% Reset all acronyms
\glsresetall

% Start using Arabic numbers
\mainmatter
% \end{introductory lists and tabels}
