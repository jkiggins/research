\chapter{Extension Astrocyte Model To Multi-Synapse
  Configurations} \label{chapter:obj3}

In biology, Astrocytes are observed influencing many synapses, in general,
hundreds of synapse across a single digit number of neurons. This spatial
integration is a fundamental property of Astrocytes, and the response to
activity from these connected synapses is observed as a cell-level \ca
response. Taking this concept and applying it to the Astrocyte model
in this work, a coordinated approach to synaptic plasticity is developed. This
approach is employed in a learning task where a multi-synapse view is critical.

%% \section{Multi-Synapse Cell-Level Response}
%% What is the ultimage goal here... to support some learning on some specific set
%% of inputs, that would not otherwise be achievable w/ STDP. What is the task?
%% what is an interesting task that involves spike timing?

%% What are some things I've got in my notes about this?

%% Gating of plasticity
%%   - Only allow plasticity on synchronous inputs
%%   - The opposite of ^
  
%% Orchestration between Synapse
%%   - Don't allow multiple synapse to learn from the same post-synaptic spike (WTA
%%   Pairing)
%%   - Global learning ``pool'' Each thr event on any synapse depletes the pool,
%%   which then refills
%%   - Relative strength of plasticity

\section{Synaptic Coupling}
This chapter introduces the concept of synaptic coupling. This is an
intermediate between the strictly local response that has been explored thus
far, and the slower cell-level responses directly observed in biology. With
synaptic coupling, there is a regional response to a hand-full of synapses.
This response is fast (unlike the cell-level \ca responses), operating at the
speed of pre and post-synaptic spikes. Some in the Computational and
Neuroscience communities believe similar intermediate astrocyte responses occur
in nature, but that researchers don't have the capabilities to observe them
yet. In any case, synaptic coupling fits the common theme of multi-level
integration, and is shown to be critical in an example learning task. Figure
\ref{fig:astro:syn_coupling} shows the progression of the Astrocyte model from
exhibiting a strictly local responses, to support coupling. There are three
major changes to the model from previous iterations.

\begin{enumerate}
\item The \ca response generated local to a synapse propagates up to the
  regional level. This is consistent with the observation of \Gls{cicr} explored in
  Chapter \ref{chapter:background}.
\item At the regional level, there is some function implemented by the
  Astrocyte, which directs plasticity across any connected synapses.
\item At the regional level, two additional internal signals \dser and
  \serca are introduced. These can be though of as chemical signals which can
  propagate from the regional level, down to a given synapse. They are
  responsible for triggering a weight change locally, or causing
  degredataion/re-uptake of \ipt, \kp, and \ca.
\end{enumerate}

\asvgf{figures/1nNs1a_diagram_obj3.svg}{Astrocyte Functional Diagram for
  Multi-Synapse Synaptic Coupling}{fig:astro:syn_coupling}{0.5} 

\section{Synaptic AND Coupling}
The first function explored for multi-synapse Astrocyte plasticity is logical
AND. This is implemented by a single Astrocyte, influencing N synapses of a
single LIF neuron (Nn1s1a). The expected behavior of an LIF neuron implementing
AND can be defined in terms of single spikes as shown in Figure
\ref{fig:logical-and-coupling}.

\asvgf{figures/AstroAndCoupling.svg}{Logical Definition for AND
  Coupling}{fig:logical-and-coupling}{0.5}

In Figure \ref{fig:logical-and-coupling} each of the blue or orange boxes
represents a spike, either pre-synaptic, or post-synaptic. Figure
\ref{fig:global-v-local-and-coupling} shows the difference between the local
Astrocyte response, vs. the global response required to implement logical
AND. Without any context of the response from other synapses, a strictly local
rule isn't able to correctly converge to an AND function. In most situations
outlined, the regional control logic will need to signal the correct response to
one or more synapses.

To implement this, first the typical activity local to a synapse is considered,
see Figure \ref{fig:local-ca-global-response}. These \ca responses propagate,
and are readily available at the global/regional level. Using the \ca transients
from each synapse, the global logic determines if weights should change on a
given synapse, and if so, in what direction. To communicate the proper response
\dser (which triggers a weight change at a synapse) and \serca, (which triggers
a reset) signals are passed from the regional control level, to individual
synapses.

%% Define equations which outline the logic of AND coupling

\asvgf{figures/AstroAndCoupling_implementation.svg}{Comparison of Astrocyte
  Response Locally, and the Desired AND Coupling
  Response}{fig:global-v-local-and-coupling}{0.5}

\asvgf{figures/AstroAndCoupling_signals.svg}{Logical Definition for AND
  Coupling}{fig:local-ca-global-response}{0.5}

In essence, all spiking activity on N synapses (for the purpose of implementing
AND) fall into one of the following categories.
\begin{itemize}
\item AND - The neuron behaved correctly, this is either $pre_0, pre_1, ...,
  pre_n \implies post_o$ or $others -> \neg post_0$ : Reset all synapses
\item Other-influence - This situation occurs when $post_o$ comes before all
  $pre_i$ inputs. This implies something happening outside the control of
  associated synapses. It doesn't make sense to change the weight in this case,
  as it is unknown why there was a $post_0$ spike. In addition, the LIF neuron
  may be in a refractory period, and it wouldn't make sense to try and update
  weights during that: Reset all synapses.
\item Early-Spike - If $post_0$ occurs before all spiking inputs $pre_0, pre_1,
  ..., pre_N$ have presented. This would indicate one or more of the input
  synapses have weights that are too high: \Gls{ltd} on synapses that received inputs.
\item LPT - If input spikes arrive at $pre_0, pre_1, ..., pre_N$ and there is no
  post-synaptic spike $post_0$, this implies weight values are too low: trigger
  \Gls{ltp} on all synapses.
\end{itemize}

Which of these categories a given set of inputs and response belongs to, depends
on some model parameters.. Namely, the timing constants and thresholds in
Equations \ref{eq:} to \ref{eq:}. Given a set of parameters, there is a
deterministic window of time spikes must fall into in order to be considered for
a particular response. Outside this window, from the perspective of the
Astrocyte, those spikes never happened. This window is dependent on many
parameters however, and possibly changes given previous activity. Instead of
solving for the exact point in time a spikes falls out of the time window, a
lower bound is defined, $\delta_{ptp}=10ms$. The Astrocyte will be tuned to
guarantee proper behavior if spikes fall within this window. This implies that
the time between the first pre-synaptic spike, to the first (possibly only)
post-synaptic spike must be no more than this value. With this timing constraint
a variety of pre-pre-post spiking events can be generated in a two-synapse
configuration. This experiment will remove the influence of synaptic weights and
LIF neuron dynamics, focusing solely on Astrocyte response \ca.

Figure \ref{fig:2n1a_and_response} Shows the response of a single Astrocyte to
two pre-synaptic spiking inputs, and single post-synaptic spiking outputs. Spike
can arrive in any order, or not at all. If spikes do occur, they abide by the
$\delta_{ptp}=10ms$ constraint. Though the figure shows the spiking bouts
continuous in time, the Astrocyte's internal state is reset between each.

\asvgf{figures/artifacts/obj3/astro_2s1a_and.svg}{Astrocyte with AND coupling
  response to two pre-synaptic spiking inputs, and a single post-synaptic spike
  randomly generated}{fig:2n1a_and_response}{0.4}

For the limited number of examples in Figure \ref{fig:2n1a_and_response}, the
Astrocyte performs correct computations. This can be observed by comparing the
ground-truth AND-coupling region at the bottom, with the \dser and \serca
activity. For example: when the region is ``Early Spike'', \dser is $-1.0$ for a
single time-step, indicating \Gls{ltd} on that synapse. Automating this comparison
process, an error rate can be computed, where each 10ms bout is labeled as a
match, or mis-match. A more exhaustive search, consisting
of 10,000 10ms bouts, showed zero mis-matches from the expected result, Equation
\ref{eq:astro_2syn_exaust} shows that 10,000 iterations should be sufficient to
cover the entire input space for two synapses. Figures
\ref{fig:3n1a_and_response} and \ref{fig:4n1a_and_response} show the same success
for three and four synapses, which is supported by the 10k simulation test.

\asvgf{figures/artifacts/obj3/astro_3s1a_and.svg}{Astrocyte with AND coupling
  response to three pre-synaptic spiking inputs, and a single post-synaptic spike
  randomly generated}{fig:3n1a_and_response}{0.4}

\asvgf{figures/artifacts/obj3/astro_4s1a_and.svg}{Astrocyte with AND coupling
  response to four pre-synaptic spiking inputs, and a single post-synaptic spike
  randomly generated}{fig:4n1a_and_response}{0.4}

Though simulations thus far are a good indication of a performant Astrocyte
model; Real-world inputs won't present in nice 10ms packets, or at least the
goal is not to be constrained in such a way. A continuous set of inputs is
explored next, with inputs generated by concatenating a number of 10ms bouts in
time. In this configuration, there are some situations where the Astrocyte does
not respond as expected (a mis-match from ground truth), in general this is due
to a \ca deviation from zero, at the boundary where generated 10ms bouts are
joined. These situations made up $\approx 14\%$ of inputs, across 10,000 10ms
bouts. Figure \ref{fig:astro_2s1a_and_cont} shows some of the failure cases
encountered.

\asvgf{figures/artifacts/obj3/astro_2s1a_and_cont_invalid.svg}{Astrocyte with AND coupling
  response to a continuous timeline of two pre-synaptic spiking inputs, and a
  single post-synaptic spike randomly generated}{fig:astro_2s1a_and_cont}{0.4}

When considering $n=3$ and $n=4$ synapses with continuous randomly generated
spiking inputs, the error rate observed is consistent with that of $n=2$. For
$n=3$ there are 283 mismatches out of 2000 10ms bouts (14.15 \% error). For
$n=4$, there are 416 mismatches out of 3000 (13.9 \% error).

Initial simulations which have isolated the Astrocyte and provided random
inputs, indicate that in a majority of cases, the Astrocyte is capable of
directing plasticity correctly to implement an AND function. The
\dser and \serca signaling reflects the activity of the global AND logic, and
the polarity of \dser matches the desired weight update. A fairly consistent
error rate of 14 \% is observed, which is theorized to be low enough for proper
convergence to implement AND.

Closing the loop fully, the weight update behavior outlined in equation
\ref{eq:astro_and_dw} is implemented. With that addition, an Astrocyte-LIF (1n2s1a)
configuration was simulated, given the same continuous inputs generated by
concatenating 10ms bouts, but only of pre-synaptic spikes for $n=2$
synapses. Where before, post-synaptic spikes were also randomly generated, now
any post-synaptic activity is driven by an LIF neuron with synaptic weights.
The default synaptic weight is $\approx 0.7$.

\begin{align}
  dw_{syn} = ca_{syn} * lr_{ltp} \text{if $ca_{syn}$ > 0} \\
  dw_{syn} = ca_{syn} * lr_{ltd} \text{if $ca_{syn}$ < 0} \\
\end{align}

\asvgf{figures/artifacts/obj3/snn_2s1a_and_xlim.svg}{Simulation of Astrocyte/LIF
  network in the 2s1n1a configuration, with the Astrocyte directing synaptic
  plasticity}{fig:snn_2s1a_and}{0.4}

\asvgf{figures/artifacts/obj3/snn_2s1a_and.svg}{Weight change in the 2s1n1a
  configuration, with the Astrocyte directing synaptic
  plasticity}{fig:snn_2s1a_and_w_dw}{0.4}

Looking at Figure \ref{fig:snn_2s1a_and}: Initially, synaptic weight values are
too high, and the neuron fires in response to a single pre-synaptic spike. As
expected, the Astrocyte traces show \dser signals arriving at the offending
synapses, triggering \Gls{ltd}. Completing the link, and allowing \ca to drive
plasticity, showed weight updates corresponding to the \dser signals and \ca
concentration. As the simulation progresses, weight values decrease, and the LIF
neuron begins to fire in response to two pre-synaptic spikes. There was a slight
over-correction, and some \Gls{ltp} events occur, bringing the weights back
up. Eventually, the synaptic weights converge and offer a good approximation of
an AND function over the inputs.

Under the limited test conditions, Figure \ref{fig:snn_2s1a_and} demonstrates
success for coordinated Astrocyte plasticity. This convergence to an AND
function needs further exploration. Since the coordinated plasticity was
developed to work with N synapses, it is important to evaluate other
configurations besides $n=2$.

\asvgf{figures/artifacts/obj3/snn_3s1a_and.svg}{Weight change in the 3s1n1a
  configuration, with the Astrocyte directing synaptic
  plasticity}{fig:snn_3s1a_and_w_dw}{0.4}

\asvgf{figures/artifacts/obj3/snn_4s1a_and.svg}{Weight change in the 4s1n1a
  configuration, with the Astrocyte directing synaptic
  plasticity}{fig:snn_4s1a_and_w_dw}{0.4}

\section{Summary}
%% Though there is a lack of general consensus, much of the Neuroscience literature
%% suggests that Astrocytes have a fast local response to synaptic activity, which
%% is then integrated to a cell-level response. Taking inspiration from this
%% multi-level integration approach, the Astrocyte model is extended. At the local
%% level, \ipt, \kp, and \ca dynamics remain mostly the same compared to previous
%% chapters. The \ca signals are generated local to the synapse, but now, in
%% addition they propagate up to a regional level. At this multi-synapse, regional
%% level \ca responses from multiple synapses are integrated and a response
%% generated. This response then travels back down via internal (to the astrocyte)
%% chemical signals to one or more synapses, affecting local plasticity. The two
%% regional functions are explored in this chapter, AND and NAND. In general, the
%% Astrocyte will drive weight values towards implementing the desired function,
%% detecting behavior that is not in alignment and updating weights
%% accordingly. For these functions in particular, having a multi-synapse view of
%% activity is critical. It is shown that in most cases, the locally controlled
%% plasticity explored in chapter two would result in incorrect weight
%% updates. With the regional logic mostly in control of when, and in what
%% direction weights move, local dynamics are responsible for the magnitude of a
%% weight change. In this way, global and local dynamics work together to implement
%% a coordinated learning model.

%% With this approach, it was shown that convergence to implementation of an AND
%% function could be achieved with 2, 3, and 4 synapse configurations. Each
%% consisting of a single LIF neuron, single Astrocyte. It was difficult to
%% determine a common set of parameters which gave optimal convergence across the
%% different number of synapses, but parameters were found that provided
%% convergence in each case.

Though there is a lack of general consensus, much of the Neuroscience literature
suggests that Astrocytes have a fast local response to synaptic activity, which
is then integrated to a cell-level response. Taking inspiration from this
multi-level integration approach, the Astrocyte model is extended. At the local
level, \ipt, \kp, and \ca dynamics remain mostly the same compared to previous
chapters. The \ca signals are generated local to the synapse, but now, in
addition they propagate up to a regional level. At this multi-synapse, regional
level \ca responses from multiple synapses are integrated and a response
generated. This response then travels back down via internal (to the astrocyte)
chemical signals to one or more synapses, affecting local plasticity. The two
regional functions are explored in this chapter, AND and NAND. In general, the
Astrocyte will drive weight values towards implementing the desired function,
detecting behavior that is not in alignment and updating weights
accordingly. For these functions in particular, having a multi-synapse view of
activity is critical. It is shown that in most cases, the locally controlled
plasticity explored in chapter two would result in incorrect weight
updates. With the regional logic mostly in control of when, and in what
direction weights move, local dynamics are responsible for the magnitude of a
weight change. In this way, global and local dynamics work together to implement
a coordinated learning model.

With this approach, it was shown that convergence to implementation of an AND
function could be achieved with 2, 3, and 4 synapse configurations. Each
consisting of a single LIF neuron, single Astrocyte. It was difficult to
determine a common set of parameters which gave optimal convergence across the
different number of synapses, but parameters were found that provided
convergence in each case.
