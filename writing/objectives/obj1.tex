% Author       : Jacob Kiggins
% Contact      : jmk1154@g.rit.edu
% Date Created : 01/31/2021
%

\chapter{Development of an Astrocyte Model Sensitive to Spike Timing} \label{chapter:obj1}
The first objective of this thesis, is to develop an Astrocyte-synapse model
that is capable of responding to pre and post-synaptic spike timing, and
explore how internal and external configuration affects this response. These
internal signals will eventually be used to drive synaptic plasticity later in
this work, and that theme will guide the experiments in Chapter \ref{chapter:obj1}.

Though the many details of chemical signaling within Astrocytes are still unknown
\cite{manninen_2018}. There is a common theme to what is known both chemically,
and from bio-inspired models and this theme has potential to greatly improve upon
existing STDP-like plasticity rules. Astrocytes behave as Master integrators of
neural activity. They are capable of sensing activity across many thousands of
synapses simultaneously. Integrating that activity quickly at the local level,
and propagating signals to the entire cell body, integrating over a longer
time-scale. Applying this multi-level integration approach to synaptic
plasticity leads to many interesting questions and potential benefits beyond
existing local learning rules to explore.

\section{Spiking Neuron Model}
This work makes use of a Leaky Integrate and Fire (LIF) neuron model. LIF
neurons are fairly simple computationally, relying on exponential functions
and thresholds in the functional definition. While not the most
bio-realistic (see Figure \ref{fig:sn_model_compare}), LIF neurons exhibit
rich behavior with few parameters, and are easily extended. In addition, LIF
neurons are the most widely used in SNN literature \cite{ponulak_2011}. The
goal here, is to remain computationally simple, while still leveraging
biology, which is the same approach used in the Astrocyte Model definition.

Equations \ref{eq:lif:i}, and \ref{eq:lif:v} define the neuron current and
voltage dynamics.

\begin{align}
d_{i} = -i(t) \tau_{syn} * d_t + z_{pre}(t) * w \label{eq:lif:i} \\
d_{v} = -\tau_{mem} * (v(t) + i(t)) * d_t \label{eq:lif:v}
\end{align}

Here, $z_{pre}(t)$ represents the spiking activity on the synapse at time
$t$ and has a value of either $0$ or $1$. $w$ is the synaptic
weight. $\tau_{syn}$ and $\tau_{mem}$ are synapse and membrane decay
time-constants.

The neuron fires according to Equation \ref{eq:lif:fire}, where $H$ is the
Heaviside step function.

\begin{align}
v'(t) = v(t) + dv \label{eq:lif:v_update} \\
z_{post}(t) = H(v'(t) - thr_{mem}) \label{eq:lif:fire}
\end{align}

After firing the neuron membrane potential is reset to $v_{reset}$, which is
generally a negative voltage $\leq thr_{mem}$, see Equation \ref{eq:lif:reset}.

\begin{align}
v(t) = v'(t) * (1 - z_{post}(t)) + z_{post}(t) * v_{reset} \label{eq:lif:reset}
\end{align}

Figure \ref{fig:1n1s1a_fn_diagram} depicts the LIF neuron described by
equations above. Figures \ref{fig:lif:sample_1}, \ref{fig:lif:sample_2}, and
\ref{fig:lif:sample_3} shows a sample of the LIF responses to a random
Poisson input with various parameter values. Table \ref{table:lif_params} shows
the baseline parameter values, with any differences being outlined in the
figures.

\begin{table}[!htp]\centering
  \caption{LIF Neuron Baseline Parameters} \label{table:lif_params}
  \scriptsize
  \begin{tabular}{lrrrr}\toprule
    Reset Voltage &Membrane Threshold &Membrane Tau &Synaptic Tau \\\midrule
    -0.2 v &0.2 v &60 1/s &500 1/s \\
    \bottomrule
  \end{tabular}
\end{table}

\asvgf{figures/artifacts/obj1/lif_sample_mem-60.0_syn-500.0.svg}{LIF Neuron
Response To Poisson Input. $tau_{mem}=60$ and
$tau_{syn}=500$}{fig:lif:sample_1}{0.7}

\asvgf{figures/artifacts/obj1/lif_sample_mem-30.0_syn-500.0.svg}{LIF Neuron
Response To Poisson Input. $tau_{mem}=30$ and
$tau_{syn}=500$}{fig:lif:sample_2}{0.7}

\asvgf{figures/artifacts/obj1/lif_sample_mem-60.0_syn-250.0.svg}{LIF Neuron
Response To Poisson Input. $tau_{mem}=60$ and
$tau_{syn}=250$}{fig:lif:sample_3}{0.7}

Comparing Figures \ref{fig:lif:sample_1} and \ref{fig:lif:sample_2}, the
neuron seems more excitable with a higher $\tau_{mem}$. This is initially
counter-intuitive, since a higher value would result in increased decay,
however this $\tau_{mem}$ is also scales the effect of the input current on
membrane voltage by the input current (see Equation \ref{eq:lif:v}). In
addition, a larger decay factor results in decreased recovery time for the
neuron from reset. In contrast a lower value for $\tau_{syn?}$ causes the
neuron to fire more often with the same input, due to the input current
decaying more slowly.

\section{Classic STDP} \label{obj1:sec:classic_stdp}
    
The proposed Astrocyte model, in the single-synapse configuration aims to
generalize STDP, and offer some additional effects and features that are
distinct from STDP. In order to have a point of comparison, an STDP
implementation must be chosen to use in this capacity. The chosen model,
referenced by \cite{song_2000} is outlined by Equation
\ref{eq:song_classic_stdp}.

\begin{align}
  F(\delta t) =
  \begin{cases} 
    Ae^{- \frac{|t_{pre}-t_{post}|}{\tau}} & t_{pre} - t_{post} \leq 0, A > 0
    \\ Ae^{- \frac{|t_{pre}-t_{post}|}{\tau}} & t_{pre} - t_{post} > 0, B < 0
  \end{cases} \label{eq:song_classic_stdp}
\end{align}

This equation is evaluated for various $\delta t$ between $-50ms$ and
$50ms$. The resulting curve is outlined in Figure \ref{fig:stdp_dw_dt}. This
functional definition is not directly implemented since there is not timeline of
spikes to work with, instead, the exponential function is approximated by
integrating Equations \ref{eq:song_impl_pre}, \ref{eq:song_impl_post}, and
\ref{eq:song_impl_dw} over time. Table \ref{table:classic_stdp_params} shows the
parameters used during simulation.

\begin{table}[!htp]\centering
  \label{table:classic_stdp_params}
  \scriptsize
  \begin{tabular}{lrr}\toprule
    A &Tau \\\midrule
    1 &100 1/s \\
    \bottomrule
  \end{tabular}
\end{table}

\asvgf{figures/artifacts/obj1/closed_form_stdp.svg}{Simulation of Charicteristic
  STDP Equation}{fig:stdp_dw_dt}{0.4}

\begin{align}
  \delta trace_{pre} = -H(Z_{pre})*(\tau * trace_{pre}) +
  A*Z_{pre} \label{eq:song_impl_pre} \\ 
  \delta trace_{post} = -H(Z_{post})(\tau * trace_{post}) +
  A*Z_{post} \label{eq:song_impl_post} \\
  F = H(Z_{pre}) * trace_{post} + H(Z_{post}) *
  trace_{pre} \label{eq:song_impl_dw}
\end{align}

Depending on the desired behavior, the $H(Z_*)$ term may be dropped from the
update Equations, this is related to the spike associations outlined in Figure
\ref{fig:astro:spike_associate}.

Figure \ref{fig:stdp_dw_dt_impl} shows a sweep of spike pairs as before from
$-50ms$ to $50ms$. This figure matches Figure \ref{fig:stdp_dw_dt}, showing that
the implementation matches the theoretical equations.

\asvgf{figures/artifacts/obj1/1n1s1a_tp_m-plasticity_u-stdp_w-thr_dwdt_classic_stdp.svg}{Simulation
  of Classic STDP Implementation, Equations \ref{song_impl_pre} -
  \ref{song_impl_dw}}{fig:stdp_dw_dt_impl}{0.4}


%%%% Model Development Section %%%%
\section{Model Development}
Model development starts with a survey of current research, and
understanding of the underlying biology of Astrocyte mediated
plasticity. This portion of the process has been completed, see Chapter
\ref{section:background}. Astrocyte models in the literature range from
computationally simple models, which are generally paired with real-valued
neurons. To complex mathematical models, which aim to mimic the calcium
dynamics within an Astrocyte. The latter models are generally paired with
spiking neural networks. From the neuroscience and engineering literature,
it was possible to identify the common input and output pathways within the
tripartite synapse.

\begin{enumerate}
  \item Pre-synaptic Glu $\implies$ IP3 Pathway $\implies$ Ca pathway
  \item Post-synaptic depolarization K+ $\implies$ Ca pathway
  \item Ca $\implies$ D-serine release (more generally, plasticity modulation)
\end{enumerate}

\asvgf{figures/1n1s1a_diagram.svg}{Astrocyte and Neuron Model Diagram For An LIF
  Neuron with One Input}{fig:1n1s1a_fn_diagram}{0.4}

Figure \ref{fig:1n1s1a_fn_diagram} outlines the developed tripartite synapse
model, as well as the internal dynamics of the LIF neuron. Some aspects of
this model will be interchanged depending on the operating mode. These
blocks are the function blocks in darker blue. The portions of the model
which are consistent across operating modes are described first, followed by
a deeper dive into the remaining blocks. The two input pathways, which drive
the state values $ip3$ and $k+$ are described by Equations
\ref{eq:astro:spike-ip3} and \ref{eq:astro:spike-k+}.

\begin{align}
  d_{ip3} = d_t * (-ip3)\tau_{ip3} + \alpha_{ip3} z_{pre}(t) \label{eq:astro:spike-ip3} \\
  d_{k+} = d_t * (-k+)\tau_{k+} + \alpha_{k+} z_{post}(t) \label{eq:astro:spike-k+}
\end{align}

Where $\tau_{ip3/k+}$ represents a time constant for a given pathway, $z_{pre}$
is $1$ if there is a pre-synaptic spike at time $t$ ($0$ otherwise), and
$z_{post}$ is $1$ when the post-synaptic neuron fires ($0$ otherwise). For all
simulations (unless otherwise specified) $d_t = 0.001$ or $1ms$.

The \emph{Update(ip3, k+)} and \emph{Effect(u)} blocks define the behavior
of the Astrocyte at a local level, given input from the internal state
values, which are representative of external activity. There are two main
operating modes of the Astrocyte that have been explored for this
objective. Though in this chapter only the internal state of the Astrocyte will
be considered. \emph{Effect(u)} is then just considered a No-Op.

%% TODO: put this somewhere in Obj2
%% In general, \emph{Effect(u)} represents $d_w$ and takes the form outlined in
%% Equation \ref{astro:dw}. Whenever the Astrocyte state variable reaches the
%% pre-determined threshold the associated weight is scaled by some factor
%% $\alpha$, which may be different, depending on the direction of weight
%% change.

%% \begin{align}
%%   d_w = H(u - u_{thr}) * \alpha_{ltp} + H(-u - u_{thr}) * \alpha_{ltd} \label{astro:dw}
%% \end{align}

%% Here, $H$ is the heaviside step function, and $u_{thr}$ a constant threshold
%% applied to $u$.
    
\section{Exploring Input Pathways Responses}

In order to get some general guidelines for valid parameter ranges, it is
necessary to sweep parameters, and explore the model's response. The parameters
used during simulation are outlined in Table \ref{table:astro_in_path_params},
with any deviations mentioned in the graph.

\begin{table}[!htp] \centering
  \caption{Common Astrocyte Input Pathway Parameters} \label{table:astro_in_path_params}
  \scriptsize
  \begin{tabular}{lrrrrr}\toprule
    dt &Alpha IP3 &Tau IP3 &Alpha K+ &Tau K+ \\\midrule
    0.001 s &1 &100 1/s &1 &100 1/s \\
    \bottomrule
  \end{tabular}
\end{table}

\asvgf{figures/artifacts/obj1/sweep-alpha_tau-ip3-spike0.265.svg}{Model
Response to Sweeping Input pathway $\alpha$, and
$\tau_{ip3/k+}$}{fig:astro:sweep_alpha_ip3_tau_1}{0.6}

\asvgf{figures/artifacts/obj1/sweep-alpha_tau-ip3-spike0.055.svg}{Model Response to
Sweeping Input pathway $\alpha$, and
$\tau_{ip3/k+}$}{fig:astro:sweep_alpha_ip3_tau_2}{0.6}

The group of similar plots in Figures \ref{fig:astro:sweep_alpha_ip3_tau_1} and
\ref{fig:astro:sweep_alpha_ip3_tau_2} depict Astrocyte responses to a common
spike pattern, with different values for $\alpha_{ip3}=\alpha_{k+}$. The
curves in these plots are scale multiples of each-other, showing that
$\alpha_{ip3/k+}$ does not effect timing, but only scale of the Astrocyte
response. The figures also indicate that small values of
$\tau_{ip3}=\tau_{k+}$ can lead to a response that doesn't reach a
steady-state within a reasonable window, such as with $\tau_{ip3/k+} = 10$.
%% TODO: add more detail about steady-state


\section{Signed Proportional Astrocyte State Change} \label{sec:sign_prop}

The first variant of the \emph{Update(ip3, k+)} function considers the
magnitude of input pathway state variables $ip3$ and $k+$ as well as which
variable is larger (the order). This gives a magnitude and sign to each
update to the Astrocyte state variable $u$. The value of U eventually leading to LTP or
LTD, which is explored in Chapter \ref{chapter:obj2}. This approach is outlined
by Equations \ref{eq:astro:rate-u} and \ref{eq:astro:u-reset}, which define the
change in \ca

\begin{align}
  T_{diff} = ip3 - k^+ \\
  dca_{delta} = (H(T_{diff} - thr_{ltp}) - H(-T_{diff} - thr_{ltd})) * |ip3*k^+| \\
  dca_{decay} = -ca * \tau_{ca} * d_t \\
  dca = dca_{delta} + dca_{decay} \label{eq:astro:rate-u} \\
  ca = ca * (1 - H(ca - ca_{thr})) \label{eq:astro:u-reset}
\end{align}

Where H is the Heaviside step function, $T_{diff}$ is the difference between
$ip3$ and $K^+$ traces at some time $t$. $thr_{ltp}$ and $thr_{ltd}$ determine
the valid ranges of $T_{diff}$ for an update to \ca to occur. $dca_{delta}$
describes the change in \ca at a given time from external influence, while
$dca_{decay}$ accounts for the leaking of \ca. Note that the threshold values
determine the sign of $dca_{delta}$, and provide the ability to shift the \ca
response, as shown in Figure \ref{fig:astro:stdp_ltd_shift}. Increasing the distance
between the $ltp$ and $ltd$ thresholds also allow for varying degrees of
tolerance to transient activity that would otherwise lead to a response by the
Astrocyte. Simulations use the baseline parameter values outlined in Table
\ref{table:ordered_prop_params}, with any variations or parameter sweeps
outlined in graphs.

\begin{table}[!htp]\centering
  \caption{Signed Proportional Astrocyte Parameters} \label{table:ordered_prop_params}
  \scriptsize
  \begin{tabular}{lrrrrrr}\toprule
    dt &Alphas &Tau IP3 &Tau K+ <P Threshold <D Threshold \\\midrule
    0.001 s &1 &100 1/s &100 1/s &0 &0 \\
    \bottomrule
  \end{tabular}
\end{table}

\asvgf{figures/artifacts/obj1/sweep_u_tau_p0.1.svg}{Response of \ca to
Poisson Distributed spikes applied to Pre and Post-Synaptic
Pathways}{fig:astro:sweep_alpha_u_tau_1}{0.8}

\asvgf{figures/artifacts/obj1/sweep_u_tau_p0.7.svg}{Response of \ca to
Poisson Distributed spikes applied to Pre and Post-Synaptic
Pathways}{fig:astro:sweep_alpha_u_tau_2}{0.8}

Figures \ref{fig:astro:sweep_alpha_u_tau_1} and
\ref{fig:astro:sweep_alpha_u_tau_2} depict the response of \ca for
different values of $tau_{ca}$ and different spiking events. In general,
larger values of $\tau_{ca}$ result in smaller deviations in \ca overall, due
to the value decaying more quickly. With smaller $\tau_{ca}$ more of the
previous activity is integrated into the value of \ca and the response is
more broad.

Expanding on the simpler timeline plots, Figure
\ref{fig:astro:u_thr_heat_rate_vs_tau_u} shows the number of times \ca crosses a
threshold of $2.5$ for a given $\tau_{ca}$ and $r$ spiking rate for the Poisson
spike-train inputs. This graph can be used to tune either $thr_{ca}$ or $\tau_{ca}$
for a given application, and gives an impression of the excitability of the
model.

\asvgf{figures/artifacts/obj1/heatmap_spike-rate_tau_u.svg}{Rate of
Astrocyte Effect on Synaptic Weight Given Different input Poisson spike
rates, and different values of $\tau_{ca}$}{fig:astro:u_thr_heat_rate_vs_tau_u}{0.3}

\subsection{Classic-STDP and Variants} \label{sec:ordered_prop:stdp}

Using the Signed Proportional update rule the Astrocyte Plasticity Model is
capable of implementing classic STDP, along with several variations.  Figure
\ref{fig:astro:classic_stdp} shows the model's response to spike pairs with
a variety of inter-spike-interval times (X-axis). This is a typical STDP
curve.

\asvgf{figures/artifacts/obj1/1n1s1a_tp_m-plasticity_u-stdp_w-thr_dwdt_classic_stdp.svg}{Astrocyte
  and Neuron Model Response to Pulse Pairs:
  Baseline}{fig:astro:classic_stdp}{0.4}

The model has additional parameters beyond what a traditional STDP rule
would have, and this offers flexibility. By providing asymmetric
$\tau_{ip3}$ and $\tau_{k+}$ the model can be bias toward LTD/LTP. Figure
\ref{fig:astro:stdp_ltd_bias} shows this effect for LTD.

\asvgf{figures/artifacts/obj1/1n1s1a_tp_m-plasticity_u-stdp_w-thr_dwdt_ltd_bias.svg}{Astrocyte-Neuron
Model Response to Pulse Pairs: LTD Bias}{fig:astro:stdp_ltd_bias}{0.4}

In addition, threshold parameters $thr_{ltp}$ and $thr_{ltd}$ can be used to
shift activity from the LTP to LTD region, or vise versa. Figure
\ref{fig:astro:stdp_ltd_shift} shows this shifting behavior.

\asvgf{figures/artifacts/obj1/1n1s1a_tp_m-plasticity_u-stdp_w-thr_dwdt_ltd_dt_shift.svg}{Astrocyte-Neuron
Model Response to Pulse Pairs: LTD Shift}{fig:astro:stdp_ltd_shift}{0.4}

%% Additional figures can be found in Appendix \ref{appendix:astro_figures}.

One of the main features of this model is it's ability to generalize STDP,
and capture a variety of variants that have been proposed an explored in the
literature. Table \ref{table:astro_varient_params} maps specific values of
model parameters to the corresponding STDP behavior. If that
variant of STDP has a name from other research it is included.

\begin{table}[!htp]\centering
\caption{Model Parameters Associated with STDP Variants} \label{table:astro_varient_params}
\scriptsize
\begin{tabular}{lrrrrrrrrrrr}\toprule
STDP Variant &Name from Literature &$alpha_{ip3}$ &$tau_{ip3}$ &$alpha_{k+}$ &$tau_{k+}$ &$tau_u$ &$ltp_{thr}$ &$ltd_{thr}$ &$reset_{ip3}$ &$reset_{k+}$ \\\midrule
Triplet STDP &Triplet STDP &1 &100 &1 &100 &10000 &0 &0 &Yes &Yes \\
Classic & &1 &100 &1 &100 &10000 &0 &0 &No &No \\
LTD Bias & &1 &100 &1 &30 &10000 &0 &0 &No &No \\
LTP Bias & &1 &30 &1 &100 &10000 &0 &0 &No &No \\
LTD Shift &Mirror STDP &1 &100 &1 &100 &10000 &0.5 &0.5 &No &No \\
LTP Shift & &1 &100 &1 &100 &10000 &-0.5 &-0.5 &No &No \\
Anti-STDP &Anti-STDP, Rev-STDP &-1 &100 &-1 &100 &10000 &0 &0 &No &No \\
\bottomrule
\end{tabular}
\end{table}

\section{Rate-based Response} \label{sec:rate_response}

When input spiking activity is sparse in time, the model behaves as outlined
in Section \ref{sec:ordered_prop:stdp} above. However, when more dense
inputs are provided, the relationship between $ip3$ and $k+$ shifts from
temporal, to activity-based. In the sparse case, $ip3 > k+$ implies that an
incoming spike happened more recently then a post-synaptic spike. When
incoming spikes are dense in time, $ip3 > k+$ would imply more activity on
the input side (on average) than the output. What follows, is a Rate-based
internal response, where \ca responds to a difference in average activity,
subject to time constants. Equation \ref{eq:astro:rate-u} above describes \ca
behavior. Equation \label{eq:astro:u-reset} outlines the reset of behavior \ca,
the value reset to $0$ when $|ca| > thr_{ca}$.

In Figure \ref{fig:astro:rp_many-w_tl}, the middle plot shows the internal
response of an Astrocyte to a random Poisson across different weight values. In
this case, there is significant response from the Astrocyte for weight values of
$2.0$, with less of a response for $4.0$. This is due to a lower firing rate out
of the LIF neuron compared to the input. Intuitively there should be a weight
value where the difference between pre and post-synaptic firing rate is
smallest. For the upper graph $w=10.0$ is close to that point, with only two
regions in the timeline with any activity.

\asvgf{figures/artifacts/obj1/astro_rp_many-w_tl.svg}{Astrocyte Response to Random
  Poission Spike Train for Various Synaptic Weights}{fig:astro:rp_many-w_tl}{0.4}

Looking at the upper plot in Figure \ref{fig:astro:rp_many-w_tl}, the effect of
$thr_{ltp}=1.75$ and $thr_{ltd}=-1.75$ can be seen. Those parameter values form
a hysteresis band, where the Astrocyte does not respond with increased \ca for
small deviations between \ipt and \kp. This leads to an increased range of
weight values resulting in stable behavior. Sweeping synaptic weight in Figure
\ref{fig:astro:rp_many_w_sweep} shows that weight values beyond $w \approx 3.0$
result in no \ca activity.

\asvgf{figures/artifacts/obj1/astro_rp_many-w_sweep.svg}{Astrocyte Response to Random
  Poission Spike Train for Various Synaptic Weights}{fig:astro:rp_many_w_sweep}{0.4}

%% One approach to mitigate the averaging effect would be lowering
%% $\tau_{ip3}$ and $\tau_{k+}$, though this would reduce sensitivity to pulse
%% pairs occurring on larger time-scales. A more elegant approach would be to
%% modify \emph{Update(ip3, k+)}, combining the input traces in a way that
%% preserves timing. $u$ can then be used to integrate pulse-pair events over
%% time and drive weight change. Equation \ref{eq:astro:temp-u} describes the
%% update for $u$.

\section{Integrated Spike Timing Response} \label{section:istp}
Next, this work considers an variant of \emph{Update(ip3, k+)} that is sensitive
to pre and post-synaptic spike timing, but not the rate of activity at each
terminal. The previous Signed Proportional model outlined in Section
\ref{sec:sign_prop} could exhibit sensitivity to spike timing. This was
demonstrated in Figure \ref{fig:astro:classic_stdp}, where the characteristic
STDP curve was reproduced by the Astrocyte model. As shown in Section
\ref{sec:rate_response} the sensitivity to spike timing only holds for sparse
inputs. With dense inputs, \ipt and \kp values represent average activity, and not
timing.

This new \emph{Update(ip3, k+)} function is define by Equation
\ref{eq:astro:temp-u}. $reset_{ip3}$ and $reset_{k+}$ parameters, determine if
additional spikes are added into the $ip3$ and $k+$ traces, or override
them. These parameters, along with the weight update behavior determine which
groupings of spike result in changes to $u$. Figure
\ref{fig:astro:spike_associate} outlines different possible spike association
behaviors.

%% TODO: Define reset behavior

\asvgf{figures/SpikeAssociation.svg}{A Variety of Spike Associations
Supported by the Neuron-Astrocyte model}{fig:astro:spike_associate}{0.5}

\begin{align}
T_{delta} = sign_{ltd} * z_{pre} * k^+ + sign_{ltp} * z_{post} * ip3 \\
dca = -ca * \tau_{ca} + d_t + T_{delta} \label{eq:astro:temp-u}
\end{align}

In this case, $z_{post}$ and $z_{pre}$ are pre and post-synaptic spikes that
the Astrocyte is sensing. The sign $sign_{ip3/k+}$ values may be functions
of $k+$ and $ip3$ as well as some constants. For some experiments they will
simply be $sign_{ltd}=-1.0$ and $sign_{ltp}=1.0$. Equations
\ref{eq:astro:temp-sign-1} and \ref{eq:astro:temp-sign-2} show definitions
of $sign_{ltd}$ and $sign_{ltp}$ that would allow a region around $dt=0$
where weight updates were 0.

\begin{align}
sign_{ltd} = -(1 - H(k+ + thr_{ltd})) \label{eq:astro:temp-sign-1}\\
sign_{ltp} = 1 - H(ip3 + thr_{ltp}) \label{eq:astro:temp-sign-2}
\end{align}

Unlike with the rate-based variation, updates to \ca happen only when spikes
occur on either pre or post-synaptic terminals, instead of continuously. As
in the rate-based model, these updates to \ca are integrated, and reset when \ca
exceeds a threshold $thr_{ca}$. The baseline parameter values for simulation are
outlined in Table \ref{table:istp_params}.

\begin{table}[!htp] \centering
  \caption{Integrated Spike Timing Astrocyte Parameters} \label{table:istp_params}
  \scriptsize
  \begin{tabular}{lrrrrr}\toprule
    dt &Alphas &Tau IP3 &Tau K+ &Ca Threshold \\\midrule
    0.001 s &1 &100 1/s &100 1/s &2.5 \\
    \bottomrule
  \end{tabular}
\end{table}

Figure \ref{fig:astro:tp_many_w_tl} shows the behavior of temporal
integration model when a series of spike impulses are presented to the
network. Each successive pulse has an additional spike from the last and the
resulting \ca activity s depicted for a given synaptic weight. This simulation
exposes the averaging effect of the spike timing model. With an impulse of 4
spikes, the neuron is driven enough to output a single spike, this happens around
300ms. With 4 pre-synaptic spikes and 1 post-synaptic spike, the Astrocyte's
\ca is increased, exceeding the threshold of $2.5$. Subsequent spike impulses
have $>4$ spikes, which leads to a drop in \ca, due to anti-causal pairing
(transiently) of pre-synaptic spikes. Regardless of these transient drops, \ca
consistently exceeds $thr_{ca}$ throughout the simulation.

\asvgf{figures/artifacts/obj1/astro_tp_impulse.svg}{Astrocyte Response to
  Ramping Impulse Input: Demonstrating pulse-pair averaging
  feature}{fig:astro:ramp_impulse}{0.5}

Figure \ref{fig:astro:tp_many_w_sweep} graphs the number of times $thr_{ca}$ is
exceeded during a fixed length simulation, for a few different values of
$thr_{ca}$. For $Ca^{2+}=2.5$, regions of the graph around $0.8$ and $1.2$ show
weight values that result in a stable configuration, where $thr_{ca}$ is not
exceeded for the length of simulation. Figure \ref{fig:astro:tp_many_w_tl} shows
a timeline of Astrocyte activity for three different weight values, and
$thr_{ca}=2.5$.

\asvgf{figures/artifacts/obj1/astro_tp_many-w_sweep.svg}{Aggregate Astrocyte
  Response to Input from a single synapse, with various synaptic weight and
  $thr_{ca}$}{fig:astro:tp_many_w_sweep}{0.5}

\asvgf{figures/artifacts/obj1/astro_tp_many-w_tl.svg}{Timeline of Astrocyte
  Activity With Input from a single synapse, given
  $thr_{ca}=2.5$}{fig:astro:tp_many_w_tl}{0.5}


To better explore the activity surrounding a stable configuration, Figure
\ref{fig:astro:tp_many_w_tl_xlim} shows a blown-up region of Astrocyte activity
with $w=1.2$. This graphs shows simulation with three synaptic weights. This
shows how stable configurations are linked to synchronous activity, where the
downstream neuron fires in time with the upstream. This is a result of a
convention used in this work that synchronous spikes don't result in any change
in \ca.

\asvgf{figures/artifacts/obj1/astro_tp_many-w_xlim_tl.svg}{Restricted Timeline of Astrocyte
  Activity with input from a single synapse, given
  $thr_{ca}=2.5$}{fig:astro:tp_many_w_tl_xlim}{0.5}

Using values $thr_{ltd}=thr_{ltp}=1.0$ (see Equation \ref{eq:astro:temp-sign-1})
results in a slight change in the model behavior, which adds the condition in
Equation \ref{eq:dca_band}, resulting in different behavior, outlined in Figure
\ref{fig:astro:tp_many_w_sweep}. In general, there are greater regions of
stability when compared to the previous simulation without the
change. For $thr_{ca}=3.0$ there are no threshold events within the range of
synaptic weights used. Essentially, this change increases the tolerance for
Astrocyte response, and for what is considered synchronous activity.

\asvgf{figures/artifacts/obj1/astro_tp_many-w_sweep_dca-max.svg}{Number of \ca
  events with different synaptic weights.}{fig:astro:tp_many_w_sweep}{0.5}

\section{Summary}
This chapter introduces a computationally efficient and intuitive Astrocyte
model. This model was derived from the many works outlined in Chapter
\ref{chapter:backround}, leaning heavily towards Engineering applications,
while taking inspiration from the Neuroscience and many highly bio-realistic
models. A relatively small number of parameters are used to define the
Astrocyte model's behavior, and the effects of varying this behavior is
explored in two different configurations of the model, with an emphasis on
signals internal to the Astrocyte. Figure \ref{fig:1n1s1a_fn_diagram} shows
an ``Update FN'' block, which represents how the input traces \ipt and \kp
effect Astrocyte \ca. With \ca concentration driving the mechanisms
identified for the Astrocyte to influence things external to
itself. Considering a variant of the model where \ca responds to a
difference in the magnitude between \ipt and \kp, it was found that \ca
response varied widely given a random Poisson input for some weight values.
For other weight values, where a quick response from the neuron was
observed, \ca was relatively stable. Switching the input to be a pair
of pre and post-synaptic spikes with variable inter-spike-interval, a graph
very similar to the characteristic STDP curve was observed (Figure
\ref{fig:astro:classic_stdp}).

Since the first variation of the Astrocyte model exhibited sensitivity to
the general firing rates at pre and post-synaptic terminals, it was logical
to explore a variation sensitive to spike timing. This ``Integrated Spike-Timing''
model, similarly, exhibited an STDP-like \ca response given a single spiking
pair with variable timing. With dense inputs this model maintained sensitivity
to recent spike timing, and exhibited bi-modal stability with respect to
synaptic weights.
