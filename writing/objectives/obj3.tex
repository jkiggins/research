\chapter{Adaptation of the Astrocyte Model To Multi-Synapse
  Configurations} \label{chapter:obj3}

In biology, Astrocytes have not been observed influencing a single synapse on a
single neuron. In general they encapsulate hundreds synapse across a single
digit number of neurons. In this chapter, single neuron, multi-synapse
configurations will be explored. Taking that small step closer to a bio-inspired
configuration, but remaining simple enough to explore fundamental properties.

\section{Multi-Synapse Cell-Level Response}
What is the ultimage goal here... to support some learning on some specific set
of inputs, that would not otherwise be achievable w/ STDP. What is the task?
what is an interesting task that involves spike timing?

What are some things I've got in my notes about this?

Gating of plasticity
  - Only allow plasticity on synchronous inputs
  - The opposite of ^
  
Orchestration between Synapse
  - Don't allow multiple synapse to learn from the same post-synaptic spike (WTA
  Pairing)
  - Global learning ``pool'' Each thr event on any synapse depletes the pool,
  which then refills
  - Relative strength of plasticity


\section{Summary}
In this chapter, the developed Astrocyte model is expanded to incorporate
multiple synapses, along with the concept of a global, cell-level response. The
goal here is to explore how an additional synapse alters existing behaviors and
properties of the Astrocyte model, as well as introduce a novel cell-level
response, which helps the model to learn a desired response to a set of
inputs. This learning bout will exhibit properties not achievable with Classic
STDP in a similar configuration.
