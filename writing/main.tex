% thesis.tex
%
% Original version ::
% Author       : James Mnatzaganian
% Contact      : http://techtorials.me
% Date Created : 08/27/15
%
% Description  : James Mnatzaganian's thesis document for an MS in CE at RIT.
%
% Organization :
%	figures/			% Figures related to the top-level
%	prologue.tex		% Prologue (before the main document)
%	glossary.tex		% Acronymns and glossary
%	mybibliography.bib			% Bibliography
%	<chapter>			% Folder for the <chapter> chapter
%		figures/		% Figures for the <chapter> chapter
%		<chapter>.tex	% Document for the <chapter> chapter
%
% Copyright (c) 2015 James Mnatzaganian
%
% Version 2 ::
% Author       : Andres Kwasinski
% Contact      : https://people.rit.edu/axkeec/
% Date Created : 01/09/2020
%




% NOTE: All filler text has "TODO" written. This must be removed in the final copy!

% \begin{document type}
%%%%%%%%%%%%%%%%%%%%%%%%%%%%%%%%%%%%%%%%%%%%%%%%%%%%%%%%%%%%%%%%%%%%%%%%%%%%%%%
% Document Type
%%%%%%%%%%%%%%%%%%%%%%%%%%%%%%%%%%%%%%%%%%%%%%%%%%%%%%%%%%%%%%%%%%%%%%%%%%%%%%%

% Define the document type
\documentclass[cmpethesis]{ritcmpethesis}  % Use this line for your thesis document
%\documentclass[cmpeproject]{ritcmpethesis}  % Use this line for your project report
%\documentclass[cmpeproject,cmpeproposal]{ritcmpethesis}  % Add the option "cmpeproposal" when preparing the proposal for your project research
%\documentclass[cmpethesis,cmpeproposal]{ritcmpethesis}  % Add the option "cmpeproposal" when preparing the proposal for your thesis research
% \end{document type}\end{}

% \begin{packages}
%%%%%%%%%%%%%%%%%%%%%%%%%%%%%%%%%%%%%%%%%%%%%%%%%%%%%%%%%%%%%%%%%%%%%%%%%%%%%%%
% Packages
%%%%%%%%%%%%%%%%%%%%%%%%%%%%%%%%%%%%%%%%%%%%%%%%%%%%%%%%%%%%%%%%%%%%%%%%%%%%%%%

% From ShareLatex google sheets extension
\usepackage{booktabs, multirow} % for borders and merged ranges
\usepackage{soul}% for underlines
\usepackage[table]{xcolor} % for cell colors
\usepackage{changepage,threeparttable} % for wide tables

% Used for creating clicking references
\usepackage[hidelinks]{hyperref}

% Used for displaying images
\usepackage{graphicx}

% Support for typesetting math
\usepackage{mathtools}

% Support for number sets
\usepackage{amsfonts}

% Support for logic notation
\usepackage{amssymb}

% Support for typesetting subcaptions
\usepackage{subcaption}

% Adding TODO Notes
\usepackage{todonotes}

%% \usepackage[backend=biber, bibencoding=utf8, style=alphabetic, citestyle=ieee]{biblatex}
\usepackage[backend=biber, bibencoding=utf8]{biblatex}
\addbibresource{mybibliography.bib}

% Typset indexes - Needed for sorting the glossary
%  - xindy: Sorting / indexing of items
\usepackage[xindy]{imakeidx}

% Support for glossaries
%  - nopostdot: Omit dot at the end of each description
%  - nonumberlist: Supress number of items
%  - acronym: Support for acronyms
%  - toc: Add glossary to table of contents
%  - xindy: Sorting / indexing of items
\usepackage[nopostdot,nonumberlist,acronym,toc,xindy]{glossaries}

% Support for displaying pseudo-code
\usepackage{algorithm}

% Support for displaying pseudo-code
%  - noend: Don't display end ...
\usepackage[noend]{algpseudocode}

% Support for pretty inline fractions
\usepackage{nicefrac}

% To generate the dummy text you'll find all over
\usepackage[english]{babel}
\usepackage{blindtext}

% Add hyperref to ref list items
\usepackage{hyperref}

% SVG package to include SVGs
\usepackage{svg}

% \end{packages}

% \begin{macros}
%%%%%%%%%%%%%%%%%%%%%%%%%%%%%%%%%%%%%%%%%%%%%%%%%%%%%%%%%%%%%%%%%%%%%%%%%%%%%%%
% Macros
%%%%%%%%%%%%%%%%%%%%%%%%%%%%%%%%%%%%%%%%%%%%%%%%%%%%%%%%%%%%%%%%%%%%%%%%%%%%%%%

% Default header for a table
\newcommand{\tableheader}[1]{\multicolumn{1}{|c|}{\textbf{#1}}}

% Section referencing
\newcommand{\sref}[1]{Section~\ref{#1}}

% Figure referencing
\newcommand{\fig}[1]{Figure~\ref{#1}}

% Aliases for common names
\newcommand{\ca}{$\textrm{Ca}^{2+}$}
\newcommand{\na}{$\textrm{Na}^+$}
\newcommand{\ipt}{\Gls{ip3}}
\newcommand{\kp}{$\textrm{K}^+$}
\newcommand{\dser}{\Gls{dser}}
\newcommand{\serca}{\Gls{serca}}

% Usage: \afig{url}{Figure caption}{label for referencing later}
\newcommand{\afig}[3]{
	\begin{figure}[H]
    	\centering
		\includegraphics[width=\linewidth]{#1}
        \caption{#2}
        \label{#3}
	\end{figure}
}

% Usage: \afigf{url}{Figure caption}{label for referencing later}
\newcommand{\afigf}[3]{
	\begin{figure}[h]
    	\centering
		\includegraphics[width=\linewidth]{#1}
        \caption{#2}
        \label{#3}
	\end{figure}
}

\newcommand{\asvgf}[4]{
	\begin{figure}[H]
    	\centering
		\includesvg[scale=#4]{#1}
        \caption{#2}
        \label{#3}
	\end{figure}
}

% Usage: \afigw{url}{Figure caption}{label for referencing later}{width 1/factor}
\newcommand{\afigw}[4]{
	\begin{figure}[H]
    	\centering
		\includegraphics[width=\linewidth/#4]{#1}
        \caption{#2}
        \label{#3}
	\end{figure}
}

% Usage: \afigs{url}{Figure caption}{label for referencing later}{width}
\newcommand{\afigs}[4]{
	\begin{figure}[H]
    	\centering
		\includegraphics[width=#4\columnwidth]{#1}
        \caption{#2}
        \label{#3}
	\end{figure}
}

% Equation referencing
\newcommand{\eq}[1]{(\ref{#1})}

% Algorithm referencing
\newcommand{\alg}[1]{Algorithm~\ref{#1}}

% Glossary referencing
\newcommand{\glsref}[1]{\\ \textit{Glossary:} \gls{#1}}

% Change comment style to use #
\algrenewcommand{\algorithmiccomment}[1]{\# #1}

% Make *proper* vector arrows - Credit to harpoon pacakge for initial idea
\newlength{\argwd}
\newlength{\arght}
\newcommand{\overharp}[3]{%
	\settowidth{\argwd}{#2}%
	\settoheight{\arght}{#2}%
	\addtolength{\argwd}{.1\argwd}%
	\raisebox{\arght}{%
		\makebox[.04\argwd][l]{%
			\resizebox{\argwd}{#3\arght}{$#1$}%
		}%
	}%
	#2%
}
\newcommand{\overrightharp}[2]{\overharp{\rightharpoonup}{#1}{#2}}
\newcommand{\vect}[2][.5]{\text{\overrightharp{\ensuremath{\boldsymbol{#2}}}{#1}}}
\newcommand{\vectmd}[2][.5]{\text{\overrightharp{\ensuremath{#2}}{#1}}}

% Make *proper* text over sim - Credit: http://tex.stackexchange.com/a/43338/66603
\newsavebox{\mybox}\newsavebox{\mysim}
\newcommand{\distas}[1]{%
  \savebox{\mybox}{\hbox{$\scriptstyle#1$}}%
  \savebox{\mysim}{\hbox{$\sim$}}%
  \mathbin{\overset{#1}{\resizebox{\wd\mybox}{\ht\mysim}{$\sim$}}}%
}
% \end{macros}

% \begin{document configuration}
%%%%%%%%%%%%%%%%%%%%%%%%%%%%%%%%%%%%%%%%%%%%%%%%%%%%%%%%%%%%%%%%%%%%%%%%%%%%%%%
% Document Configuration
%%%%%%%%%%%%%%%%%%%%%%%%%%%%%%%%%%%%%%%%%%%%%%%%%%%%%%%%%%%%%%%%%%%%%%%%%%%%%%%

% Add DRAFT to the document
% Comment the next 5 lines to remove "DRAFT" watermark
\usepackage{draftwatermark}
\SetWatermarkText{DRAFT}
\SetWatermarkScale{9}
\SetWatermarkColor[gray]{0.90}
\SetWatermarkAngle{45}

% Set the path for the figures
\graphicspath{{figures/}{01_introduction/figures/}{02_background/figures/}}

% Author, title, and date
\author{Jacob Kiggins}
\title{Development of a Bio-Inspired Computational Astrocyte Model for Spiking Neural Networks}
\date{December 2022}

% Advisor details
\advisor{Department of Computer Engineering}{Cory}{Merkel}
\committee{Research Professor at Binghamton University}{David}{Schaffer}{}
\committee{Department of Computer Engineering}{Alexander}{Loui}{}
% \end{document configuration}\end{}

% \begin{glossary}
%%%%%%%%%%%%%%%%%%%%%%%%%%%%%%%%%%%%%%%%%%%%%%%%%%%%%%%%%%%%%%%%%%%%%%%%%%%%%%%
% Glossary Setup
%%%%%%%%%%%%%%%%%%%%%%%%%%%%%%%%%%%%%%%%%%%%%%%%%%%%%%%%%%%%%%%%%%%%%%%%%%%%%%%

% Load the glossary
\loadglsentries{glossary}

% Load the acronyms
\loadglsentries[type=\acronymtype]{acronym}

% Initialize the glossary
\makeglossaries
\setglossarystyle{altlist}

% Sort the glossary
\makeindex
% \end{glossary}

% \begin{document}
%%%%%%%%%%%%%%%%%%%%%%%%%%%%%%%%%%%%%%%%%%%%%%%%%%%%%%%%%%%%%%%%%%%%%%%%%%%%%%%
% Document Start
%%%%%%%%%%%%%%%%%%%%%%%%%%%%%%%%%%%%%%%%%%%%%%%%%%%%%%%%%%%%%%%%%%%%%%%%%%%%%%%

\begin{document}
	
	% Pre chapter stuff
	% prologue.tex
%
% Author       : James Mnatzaganian
% Contact      : http://techtorials.me
% Date Created : 08/27/15
%
% Description  : The prologue used by "thesis.tex".
%
% Copyright (c) 2015 James Mnatzaganian

% NOTE: All filler text has "TODO" written. This must be removed in the final copy!

% Initialize starting pages to use Roman numerals
\frontmatter

% \begin{acknowledgments}
%%%%%%%%%%%%%%%%%%%%%%%%%%%%%%%%%%%%%%%%%%%%%%%%%%%%%%%%%%%%%%%%%%%%%%%%%%%%%%%
% Acknowledgments
%%%%%%%%%%%%%%%%%%%%%%%%%%%%%%%%%%%%%%%%%%%%%%%%%%%%%%%%%%%%%%%%%%%%%%%%%%%%%%%

\begin{acknowledgments}
Dr. Cory Merkel
Dr. David Schaffer
Dr. Alexander Loui
Jay Cisco - For putting in many late nights with me
My many colleagues at D3 Engineering
RIT Computer Engineering Dept. - Continued support, and help funding my work
\end{acknowledgments}
% \end{acknowledgments}

% \begin{dedication}
%%%%%%%%%%%%%%%%%%%%%%%%%%%%%%%%%%%%%%%%%%%%%%%%%%%%%%%%%%%%%%%%%%%%%%%%%%%%%%%
% Dedication
%%%%%%%%%%%%%%%%%%%%%%%%%%%%%%%%%%%%%%%%%%%%%%%%%%%%%%%%%%%%%%%%%%%%%%%%%%%%%%%

\begin{dedication}
\end{dedication}
% \end{dedication}

%\begin{abstract}
%%%%%%%%%%%%%%%%%%%%%%%%%%%%%%%%%%%%%%%%%%%%%%%%%%%%%%%%%%%%%%%%%%%%%%%%%%%%%%%
% Abstract
%%%%%%%%%%%%%%%%%%%%%%%%%%%%%%%%%%%%%%%%%%%%%%%%%%%%%%%%%%%%%%%%%%%%%%%%%%%%%%%

% Abstract
\begin{abstract}
The mammalian brain is the most capable and complex computing entity known
today. For many years there has been research focused on reproducing the brain's
processing capabilities. An early example of this endeavor was the perceptron
which has become the core building block of neural network models in the deep
learning era. Deep learning has had tremendous success in well-defined tasks
like object detection, games like go and chess, and automatic speech
recognition. In fact, some deep learning models can match and even outperform
humans in specific situations. However, in general, they require much more
training, have higher power consumption, are more susceptible to noise and
adversarial perturbations, and have very different behavior than their
biological counterparts. In contrast, spiking neural network models take a step
closer to biology, and in some cases behave identically to measurements of real
neurons. Though there has been advancement, spiking neural networks are far from
reaching their full potential, in part because the full picture of their
biological underpinnings is unclear. This work attempts to reduce that gap
further by exploring a bio-inspired configuration of spiking neurons coupled
with a computational astrocyte model. Astrocytes, initially thought to be
passive support cells in the brain are now known to actively participate in
neural processing. They are believed to be critical for some processes, such as
neural synchronization, self-repair, and learning. The developed astrocyte model
is geared towards synaptic plasticity and is shown to improve upon existing
local learning rules, as well as create a generalized approach to local
spike-timing-dependent plasticity. Beyond generalizing existing learning
approaches, the astrocyte is able to leverage temporal and spatial integration
to improve convergence, and tolerance to noise. The astrocyte model is expanded
to influence multiple synapses and configured for a specific learning task. A
single astrocyte paired with a single leaky integrate and fire neuron is shown
to converge on a solution in 2, 3, and 4 synapse configurations. Beyond the more
concrete improvements in plasticity, this work provides a foundation for
exploring supervisory astrocyte-like elements in spiking neural networks, and a
framework to implement and extend many three-factor learning rules. Overall,
this work brings the field a bit closer to leveraging some of the distinct
advantages of biological neural networks.

\end{abstract}
%\end{abstract}

% \begin{introductory lists and tabels
%%%%%%%%%%%%%%%%%%%%%%%%%%%%%%%%%%%%%%%%%%%%%%%%%%%%%%%%%%%%%%%%%%%%%%%%%%%%%%%
% Introductory Lists and Tables
%%%%%%%%%%%%%%%%%%%%%%%%%%%%%%%%%%%%%%%%%%%%%%%%%%%%%%%%%%%%%%%%%%%%%%%%%%%%%%%

% Add TOC, list of figures, list of tables in that order
\makealllists

% Add the acronyms
\glsaddall
\printglossary[type=\acronymtype]

% Reset all acronyms
\glsresetall

% Start using Arabic numbers
\mainmatter
% \end{introductory lists and tabels}

	
	% Include Chapters
	% introduction.tex
%
% Author       : James Mnatzaganian
% Contact      : http://techtorials.me
% Date Created : 08/27/15
%
% Description  : Introduction chapter used by "thesis.tex".
%
% Copyright (c) 2015 James Mnatzaganian
%
% Version 2 ::
% Author       : Andres Kwasinski
% Contact      : https://people.rit.edu/axkeec/
% Date Created : 01/09/2020
%

% NOTE: All filler text has "TODO" written. This must be removed in the final copy!

\chapter{Introduction}\label{section:introduction}
\section{Motivation}
Biological neurons and neuronal networks originally inspired traditional,
ANNs. Decades of research and hardware advancements later, ANNs have achieved
and even surpassed humans in specialized tasks. Currently neural networks
outperform humans in object recognition and classification, and a variety of
games (chess, go, some video games). This performance comes at a cost however,
huge networks with millions of parameters are needed. Computing their decision
in real-time is challenging, and if possible requires specialized, power-hungry
hardware. In addition their ability to handle time-series data is limited. The
human brain on the other hand, is able to handle a wide variety of complex tasks
with $\approx$ 20 W of power, and seamlessly handle time-series input. Over the
decades, research in the neuroscience field has advanced as well and it is known
that the perceptron neuron is poor approximation of a biological
neuron. Neuroscientists have also discovered a variety of temporal encoding
schemes beyond just rate-based coding, which initially inspired the
simplifications of the perceptron. This suggests traditional artificial neurons
can't approximate biological neuron communication and encoding. As the
functional gap between biological and artificial neurons is realized it is
important to consider the differences, and possible advantages to closing this
gap. Spiking neural networks, are a step in that direction and there is growing
interest in exploring their properties. Broadly, the goal of such research is to
tease out the elements of biological neural networks that facilitate spiking
neural networks processing of time-series data, in addition to real-time,
on-line training. Achieving these things may require a fundamental shift in the
models used, a shift towards spiking neural networks. Reducing the power
requirement for a given task would be extremely beneficial from an engineering
perspective, but there are further implications. SNNs lend themselves to more
efficient, and simpler hardware implementations compared to ANNs. Reducing the
number of transistors required for a single neuron could open doors for deeper
and larger networks then would be feasible today.
    
\section{Thesis Objectives}
The overarching goal of this thesis is to develop a bio-inspired Astrocyte
model, and more broadly an Astrocyte-Neuron interaction model. This model will
be computationally simple, scalable, and captures common themes within
Neuroscience literature. To direct this research, the developed Astrocyte
element will drive synaptic plasticity; First generalizing, then diverging from
the widely used STDP rule. Benefits of Astrocyte-like control of synaptic
plasticity will be demonstrated in the single synapse, single neuron
configuration (1S1N) as well as select multi-synapse configurations. Inputs will
consist of procedurally generated spiking inputs. This would include poisson
rate-coded spike trains, spiking inputs representing Boolean variables, and
temporal inputs that follow specific patterns. To evaluate each experiment
different intrinsic properties will be observed, such as convergence/divergence
of weights, speed of convergence, stable points across a parameter space, and
lastly performance on any task driven experiments. Statistical analysis of
various intermediate signals, and weights may be employed as well.

More specific objectives.
\begin{enumerate}
  %% Specify -> developing a novel astrocyte model. Stress what is new and value-added
\item To Develop an Astrocyte \Gls{lif}-Neuron model based on bio-chemical pathways
  common in background literature. This Astrocyte should respond internally to
  pre and post-synaptic spikes In a way that can mimic STDP, but is more
  generalized and flexible.

  \item Show, in the case of a single spiking neuron with one input, how the
    developed Astrocyte-\Gls{lif} model can drive synaptic plasticity in a way that
    generalizes, and extends STDP.

  \item Extend the Developed Astrocyte-Synapse Model to include coordination of
    plasticity rules across multiple synapses, and show how this coordination is
    critical to a learning task.

\end{enumerate}

	% background.tex
%
% Author       : Jacob Kiggins
% Contact      : jmk1154@g.rit.edu
% Date Created : 01/09/2020
%

% NOTE: All filler text has "TODO" written. This must be removed in the final copy!

\chapter{Background and Related Work}\label{section:background}
    \section{Spiking Neural Networks}
    Spiking Neural Networks (SNNs) are a type of brain-inspired computing model. This class of
    Artificial Neural Network (ANN) is the most bio-realistic that is used in
    engineering applications. They are an important topic in the machine
    learning field specifically, and have gained interest in recent years. There
    are a variety of motivations for exploring bio-inspired compute models. As
    of 2015 most papers were concerned with the low-power and parallelism
    offered by SNNs \cite{schuman_2017}. These bio-inspired models
    have distinct advantages when looking at hardware implementations, and can
    offer significantly better performance/watt when compared to traditional
    ANNs. In addition, SNNs offer a method of exploring some of the important
    features of biological neural networks that aren't yet possible to replicate
    in artificial networks. These include one-shot learning, short-term memory,
    and problems involving time-series data, where the mammalian brain excels
    \cite{kasabov_2013}.

    \todo{Add figure of SNN topology}
    
    Spiking neural networks, formed by connections between spiking neurons, are
    a bio-realistic type of artificial neural networks. In traditional ANNs real
    valued numbers propagate through the network, are multiplied and added, with
    activation functions applied at the output of each neuron. Spiking neurons
    instead emit voltage spikes. These spikes propagate across a synapse, are
    usually multiplied by a weight, and are sometimes transformed in other ways
    (delay, filtering, etc...). This transformed value is called a post-synaptic
    potential (PSP). PSPs accumulate at the neuron downstream, in the form of
    membrane voltage, or more generically a state variable. Once a certain
    threshold is reached the post-synaptic neuron fires, or emits a voltage
    spike. Since there are many Spiking Neuron models, it is necessary to
    formalize what is considered a spiking neuron. In general they process
    information from one or more inputs, and produce a single spike-like
    output. The probability of firing is increased by excitatory inputs, and
    decreased by inhibitory inputs. There is at least one internal state
    variable, and depending on this variable output spikes are generated
    (normally a threshold) \cite{ponulak_2011}.
    
    There are many potential models that fit the requirements of a spiking
    neuron. In addition, some models are more often paired with other
    bio-inspired components such as Astrocytes, which it is a goal of this work
    to explore. Relevant models are outlined below. These models each balance
    bio-realism with computational efficiency, with some excelling at
    both. Figure \ref{fig:sn_model_compare} shows where a number of models fall
    when measured by these criteria.
    
    \afigw{figures/sn-model-compare.png}{Comparison of Spiking Neuron
      Models}{fig:sn_model_compare}{4/3}

    %%%%%%%%%%%% Spiking Neuron Models %%%%%%%%%%%%
    \subsection{Leaky Integrate and Fire Neurons}
    The most popular Spiking Neuron model used in the literature is by far the
    Leaky Integrate and Fire or LIF neuron. In this model the state variable is
    represented by an internal voltage, which is increased as current flows into
    the post-synaptic neuron, and charge accumulates. Current leaks from this
    internal reservoir lowering the voltage. Equation \ref{eq:lif} defines the
    behavior.
    
    \begin{align}
        C \frac{du}{dt}(t) = -\frac{1}{R}u(t)+(i_o(t) + \Sigma w_ji_j(t))
    \end{align}
    
    Once the state variable u (generally thought of as a voltage) reaches a
    certain threshold the neuron outputs a spike. After this time the state
    variable is reset, and is held at that value for a period known as the
    absolute refractory period. This mimics the behavior of biological neurons
    \cite{ponulak_2011}.

    \todo{Add 2 tau model from Norse}

    \subsection{FitzHugh - Nagumo model}
    The FitzHugh–Nagumo is a common Spiking neuron model that is coupled with
    Astrocytes in a few works that will be explored later (\cite{postnov_2009},
    \cite{postnov_2007}). It is based on the highly (computationally) complex
    Hodgkin-Huxley model, with a variety of parameters fixed and others tuned to
    get different firing patterns. \cite{postnov_2009} outlines the membrane
    voltage dynamics, with a tanh activation function forming the synaptic
    coupling between neurons (or neuron and Astrocyte).

    \begin{align}
      \epsilon \frac{dv_1}{dt} = v1 - \frac{v_1^3}{3} - w_1 \\
      \frac{dw_1}{dt} = v_1 + I_i - I_{app} \\
      \tau_s \frac{d_z}{dt} = (1 + tanh(S_s(v_1 - h_s)))(1 - z) - \frac{z}{d_s}
      \\
      I_{syn} = (K_s - \delta G_m)(z - z_0)
    \end{align}

    The constant $\epsilon_1 = 0.04$ is the time-separation parameter and
    $I_1 = 1.02$ defines the operating regime as excitatory. $H_s$ and $D_s$
    control activation and relaxation of the neuron. $I_{app}$ represents the
    input to the synapse at all sources.

    \todo{Add Image of simulation postnov 2009 pg 492}
    \todo{Add sections for SRM and Izh}
    
    \subsection{Spiking Neural Network Topologies}
    Spiking networks can take on a variety of topologies, similar to traditional
    ANNs. Feed forward, Recurrent, and a Hybrid of both. One, more specific
    hybrid is a Synfire chain, which is feed forward between sub-populations of
    recurrent networks. Within a standard feed-forward network lateral
    inhibition is often employed, implementing a Winner Take All (WTA)
    configuration. In this case a layer of neurons has feed forward connections
    from the previous layer, in addition to inhibitory connections within the
    layer. The result is the first neuron to emit a spike prevents all other
    neurons within that layer from doing so \cite{ponulak_2011}.
    
    Synfire chains are of particular interest, as they show a coordination
    within the spiking network. From input to output, in a feed-forward pattern,
    populations of neurons coordinate, passing along a "packet" of neuronal
    activity. It was observed in monkey's, that the precise timing of spikes was
    correlated with behavior, and synfire activity was given as a possible
    explanation \cite{aertsen_1996}.

    
    %%%%%%%%%%%% Coding Schemes %%%%%%%%%%%%
    \section{Coding Schemes}
    With the addition of a temporal dimension in spiking neural nets, there are
    a variety of ways to encode information. The coding schemes used in research
    are generally based on observations in biology \cite{ponulak_2011}.
    
    Rate-based coding is based on some of the earliest observations of sensory
    neuron activity. As pressure on a tactile nerve was increased researchers
    observed an increase in firing rate of that receptor neuron. This translated
    into rate-based coding in artificial SNNs \cite{ponulak_2011}. Rate-base
    coding couldn't explain certain observations in biology however. Some
    responses were too fast for the neurons involved to estimate the firing
    rate. In addition, dynamic responses have been observed in the primary
    auditory cortex without a changing of firing rate, but instead, firing
    pattern within a sub-population of neurons. More specifically, the relative
    timing of two or more spikes can encode information. This is very
    advantageous, since a spike lasts $\approx 10^-3s$, but relative timing
    between spikes can distinguished down to $10^-8s$ \cite{ponulak_2011}. This
    greatly increases the bandwidth of communication, and theoretical minimum
    latency. The general trend, is that any neurological system where processing
    speed is especially important, will tend to rely on a spike timing based
    encoding for information.
    
    A variety of temporal coding schemes have been proposed, and
    investigated. Time to first spike, as the name suggests, encodes information
    in the latency between stimulus and the first spike. This is accomplished by
    using a neuron model that has an inhibitory feedback connection, suppressing
    additional spikes. Rank order coding (ROC) represents information in the
    order of spikes within a population of neurons, that each emit a single
    spike. This allows fast information processing, since the time between
    spikes isn't important, and can be very small. This scheme was proposed as
    an explanation for ultra-fast processing in the primate visual
    cortex. Similar to ROC, with latency coding, information is represented in
    the time between spikes of a population of spiking neurons. As with ROC,
    each neuron emits a single spike. This encoding scheme is supported by
    evidence of biology, where it has been observed that moving the timing of a
    single spike by ~10ms can change downstream activity \cite{ponulak_2011}.
    
    These include rate-based coding, where firing rate over some window
    indicates activity. Time to first spike, or the latency between the start of
    stimulus and the first spike of a neuron. This has been observed in tactile
    sensing neurons in humans \cite{ponulak_2011}.
    
    %%%%%%%%%%%% Encoding Real Values %%%%%%%%%%%%
    \subsection{Encoding Techniques}
    In general, data being presented to a spiking neural network will take the
    form of a real value. This is true of images, sound, values from various
    integrated sensors, such as temperature or acceleration. A method is needed
    that can convert these real values into a spike, or spikes, as defined by a
    coding scheme.
    
    \subsection{Poisson Encoding/Decoding}
    A very popular method of encoding real values into spikes , or just to
    generate random spikes, is treating spikes as a Poisson process. In general,
    the rate parameter is either random, or equal to the real value being
    encoded. There are two main approaches for sampling spikes from a Poisson
    distribution. The first calculates the probability of a spike in a given
    time-step, with the real value as a firing rate. The resulting probability
    is compared to the result of sampling a uniform distribution sampled once
    for each discrete time step. If the uniformly random value is below the
    Poisson probability a spike is inserted, at that time. One potential issue
    with this, relates to numeric precision. If the simulation time step is
    small in comparison with the rate parameter, then the probability of an
    event may be quite small. Go small enough, and the uniformly random number
    won't have enough precision to ever dip below the threshold. In this case no
    spikes would be produced.
    
    \todo{ref  https://www.cns.nyu.edu/~david/handouts/poisson.pdf}
    
    \begin{align}
        P(k) = \frac{\lambda^k e^{-\lambda}}{k!} \\ P(1) = \lambda e^{-\lambda}
        \\ spike_i = P(1) > X_i
    \end{align}
    
    The next, and more numerically stable method, is to use the exponential
    distribution, or Poisson waiting time to sample intervals between
    spikes. Interval times are randomly sampled from the exponential
    distribution.
    
    \begin{align}
        interval = \lambda e^{-\lambda x} \\ x = rand(0,1) \\ spike_i = \lambda
        e^{-\lambda x_i}
    \end{align}
    
    %This technique generates a spikes train as shown in the following graph.
    
%    \afig{figures/encode_poisson_sweep.png}{Poisson encoded spike train for real values between 0 and 1}{fig:enc_poisson_sweep}
    
    To determine how different operations process data within a spiking neural
    network, it is useful to be able to decode a spike train into real
    values. This involves finding the most likely $\lambda$ parameter for given
    spike train. Using maximum likelihood estimation this is found to be the
    inverse of the sample mean.
    
    \begin{align}
        \lambda = \frac{N}{\Sigma x_i}
    \end{align}
    
    This decoding process is subject to error, which improves with sample time.
    
    \subsection{Temporal Coding}
    There are a variety of time-based (vs. rate-based) coding schemes, each
    slightly different, they each rely on precise spike timing to encode
    information. Beyond a qualification of these methods, a precise definition
    (or definitions) need to be proposed and used during simulation.
    
    \subsubsection{Rank Order Coding}
    Information is encoded in the relative spike time within a population of
    neurons. Encoding itself is fairly straight-forward. Real
    valued inputs are transformed into spike latency, and for a given sample
    neurons fire once. Generally larger values are assigned a smaller latency,
    and smaller values a larger one \cite{delorme_2001}.
    
    Some changes to the network itself have shown success when paired with rank
    order coding. These Include a progressive desensitization, similar to the
    biological concept of shunting inhibition. Equation \ref{eq:roc_activation}
    shows the activation of a neuron, given a modulation factor $\alpha \in
    (0,1)$ and a set of incoming connections $a$. Subsequent spikes along a
    given input connection have less and less of an affect on the overall
    activation of the neuron \cite{delorme_2001}.
    
    \begin{align}
        Activation(i,t) = \Sigma_{j \in
          [1,m]}\alpha^{order(a_i)}W_{j,i} \label{eq:roc_activation}
    \end{align}
    
    %%%%%%%%%%%% Convolution in SNN %%%%%%%%%%%%
    \section{Convolutions With Spiking Networks}
    Convolution is not as straightforward to implement in the spiking domain,
    when compared to traditional ANNs. There are certainly many ways one could
    implement convolution. Including, in the same way they are handled in ANNs,
    where neurons aren't explicitly involved. However, the goal is to be
    biologically plausible. Observations in the human visual cortex inspired the
    tiling window approach to convolution \cite{wang_2016}. Instead of applying
    an activation like ReLU, the activation is a spiking neuron. In addition,
    instead of defining convolution as a time-based operation, it is implemented
    in space, where the kernel is represented by same-valued sets of weights,
    connecting feature maps, represented by spiking neurons
    \cite{mozafari_2018}. Training the weights of a convolution,
    
    %% \section{Reservoir Computing}
    %% A reservoir is a type of recurrent neural network architecture. There are
    %% many variants, but they all share a common theme. The reservoir is made of
    %% multiple non-linear units (neurons). These units are randomly connected with
    %% some probability. Inputs are fed into the reservoir, and a layer of neurons
    %% receiving output from the reservoir act as readouts. The shape and level of
    %% connectivity into and out of the reservoir, as well as the connectivity
    %% within the reservoir are tune-able parameters. There are to major benefit
    %% that have driven research into reservoir computing. One, is that SOTA
    %% results can be achieved with only a single trainable layer, the readout
    %% layer. The reservoir is able to extract features from input data in an
    %% unsupervised way, allowing a single layer to perform classification. In
    %% fact, multiple tasks can be performed on the same input, using the same
    %% reservoir with different readout layers. Second, is that reservoirs have
    %% memory, and can seamlessly handle time-series data \cite{schrauwen_2007}.
    
    %% \subsection{Liquid State Machines}
    %% LSMs are a type of reservoir computing architecture, built from spiking
    %% neurons. In general they are formed from random between neurons in some a
    %% pool. Neurons of the LSM take in some input, which is then projected to some
    %% transient internal state. Readout neurons tap into a feature representation
    %% derived by the LSM, and produce a linear readout. Classification performed
    %% on this readout layer generally provides good results, and with only one
    %% trainable layer.  LSMs naturally have a fading memory, and lend themselves
    %% to time-series data. \cite{wang_2016}
    
    %%%%%%%%%%%% SNN Learning Approaches %%%%%%%%%%%%
    \section{SNN Learning Approaches}
    In biology, the changing of synaptic weights is referred to as synaptic
    plasticity, and is considered to be one method that facilitates learning and
    memory. Changes in plasticity can be quick, such as with pulse-paired
    facilitation (STDP-like learning) or more gradual, such as with long-term
    potentiation \cite{ponulak_2011}.
    
    %%%%%%%%%%%% STDP %%%%%%%%%%%%
    \subsection{STDP}
    One of the first learning approaches considered for SNNs is Spike Timing
    Dependant Plasticity or STDP. This is an unsupervised approach which
    implements Hebbian learning. Simply stated, if a pre-synaptic spike appears
    to illicit a post-synaptic spike (within some time window) the weight of
    that connection is strengthened. If the opposite is true, the weight is
    decreased. That process can be reversed (decrease instead of increase
    weights), a process known as anti-STDP. This approach lends itself to a
    variety of pattern recognition problems, but is not well suited for some of
    the more traditional applications of neural networks
    \cite{tavanaei_2019}. Several multi-layer SNN architectures have seen success
    in recognition tasks. However, in these experiments only the last layer was
    trained using STDP. One of the distinct advantages of STDP is the ease with
   which it can be implemented in hardware, this property is a driving force
    for continued research.
    
    %% \todo{cite https://arxiv.org/pdf/1611.03000.pdf}.
    
    With STDP, weight initialization is very important, \todo{add more detail
      and cite "A critical survey of STDP in Spiking Neural Networks for Pattern
      Recognition"}
    
    STDP is a biologically inspired learning mechanism for spiking neural
    networks. Equation \ref{eq:stdp} outlines the rule for connection weight
    updates
    
    \begin{align}
        \Delta W =
        \begin{cases} 
          Ae^{-\frac{|t_pre-t_post|}{\tau}} & t_{pre} - t_{post} \leq 0, A > 0
          \\ Ae^{-\frac{|t_pre-t_post|}{\tau}} & t_{pre} - t_{post} > 0, B < 0
       \end{cases} \label{eq:stdp}
    \end{align}
    
    Through the approximation of sigmoidal neurons, spiking networks can
    represent arbitrary mappings, and satisfy the universal approximation
    property. Spiking networks are less useful when organized in this way, as it
    removes any advantage associated with the dynamic nature of their
    behavior. Instead, architectures taking advantage of dynamic temporal coding
    of spiking neurons, with weight updates governed by STDP leverage spiking
    neurons more effectively. In this configuration a new question arises. What
    mappings can STDP learn? This question is explored in
    \cite{legenstein_2005}. Their approach is to consider the possible weight
    values at the steady state after training. It is well known that STDP will
    force weights to one or the other extreme, either maximally excitatory, or
    maximally inhibitory. This restricts the mappings to those the could be
    learned by STDP, and the question then becomes can STDP learn each
    mapping. To test this researchers sampled mappings randomly, and applied a
    forced learning approach. The output spikes (of the output layer) are fixed
    to the ideal values, which affect updating of the previous weights. All
    other weight updates proceed in an unsupervised manor. The results suggested
    that on average, a spiking network trained using STDP can represent an
    arbitrary mapping, within its training domain.
    
    Recently, there has been some success with a supervised variant on STDP,
    known as reward-modulated STDP. In this case a deep convolutional spiking
    neural network (DCSNN) was applied to the MNIST digit recognition
    task. Early layers of the network were trained in an unsupervised manor,
    with normal STDP. Later layers were updated using reward-modulated
    STDP. That is, if the output was correct, STDP was applied. If not,
    anti-STDP was applied. This learning approach has basis in biology, with the
    reward modulation mimicking the activity Dopamine and Acetylcholine
    (ACh). As a proof of concept, researchers applied their learning method to a
    shallow network, with a single trainable layer. They used rank order coding,
    and at most one spike per neuron. The output class is determined by which
    neuron in the output layer exhibits a spike first. This architecture gave
    passable results, but wasn't well suited to multi-layer training.
    
    For their deep architecture, shown in figure \ref{fig:rstdp_dcnn} better
    results, an accuracy of 97.2\% which is on par with the state-of-the-art was
    achieved. \cite{mozafari_2018}.
    
    \afig{figures/rstdp_dcnn_arch.png}{Spiking DCNN
      Architecture}{fig:rstdp_dcnn}{0.6}
    
    
    %%%%%%%%%%%% ReSuMe %%%%%%%%%%%%
%    \subsection{Teaching-Signal Rules}
%    \todo{populate this section} \cite{ponulak_2011} - 417
    
    %%%%%%%%%%%% Back Propagation %%%%%%%%%%%%
    \subsection{Back Propagation}
    
    Back-propagation is traditionally difficult to implement in spiking neural
    nets, as the spike function is not differentiable. There are some shortcuts
    proposed, such as using the membrane voltage derivative instead of spike
    output. However applying backprop may not be the best approach. The brain
    doesn't (as far as is known) have a method of back-propagating errors. It
    has been proposed to adapt the network architecture to allow rewards at the
    output to affect all portions of the network. A hybrid reward-STDP approach
    shows promising results \cite{tavanaei_2019}.
    
    %%%%%%%%%%%% Deeper Networks %%%%%%%%%%%%
    \section{Deeper Networks}
    Deep learning is still largely unexplored in Spiking Neural Networks, as
    well as multi-layer learning in the deep architectures that have been
    explored. Motivating further exploration is both the success of deep
    Artifical Neural Networks, and the observation that the mammalian brain
    relies on a deep architecture for visual tasks such as detection and
    recognition \cite{tavanaei_2019}.
    
    \section{Notable Experiments}
    A number of papers have shown promising results on SOTA recognition tasks,
    including MNIST written digit classification \cite{mozafari_2018}. As well as
    facial recognition \cite{delorme_2001}. These papers, among others use
    hand-crafted Difference of Gaussian's filters in early convolution
    layers. End-to-end trainable SNN architectures, for image processing tasks
    appear to be lacking in literature.
    
    \cite{mozafari_2018} has shown that a mixture of DoG filters, STDP, and
    reward-modulated STDP can achieve SOTA results on MNIST. The network wasn't
    end-to-end trainable, but did have a neuron-based readout, requiring no
    external classifier, such as SVM.
    
    \cite{delorme_2001} uses ROC along with some architectural and learning rule
    modifications, to better support the coding scheme. The goal of this paper
    is to show that features can be extracted from ROC encoded images after DoG
    filtering. 3x3 DoG filters were used, followed by ROC encoding, and then 32
    maps generated using an 11x11 receptive field. Connections from DoG maps to
    LIF neurons are organized in such a way to implement convolution, but still
    be trainable. If weights are updated in one region, due to training, the
    weights are simultaneously updated in the other regions. In addition, there
    is inhibition between feature maps. If a neuron fires within one map, the
    corresponding neuron is inhibited in the others. The goal was to force maps
    to learn unique features. Results were promising, and activation of LIF
    neuron maps showed selectivity for contour orientation, end-stop and blob
    types.
    
    %%%%%%%%%%%% Astrocytes Intro %%%%%%%%%%%%
    \section{Introduction to Astrocytes}
    Astrocytes are a type of glial cell found in mammalian brains. Their
    structure and function are still the topic of cutting edge research
    today. It is known that they are vital for normal brain functions, including
    cognition and behavior \cite{mederos_2018}. In the human brain astrocytes
    are known to tightly wrap many synapses, as well as dendrites and cell
    bodies. Since the Astrocytes engulf the synapic cleft, they are ideally
    placed to control extracellular neurotransmitter and ion
    concentrations. There is substantial evidence that they regulate
    extracellular K+, which is required for propagation of action potentials
    through neuron bodies. GABA transporters in astrocyte cell membranes serve
    to clean up neurotransmitters, and can help mitigate exitotoxcity. It should
    be noted that Astrocytes do not wrap, or even influence every synapse, and
    that their density varies widely depending on the region of the brain. More
    interesting, their morphology can change throughout life, and in response to
    other bodily functions, such as food intake \cite{mederos_2018}.

    From a computational point of view, at a high level, Astrocytes listen and
    respond to activity at the synaptic cleft. In addition, input to the
    Astrocyte (and subsequently the response) comes not only from local
    activity, but regional, including all synapses associated with a given
    astrocyte, and signals that propagate from other Astrocytes
    \cite{min_2012}. Activity at the synaptic cleft that Astrocytes are
    sensitive to include pre and post synaptic potentials, via neurotransmitter
    uptake. In general, the neurotransmitter in question is Glutamate for
    excitatatory transmission, and GABA for inhibitory. The Astrocyte will then
    respond with one or more Gliotransmitters. These include glutamate,
    D-serine, ATP or TNF-alpha. These neurotransmitters have different effects
    depending on where and when they are released.
    \begin{itemize}
      \item Glutamate: When released to the pre-synaptic neuron Glu generally
        results in an increase in probability or release (PR). This equates to
        an increase in the weight for a computational model. At the
        post-synaptic terminal, Glutamate release results in depolarization, or
        a so called slow inward current (SIC) \cite{pitta_2016}.
      \item D-Serine: this nurochemical gates the NDMAR receptors, and
        subsequently gates LTP/LTD \cite{mederos_2018}.
      \item ATP has a depressive affect, opposite to Glu \cite{mederos_2018}
      \item TNF-alpha results in an increase in the number of post-synaptic
        surface Glu receptors, and a decrease in GABA receptors
        \cite{chung_2015}.
    \end{itemize}

    Gliotransmitter release from an Astrocyte is dependent on the integration of
    synaptic activity over 100s of milliseconds (for a local response)
    \cite{pitta_2016} to 10s of seconds for a whole cell response
    \cite{mederos_2018}. This integration is mediated by a variety of input
    pathways, some of which are more thoroughly understood than others. These
    input pathways generally converge to provide an increase to intracellular
    Ca2+ concentration, within an Astrocyte. Though there is some controversy
    surround Ca2+ signaling within astrocytes \cite{mederos_2018}. Initial
    results of invitro experiments showed a slow Ca2+ response in Astrocyte some
    following intense neuronal activity. This showed that while Astrocytes were
    active, they could not exert rapid or granular control over information flow
    at the synapse. Further, more recent research has shown that there are
    syntactically local Ca2+ responses that are much quicker and respond to
    lower levels of activity \cite{araque_2014}. The bridge between local and
    cell level response is mediated by Calcium Induced Calcium Release
    (CICR). As local activity increases beyond a certain threshold, the Calcium
    concentration causes additional Ca2+ release from the ER. This CICR
    propagates like a wave, and can reach the main cell body, other astrocyte
    processies, and even other Astrocytes \cite{manninen_2018}.

    %% TODO: Add in the CICR ER picture, it shows how calcium waves can
    %% propagate to the cell soma
    %% TODO: More detail here

    \section{High Level Roles for Astrocytes In Biology}
    
    There are a variety of theories as to the functional role of Astrocytes,
    with some being somewhat wild speculation, and others specific and highly
    grounded in experimental data.

    %% Astrocytes are Master Integrators
    Astrocytes have been shown to act as integrators of synaptic activity. Early
    studies discovered that Astrocytes become active in response to intense
    neuronal firing. Subsiquent reasearch shows that this response, represented
    by Calcium concentration, was highly complex in time
    \cite{araque_2014}. This complex global response which happens on a seconds 
    timescale is preceded by calcium activity in Astrocyte processes local to
    synapses. Though there isn't direct evince of this, it is thought that these
    local Ca2+ sparks propagate their way to the main cell body, and induces the
    response observed at the cell body \cite{araque_2014}. In this way
    Astrocytes can locally integrate neuronal activity temporally, on a 100s of
    milliseconds time-scale. These local events can then be temporally and
    spatially integrated on a seconds timescale. There is some evidence, when
    looking at Astrocyte mGluR receptors, that along a common astrocye process,
    there are clusters of receptors which form a local region. Each of these
    regions is associated with synaptic terminals, and may integrate their
    activity at different spatial and temporal time-scales as their neighbors
    \cite{pitta_2012}. Furthermore, experiments with computational models show
    that varying astrocyte modalities (as is the case in the brain) leads to
    different patterns of signal transmission. This could indicate that
    Astrocytes in different areas of the brain perform different functions
    \cite{pitta_2012}.

    %% Astrocytes facilitate long-range spatial influence
    A single Astrocyte, through its end-foot processies can influence many
    synapses simultaneously. In the brain, Astrocytes are physically distributed
    via a mechanism called contact spacing, where their end-foot processes
    connect at the periphery of an Astrocyte's domain \cite{pitta_2012}. This
    spacing is not always uniform, some micro-domains are formed favoring a
    neuron signal pathway, with adjacent astrocyte connections observed to be
    absent. One example is in the Ferret visual cortex, where astrocytes (like
    Neurons) for receptive fields on the visual input \cite{pitta_2012}.

    Signals propagate in these Astrocyte networks (AN) in a few different ways. One
    pathway, involves IP3 diffusion through the AN's gap junctions. Once across
    sufficiently high levels of IP3 cause CICR, and in turn Astrocyte
    waves. Ca2+ may also diffuse across gap junctions in the same way. This
    propagation is dubbed Calcium Waves, due to the wave-like nature of the
    propagation. In addition to the gap-junction mediated effects, ATP released
    from an Astrocyte may diffuse extra-cellularly, and influence other
    Astrocytes \cite{amiri_2013}.

    
    %% Astrocytes Modulate STP/STD
    \cite{pitta_2012} explores the pre-synaptic Glu mediated Astrocyte
    stimulation and response loop, as a mechanism for modulation of short term
    plasticity. During normal neuronal activity, the current value of the PR, as
    well as the state of the pry-synaptic astrocyte tend to bias a synapse
    towards STD/STP. For depression, the PR is generally above some
    threshold. With a higher PR value the NT resources at the pre-synaptic
    terminal are depleted more quickly, leading to an overall
    depression. Conversely if the PR is low the synapse will tend toward
    STP.

    %% Astrocyte Modulation of LTD/LTP
    Astrocytes have been shown to play a key role in the bio-chemical pathways
    that lead to STDP at neuronal synapses, and can gate LTD/LTP via D-Serine
    release \cite{manninen_2019}. Beyond passively participating in the normal
    STDP, astrocytes may gate it's activity, or reverse the polarity,
    implementing anti-STDP \cite{min_2012}. In addition, Glutamate released by
    astrocytes in response to pre and post-synaptic activity may strech or shift
    the traditional STDP curve in both experimental results, and computational
    models \cite{pitta_2016}.

    Astrocytes can modulate the concentration of their surface glu transporters,
    this in turn modulates the level of glu spill-over, beyond the synapse. This
    in-turn increases the excitability (depolarizing neurons partially towards
    firing) post-synapse for a local region of synapses. Astrocytes may also
    respond to glutamate activity (as sensed by transporter uptake) by releasing
    more glutamate, or by releasing ATP, based on signaling frequency. Within
    the Hippocampus, it has been observed that GABA release from a pre-synaptic
    neuron can be taken up by GAT-1 and GAT-3 transporters on an Astrocyte's
    surface. This leads to an increase in Na+/$Ca^{2+}$ within the Astrocyte,
    and the eventual release of ATP. ATP acts as an ihibitor, down-regulating
    excitatory transmission \cite{mederos_2018}.
    
    Astrocytes also play a major role is Spike-Dependant plasticity, mediating
    LTD/LTP at the synapse and gating it. Both LTD and LTP are controlled by the
    NMDAR receptor, with the distinction between depression and potentiation
    being the presence of K+ ions, indicating post-synaptic
    depolarization. Figure \ref{fig:astro_plastic} shows this pathway for
    LTD. When a post-synaptic depolarization is followed closely by pre-synaptic
    depolarization a signaling pathway proceeds, leading to glutamate release by
    the astroctye, which in turn triggers NDMAR receptors in the pre-synaptic
    neuron. The influx of ions leads to a decrease in pre-synaptic
    neurotransmitter release probability (through some mechanism not described
    here) \cite{min_2012}.
        
    \afig{figures/astrocyte_ltd_ltp.png}{Astrocyte-MediatedPlasticity}{fig:astro_plastic}
    
    \section{Biochemistry Inspires Computational Models of Various Pathways}
    Over the last 30 years there have been experiments and study surrounding the
    role of Astrocytes in the mammalian brain. This research has lead to the
    identification of a variety of signaling pathways, along with some
    quantitative data. These new insights in the neuroscience field, have lead
    to a wave of computational models, which attempt to either mimic
    Neuron-Astrocyte interactions, or extract some computational benefit by a
    simplified, but still very bio-plausible model. The first clue to Astrocyte
    involvement in computation was the observation of Glutamate released from
    the pre-synaptic neuron, and subsequent increase in $Ca_2^+$
    concentration within astrocyte cell bodies and foot processes. Around 2010,
    new experimental data emerged from in-vivo studies, which furthered the
    functional understanding of Astrocyte signaling \cite{manninen_2018}. Some
    pathways are responsible for modulating plasticity while other lead to
    transient changes in neuron dynamics. A key step in developing a
    computational model, that captures the key features involved in information
    processing, is understanding the biological pathways.


    %% TODO: Add pathway figure
    
    Figure \ref{fig:astro_pathways} shows the possible signaling pathways at the
    Tripartied synapse. Each pathway has some underlying research into the
    behavior at that pathway, either in vitro or in vivo. This experimental data
    has led to a variety of computational models, which share a common theme but
    differ in complexity, level of bio-realism, and computational efficiency.

    \subsection{Neuron-Astrocyte Pathways - Presynaptic to Astrocyte}

    One of the first experimental observations of Astrocytes was a transient
    increase in Ca2+ in response to high levels of neuron activity. One of the
    main pathways mediating this response is the Pre-synatic Glu
    pathway. Excitatory cortical neurons release Glu in response to input
    stimulus. Glu binds to the G-couple mGluR receptors receptors on the surface
    of an astrocyte. This sets in motion a cascade involving the IP3 second
    messenger, which ultimately leads to Ca2+ release in the astrocyte soma
    \cite{pitta_2012}. IP3 induced Ca2+ release from the ER leads to a rapid
    breakdown of IP3, creating a kind of local calcium spike. This Ca2+
    concentration can integrate within the cell soma, but does degrade due to
    the activity of pumps at ER surface \cite{pitta_2012}.

    To model this \cite{pitta_2009} uses a three variable approach, which
    extends the Li–Rinzel model with bio-realistic IP3 dynamics. In general Ca
    concentration (when considering the IP3 mediated pre-synaptic pathway only)
    is dependant on two internal activites, and one external. Internally, there
    is a $J_{leak}$ factor, describing a differential based leak from the ER
    into the Astrocyte Cytosol. To override this leak, and maintain the
    differential, SERCA pumps move Ca2+ into the ER. This is noted by
    $J_{pump}$. $J_{chan}$ Accounts for the flux of Ca2+ from the ER into the
    cytoplasm due to IP3 levels, or CICR, depending on the Ca2+ concentration
    \cite{pitta_2009}. In this case IP3 levels are a function of glutamate
    release from the pre-synaptic neuron.  \cite{pitta_2016} extends this model,
    including the Glio-transmitter release from the Astrocyte in response to
    Ca2+ transients. In their work a spike of GT is released (similar to a
    neuron) when Ca2+ concentration reaches some threshold. It should be noted,
    that the IP3/Ca2+ spiking response (IP3 -> Ca release -> IP3 cleanup + Ca
    cleanup) is independent of the GT release, meaning there can be
    sub-threshold Calcium spikes).

    The above outlines the common theme for presynaptic Astrocyte
    modulation. The general form of the pathway outlined above is shared by
    \cite{postnov_2009}, and \cite{wade_2011}. Though there are variations among
    modeling approach which will be covered in further detail.

    
    \subsection{Neuron-Astrocyte Pathways - Postsynaptic to Astrocyte}
    
    Another common pathway that is explored both in neuroscience and
    computationally is the ``fast'' post-synaptic pathway \cite{bassam_2015}. In
    this pathway, the post-synaptic neuron fires, and subsequently releases
    K+. This K+ spillover is quickly shuttled into the Astrocyte, and causes
    depolarization. Voltage-gated channels on the ER then lead to Ca2+ release
    into the cytoplasm. It is considered fast pathway, because the effect of K+
    is direct, vs. the pre-synaptic pathway involving a second messenger
    \cite{bassam_2015}. The astrocyte to post-synaptic pathway

    \subsection{Calcium and Other Messenger Dynamics}
    Calcium, IP3 and K+ are the main substances involved in signaling pathways
    within the Astrocyte. What isn't immediately obvious unless one is paying
    close attention to the modeling (or more bio-realistiic) equations, is that
    there are spikes and thresholds within these internal Astrocyte
    concentrations. Neuroscience experiments involving monitoring of astrocyes
    revealed oscillations, which are more intense and propagate further
    depending on Activity. These oscillations are the consequence of non-linear
    negative feedback, leading to fast cleanup, and spike-like behavior
    \cite{postnov_2009}. The second messenger IP3 has a similar behavior, with a
    threshold and tanh (with decay factor) resulting in spike-like behavior
    \cite{postnov_2009}.

    \cite{wade_2011} cites similar behavior, except it is the output GT release
    that exhibits spiking behavior, where there isn't spiking observed in the
    calcium. In this model, like with \cite{postnov_2009} there is a threshold
    gating GT release.

    \cite{pitta_2009} introduces the AM/FM variants on the $J_{pump} - J_{chan}
    - J_{leak}$ model, which exhibits spiking and non-spiking calcium dynamics,
    depending on choice of parameter. This is the same model outlined in \cite{wade_2011}.

    It is also well established that the IP3 pathway is operates
    on a slower time-scale, then more direct pathways such as ATP or K+
    \cite{postnov_2009}, \cite{bassam_2015}.


    \section{Evolution of Astrocyte Models}
    \cite{manninen_2018} provides a very insightful review of astrocyte models
    from 1995 until about 2017, and, more importantly an analysis on the origin
    of models for many papers considered. In this work, four different core
    Neuron-Astrocyte models have been identified in the literature over the
    time-period considered. These astrocyte models are modified throughout the
    works, to show different effects.

    \subsection{Foundational Astrocyte Models}
    \cite{manninen_2018} Grouped models from early works that appeared to
    preceed many other models (qualitatively) through 2017: One of these groups
    consists of models defined by De Young and Keizer (1992)a and Li and Rinzel
    (1994). De Young and Keizer like models are characterized by a 
    
    
    \section{Astrocyte Mediated Effects}
    \subsection{GT Release From Astrocyte}
    To signal back to neurons, Astrocytes release various gliotransmitters,
    which were outlined above. In general, this release is dependent on the
    concentration of Ca2+ local to the synapse. Input pathways converge with
    Ca2+ concentration, and then divergent effects are observed via the release
    of multiple gliotransmitters.

    ATP release from astrocytes can travel between itersiticial spaces and
    effect synaptic transmission at physically local
    synapses. \cite{postnov_2009} explores the phenomenon of hetero-synaptic
    suppression, which is mediated by Glu and ATP release. Their experiment
    depends on an existing computational model, with some modifications to
    suport ATP release. A synapse local to an Astrocyte is potentiated in the
    short term via glutamate release. Over a longer time-scale, ATP is released
    from the Astrocyte and diffuses to a neighboring synapse, decreasing
    synaptic PR. This results in the signal generated at N1 is passed through to
    the downstream neuron, where the N3 signal is suppressed.

        
    \section{Astrocyte Networks}
    Astrocytes form their own networks, separate from the connections of
    neurons. Gap junctions between cells can pass various molecules including
    ions and secondary messengers. Gap junctions are not evenly distributed, or
    random, meaning they form meaningful connections between specific
    astrocytes. These networks appear to form non-overlapping territories, where
    groups are interconnected, but distinct from other
    groups\cite{mederos_2018}. The shapes of Astrocyte networks are varied,
    and in the visual cortex consist of between 2 and 10 Astrocytes. In
    addition, Astrocyes are generally not found un-coupled in the brain, further
    supporting the significance of Astrocyte networks \cite{postnov_2009}.
    
    There is evidence supporting changes in these connections (gap-junctions
    between astrocytes), indicating plasticity within Astrocyte networks. It has
    been demonstrated that Astrocyte gap junctions are sensitive to dynamic
    properties within the cell, such as intra-cellular $Ca^{2+}$ concentration
    and pH. 

    The functional role, and information processing implications of Astrocyte
    networks isn't yet well understood. There are a few consequences of
    Astrocyte networks that have been observed and modeled however. First, these
    networks provide additional pathways for Calcium waves to propagate, and in
    some cases form cycles, where astrocyte waves return to their place of
    origin. In addition, under intense neural activity, ANs support far reaching
    synaptic modulation \cite{postnov_2009}. There have been attempts at
    modeling astrocyte network, with variying degrees of
    complexity. \cite{postnov_2009} developed a random procedural algorithm for
    generating a 2D bio-plausible Astrocyte network. In this case some
    experimental results were reproduced, such as the far-reaching propagation
    of high intensity firing, and Calcium waves. Other works, such as \cite{gordleeva_2021} employ
    Astrocyte networks to support working memory.

    
    \section{Related Work - Organized By Paper and Author}
    
    There have been a variety of works attempting to bring an Astrocyte model
    into ANNs. In general these models were either not biologically inspired
    enough, or too complex in their modeling of Astrocyte activity for real-time
    and scalable use \cite{bassam_2015}. These models generally consisted of,
    in the simple case, a supervising element per neuron. If this neuron emits
    sufficiently many spikes in some time-frame, the Astrocyte will increase the
    downstream weighs for that neuron. Alternatively if a neuron is inactive,
    measured by fewer than some number of spikes within a time-frame, downstream
    weights will be decreased \cite{mesejo_2015}. Models like this one
    however, have no concept of calcium signaling, and can't easily support
    Astrocyte networks.
    
    \cite{bassam_2015} proposes a Spiking Response Model which uses $Ca^+$
    concentration as an internal state variable, and incorporate multiple
    pathways to affect a change in $Ca^+$. Figure \ref{fig:srm0} gives a
    high-level overview of this model. Spikes from some number of neurons enter
    the synapse. A key difference in this work is the emphasis placed on the
    difference in time-scales between the Pre-synaptic slower Glu -> IP3 -> Ca2+
    pathway, vs the postsynaptic faster K+ -> Ca2+ pathway.
    
    \begin{align}
        V_j(t) = \eta(t - t_j) + \Sigma^{N_{i+1}}_{i=1}W_{ij}\Sigma^{K_i}_{k=1}
        \epsilon(t - t_i^k - d_{ij})
    \end{align}
    
    Where $V_j(t)$ represents the membrane potential of a neuron, with $K_i$
    spikes coming from each of $N_{i+1}$ neurons. The voltage change associated
    with each spike is the product of the weight $W_{ij}$ and a response kernel
    $\epsilon$.
    
    \afigs{figures/snn_model.png}{Spiking Response Model}{fig:srm0}{0.6}
    
    The spike response kernel $\epsilon$ is defined by Equation \ref{eq:srm_eps}
    \begin{align}
        \epsilon (s) = \frac{s}{\tau s} e^{\frac{s}{\tau s}}
        H(s) \label{eq:srm_eps} \\ s = (t - t_i^k - d_{ij}) \\ H(s) =
        \begin{cases} 
          1 & s >= 0 \\ 0 & s < 0
       \end{cases}
    \end{align}
    
    Where $H(s)$ can be recognized as the Heavy-side Step Function. After a
    spike, the neuron enters a refractory period.
        
    \begin{align}
        ca^{2+} = r + S_{mod} + PS_{mod} \label{eq:srm_astro_ca} r = 0.31
    \end{align}    
    
    %%%%%%%%%%%% SNNs on an FPGA %%%%%%%%%%%%
    \section{SNNs on an FPGA}
    \cite{cassidy_2017} have shown the LIF based spiking neural networks can be
    implemented on FPGA using adders to numerically integrate discrete
    spikes. STDP weight updating was also included in the implementation. Figure
    \ref{fig:fpga_lif} shows the hardware implementation of an LIF neuron. The
    normal continues LIF equations collapse to an adder with a negative bias.
    
    \afigw{figures/fpga_spike.png}{FPGA Implementation of LIF
      Neuron}{fig:fpga_lif}{2}
    
    Traditional ML tasks weren't performed on the hardware spiking network, but
    a few experiments were performed, which may provide insight. First network
    weights were optimized to provide a maximally informative Spacio-temporal
    receptive field (STRF) for English speech. This is a key step in voice
    recognition. Second the researchers exploited a well-known property of STDP
    to validate network and training behavior. Each input was driven by a
    Poisson spike train, each with the same $\lambda$. The expected result
    (after some time) is that the network will reach a steady state, with an
    equal number of exitatory and inhibitory weights. This result was observed,
    validating the FPGA implementation aligns with the expected result.
    
    
    \section{Theories on High-level Functions of Astrocytes in Biological
      Brains}
    
    It is important to speculate on some higher level functional roles of
    Astrocytes, even when the availability of concrete data isn't sufficent to
    analyze these speculations. In general astrocytes can modulate synapse
    excitability through a variety of mechanisms, and modulate synaptic
    plasticity through the release of glutamate and D-serine. There is evidence
    that astrocytes modulate LTP/LTD based on external inputs and internal
    state, this could provide a mechanism for reward-modulated STDP
    \cite{min_2012}. In addition It has been shown that smaller learning rates
    coupled with STDP lead to better memory, when presented with test and
    challenge inputs  \cite{van-rossum_2012}. Astrocytes modulate learning at a regional level,
    providing an opportunity for one portion of the network to learn, while
    another retains what it learned previously. In addition to regional gating
    (or more fine-grained modulation) of learning, Astrocytes, through local
    differences in response (within the same cell) can bias a network of spiking
    neurons to learn a particular function, or class of functions.

    In addition to plasticity, Astroctes can modulate the synaptic tone, or
    basal level of excitation. It has been suggested that this effect, coupled
    with inter-astrocyte communication can "activate" a region of a network,
    possibly choosing it as most important for the given input. In essence,
    Astrocytes would implement a context switch \cite{min_2012}. Another
    result of this regional excitation is synchronization. As astrocytes lower
    the threshold for post-synaptic firing, pre-synaptic potentials that are
    disparate in time, or strength can result in synchronized post-synaptic
    firing. This effect could be important in a variety of situations, but one
    of particular note is STDP. Since STDP is very sensitive to relative spike
    timing, having synchronized inputs (and not penalizing an input for being
    ever so slightly late) would result in better stability \todo{either prove
      this, for find a paper that proves it}.

    Given Astrocyte's level of connectedness with neurons, and the formation of
    Astrocyte networks, there is a distinct opportunity for information transfer
    across long distances without the need for supporting neural
    connections. It is thought that these pathways underlie some of the
    coordinated activity between neural circuits from different brain
    regions. Some researchers even claim Astrocytes for the basis of human
    consciousness.

    % objective_details.tex
%
% Author       : Jacob Kiggins
% Contact      : jmk1154@g.rit.edu
% Date Created : 01/31/2021
%

% NOTE: All filler text has "TODO" written. This must be removed in the final copy!

\chapter{Methods and Resources} \label{section:methods}
    The overarching goal of this thesis is to extend the bio-inspiration of
    spiking networks by introducing an Astrocyte-like element, and forming a
    tri-partied synapse. This Astrocyte element will drive synaptic plasticity;
    First implementing, then extending the widely used \gls{stdp} rule. Benefits of
    Astrocyte-like control of synaptic plasticity will be demonstrated in the
    single synapse, single neuron configuration (1S1N) as well as select
    multi-synapse configurations (i.e. 2S1N, 3S1N, etc...). Inputs will mostly
    consist of procedurally generated spiking inputs. This would include Poisson
    rate-coded spike trains, spiking inputs representing Boolean variables, and
    temporal inputs that follow specific patterns. In general, for each
    experiment there will be some comparison necessary between the Astrocyte
    driven plasticity and classical STDP. To evaluate each approach
    convergence/divergence of weights or speed of convergence may be
    used. Statistical analysis of various intermediate signals, or the weights
    may be useful as well, depending on the experiment. If it is difficult to
    compare an experiment to classical STDP, it should instead be possible to
    compare to another astrocyte driven experiment.

    %%%%%%%%%%%%% Datasets %%%%%%%%%%%%%
    \section{Datasets} \label{section:datasets}
    No formal dataset is used in this research. Instead, inputs to spiking
    neurons and spiking neural networks are simulated using rate-based and
    time-based spike-trains. These spike trains have specific patterns and are
    chosen based on the experiment at hand. Some example pattern include.
    \begin{enumerate}
    \item Random Poisson spike train with a single rate parameter
    \item Random Poisson spike train with a variable rate parameters
    \item A bust of spikes which may be inserted into another spike train
    \item Spikes representing Noise: Uniformly random, Gaussian, etc...
    \item Specific temporal patterns either hand-picked, spatial, or driven by a
      boolean function
    \end{enumerate}
        
    %%%%%%%%%%%%% Metrics %%%%%%%%%%%%%
    \section{Metrics} \label{section:metrics}
    Few of the common metrics used in machine learning will be applicable for
    this research. Instead Results will be presented using comparisons between
    plasticity variants, where the methods in this research show some
    improvement. In addition, results may display features not previously
    observed. In that case novelty will be the metric. Accuracy will likely
    still be applicable, especially with supervised variants of the Astrocyte
    Plasticity model.
    
    Two metrics will be used to monitor the network from a more
    application-independent perspective. This will give insight into how
    different architectures are behaving at a low level. The first is a
    per-neuron spike-trace. This will provide information like average spike
    rate, and show any areas of the network with increased activity. From this
    trace a higher-level insights can be made, namely counting the number of
    poly-synchronous groups, as outlined in \cite{sgnn_transistor}.
    
    %%%%%%%%%%%%% SW Components %%%%%%%%%%%%%
    \section{Software Components}
    Spiking Neural Networks are very much in their infancy as far as software
    and community support. This isn't due to lack of options, many frameworks
    are capable of simulating SNNs. However, these frameworks aren't as
    sophisticated, complete, error-free, or supported when compared to deep
    learning frameworks. In addition, many of the more bio-realistic 
    frameworks can't simulate networks at scale (such as \cite{nest}). Recently, a few
    python libraries have emerged that are usable and very close to complete
    solutions. The first is BindsNet, which has been referenced in a variety of
    papers. The original implementation included an architecture that achieved
    \acrshort{sota} results on \gls{mnist} classification. The second is PySNN, it was written by
    a PhD student, and builds on-top of PyTorch. Many common Spiking Neuron
    Models are included out of the box, such as \gls{lif}, and the more bio-inspired
    Izhikevich model. Another framework, somewhat newer is Norse. Unlike others
    listed here, Norse is in active development. Similar to PySNN, Norse
    provides a very clean interface and is built ontop of PyTorch. Specifically
    Norse leverages PyTorch's torchscript interface, which allows Norse to claim
    (and in some cases defend) that they have the most efficient library for
    simulation spiking neural networks \parencite{norse2021}.

    This research will leverage a modified version of Norse for all simulations.
    

	% Author       : Jacob Kiggins
% Contact      : jmk1154@g.rit.edu
% Date Created : 01/31/2021
%

\chapter{Development of an Astrocyte Model Sensitive to Spike Timing} \label{chapter:obj1}
The first objective of this thesis, is to develop an Astrocyte-synapse model
that is capable of responding to pre and post-synaptic spike timing, and
explore how internal and external configuration affects this response. These
internal signals will eventually be used to drive synaptic plasticity later in
this work, and that theme will guide the experiments in Chapter \ref{chapter:obj1}.

Though the many details of chemical signaling within Astrocytes are still unknown
\cite{manninen_2018}. There is a common theme to what is known both chemically,
and from bio-inspired models and this theme has potential to greatly improve upon
existing STDP-like plasticity rules. Astrocytes behave as Master integrators of
neural activity. They are capable of sensing activity across many thousands of
synapses simultaneously. Integrating that activity quickly at the local level,
and propagating signals to the entire cell body, integrating over a longer
time-scale. Applying this multi-level integration approach to synaptic
plasticity leads to many interesting questions and potential benefits beyond
existing local learning rules to explore.

\section{Spiking Neuron Model}
This work makes use of a Leaky Integrate and Fire (LIF) neuron model. LIF
neurons are fairly simple computationally, relying on exponential functions
and thresholds in the functional definition. While not the most
bio-realistic (see Figure \ref{fig:sn_model_compare}), LIF neurons exhibit
rich behavior with few parameters, and are easily extended. In addition, LIF
neurons are the most widely used in SNN literature \cite{ponulak_2011}. The
goal here, is to remain computationally simple, while still leveraging
biology, which is the same approach used in the Astrocyte Model definition.

Equations \ref{eq:lif:i}, and \ref{eq:lif:v} define the neuron current and
voltage dynamics.

\begin{align}
d_{i} = -i(t) \tau_{syn} * d_t + z_{pre}(t) * w \label{eq:lif:i} \\
d_{v} = -\tau_{mem} * (v(t) + i(t)) * d_t \label{eq:lif:v}
\end{align}

Here, $z_{pre}(t)$ represents the spiking activity on the synapse at time
$t$ and has a value of either $0$ or $1$. $w$ is the synaptic
weight. $\tau_{syn}$ and $\tau_{mem}$ are synapse and membrane decay
time-constants.

The neuron fires according to Equation \ref{eq:lif:fire}, where $H$ is the
Heaviside step function.

\begin{align}
v'(t) = v(t) + dv \label{eq:lif:v_update} \\
z_{post}(t) = H(v'(t) - thr_{mem}) \label{eq:lif:fire}
\end{align}

After firing the neuron membrane potential is reset to $v_{reset}$, which is
generally a negative voltage $\leq thr_{mem}$, see Equation \ref{eq:lif:reset}.

\begin{align}
v(t) = v'(t) * (1 - z_{post}(t)) + z_{post}(t) * v_{reset} \label{eq:lif:reset}
\end{align}

Figure \ref{fig:1n1s1a_fn_diagram} depicts the LIF neuron described by
equations above. Figures \ref{fig:lif:sample_1}, \ref{fig:lif:sample_2}, and
\ref{fig:lif:sample_3} shows a sample of the LIF responses to a random
Poisson input with various parameter values. Table \ref{table:lif_params} shows
the baseline parameter values, with any differences being outlined in the
figures.

\begin{table}[!htp]\centering
  \caption{LIF Neuron Baseline Parameters} \label{table:lif_params}
  \scriptsize
  \begin{tabular}{lrrrr}\toprule
    Reset Voltage &Membrane Threshold &Membrane Tau &Synaptic Tau \\\midrule
    -0.2 v &0.2 v &60 1/s &500 1/s \\
    \bottomrule
  \end{tabular}
\end{table}

\asvgf{figures/artifacts/obj1/lif_sample_mem-60.0_syn-500.0.svg}{LIF Neuron
Response To Poisson Input. $tau_{mem}=60$ and
$tau_{syn}=500$}{fig:lif:sample_1}{0.7}

\asvgf{figures/artifacts/obj1/lif_sample_mem-30.0_syn-500.0.svg}{LIF Neuron
Response To Poisson Input. $tau_{mem}=30$ and
$tau_{syn}=500$}{fig:lif:sample_2}{0.7}

\asvgf{figures/artifacts/obj1/lif_sample_mem-60.0_syn-250.0.svg}{LIF Neuron
Response To Poisson Input. $tau_{mem}=60$ and
$tau_{syn}=250$}{fig:lif:sample_3}{0.7}

Comparing Figures \ref{fig:lif:sample_1} and \ref{fig:lif:sample_2}, the
neuron seems more excitable with a higher $\tau_{mem}$. This is initially
counter-intuitive, since a higher value would result in increased decay,
however this $\tau_{mem}$ is also scales the effect of the input current on
membrane voltage by the input current (see Equation \ref{eq:lif:v}). In
addition, a larger decay factor results in decreased recovery time for the
neuron from reset. In contrast a lower value for $\tau_{syn?}$ causes the
neuron to fire more often with the same input, due to the input current
decaying more slowly.

\section{Classic STDP} \label{obj1:sec:classic_stdp}
    
The proposed Astrocyte model, in the single-synapse configuration aims to
generalize STDP, and offer some additional effects and features that are
distinct from STDP. In order to have a point of comparison, an STDP
implementation must be chosen to use in this capacity. The chosen model,
referenced by \cite{song_2000} is outlined by Equation
\ref{eq:song_classic_stdp}.

\begin{align}
  F(\delta t) =
  \begin{cases} 
    Ae^{- \frac{|t_{pre}-t_{post}|}{\tau}} & t_{pre} - t_{post} \leq 0, A > 0
    \\ Ae^{- \frac{|t_{pre}-t_{post}|}{\tau}} & t_{pre} - t_{post} > 0, B < 0
  \end{cases} \label{eq:song_classic_stdp}
\end{align}

This equation is evaluated for various $\delta t$ between $-50ms$ and
$50ms$. The resulting curve is outlined in Figure \ref{fig:stdp_dw_dt}. This
functional definition is not directly implemented since there is not timeline of
spikes to work with, instead, the exponential function is approximated by
integrating Equations \ref{eq:song_impl_pre}, \ref{eq:song_impl_post}, and
\ref{eq:song_impl_dw} over time. Table \ref{table:classic_stdp_params} shows the
parameters used during simulation.

\begin{table}[!htp]\centering
  \label{table:classic_stdp_params}
  \scriptsize
  \begin{tabular}{lrr}\toprule
    A &Tau \\\midrule
    1 &100 1/s \\
    \bottomrule
  \end{tabular}
\end{table}

\asvgf{figures/artifacts/obj1/closed_form_stdp.svg}{Simulation of Charicteristic
  STDP Equation}{fig:stdp_dw_dt}{0.4}

\begin{align}
  \delta trace_{pre} = -H(Z_{pre})*(\tau * trace_{pre}) +
  A*Z_{pre} \label{eq:song_impl_pre} \\ 
  \delta trace_{post} = -H(Z_{post})(\tau * trace_{post}) +
  A*Z_{post} \label{eq:song_impl_post} \\
  F = H(Z_{pre}) * trace_{post} + H(Z_{post}) *
  trace_{pre} \label{eq:song_impl_dw}
\end{align}

Depending on the desired behavior, the $H(Z_*)$ term may be dropped from the
update Equations, this is related to the spike associations outlined in Figure
\ref{fig:astro:spike_associate}.

Figure \ref{fig:stdp_dw_dt_impl} shows a sweep of spike pairs as before from
$-50ms$ to $50ms$. This figure matches Figure \ref{fig:stdp_dw_dt}, showing that
the implementation matches the theoretical equations.

\asvgf{figures/artifacts/obj1/1n1s1a_tp_m-plasticity_u-stdp_w-thr_dwdt_classic_stdp.svg}{Simulation
  of Classic STDP Implementation, Equations \ref{song_impl_pre} -
  \ref{song_impl_dw}}{fig:stdp_dw_dt_impl}{0.4}


%%%% Model Development Section %%%%
\section{Model Development}
Model development starts with a survey of current research, and
understanding of the underlying biology of Astrocyte mediated
plasticity. This portion of the process has been completed, see Chapter
\ref{section:background}. Astrocyte models in the literature range from
computationally simple models, which are generally paired with real-valued
neurons. To complex mathematical models, which aim to mimic the calcium
dynamics within an Astrocyte. The latter models are generally paired with
spiking neural networks. From the neuroscience and engineering literature,
it was possible to identify the common input and output pathways within the
tripartite synapse.

\begin{enumerate}
  \item Pre-synaptic Glu $\implies$ IP3 Pathway $\implies$ Ca pathway
  \item Post-synaptic depolarization K+ $\implies$ Ca pathway
  \item Ca $\implies$ D-serine release (more generally, plasticity modulation)
\end{enumerate}

\asvgf{figures/1n1s1a_diagram.svg}{Astrocyte and Neuron Model Diagram For An LIF
  Neuron with One Input}{fig:1n1s1a_fn_diagram}{0.4}

Figure \ref{fig:1n1s1a_fn_diagram} outlines the developed tripartite synapse
model, as well as the internal dynamics of the LIF neuron. Some aspects of
this model will be interchanged depending on the operating mode. These
blocks are the function blocks in darker blue. The portions of the model
which are consistent across operating modes are described first, followed by
a deeper dive into the remaining blocks. The two input pathways, which drive
the state values $ip3$ and $k+$ are described by Equations
\ref{eq:astro:spike-ip3} and \ref{eq:astro:spike-k+}.

\begin{align}
  d_{ip3} = d_t * (-ip3)\tau_{ip3} + \alpha_{ip3} z_{pre}(t) \label{eq:astro:spike-ip3} \\
  d_{k+} = d_t * (-k+)\tau_{k+} + \alpha_{k+} z_{post}(t) \label{eq:astro:spike-k+}
\end{align}

Where $\tau_{ip3/k+}$ represents a time constant for a given pathway, $z_{pre}$
is $1$ if there is a pre-synaptic spike at time $t$ ($0$ otherwise), and
$z_{post}$ is $1$ when the post-synaptic neuron fires ($0$ otherwise). For all
simulations (unless otherwise specified) $d_t = 0.001$ or $1ms$.

The \emph{Update(ip3, k+)} and \emph{Effect(u)} blocks define the behavior
of the Astrocyte at a local level, given input from the internal state
values, which are representative of external activity. There are two main
operating modes of the Astrocyte that have been explored for this
objective. Though in this chapter only the internal state of the Astrocyte will
be considered. \emph{Effect(u)} is then just considered a No-Op.

%% TODO: put this somewhere in Obj2
%% In general, \emph{Effect(u)} represents $d_w$ and takes the form outlined in
%% Equation \ref{astro:dw}. Whenever the Astrocyte state variable reaches the
%% pre-determined threshold the associated weight is scaled by some factor
%% $\alpha$, which may be different, depending on the direction of weight
%% change.

%% \begin{align}
%%   d_w = H(u - u_{thr}) * \alpha_{ltp} + H(-u - u_{thr}) * \alpha_{ltd} \label{astro:dw}
%% \end{align}

%% Here, $H$ is the heaviside step function, and $u_{thr}$ a constant threshold
%% applied to $u$.
    
\section{Exploring Input Pathways Responses}

In order to get some general guidelines for valid parameter ranges, it is
necessary to sweep parameters, and explore the model's response. The parameters
used during simulation are outlined in Table \ref{table:astro_in_path_params},
with any deviations mentioned in the graph.

\begin{table}[!htp] \centering
  \caption{Common Astrocyte Input Pathway Parameters} \label{table:astro_in_path_params}
  \scriptsize
  \begin{tabular}{lrrrrr}\toprule
    dt &Alpha IP3 &Tau IP3 &Alpha K+ &Tau K+ \\\midrule
    0.001 s &1 &100 1/s &1 &100 1/s \\
    \bottomrule
  \end{tabular}
\end{table}

\asvgf{figures/artifacts/obj1/sweep-alpha_tau-ip3-spike0.265.svg}{Model
Response to Sweeping Input pathway $\alpha$, and
$\tau_{ip3/k+}$}{fig:astro:sweep_alpha_ip3_tau_1}{0.6}

\asvgf{figures/artifacts/obj1/sweep-alpha_tau-ip3-spike0.055.svg}{Model Response to
Sweeping Input pathway $\alpha$, and
$\tau_{ip3/k+}$}{fig:astro:sweep_alpha_ip3_tau_2}{0.6}

The group of similar plots in Figures \ref{fig:astro:sweep_alpha_ip3_tau_1} and
\ref{fig:astro:sweep_alpha_ip3_tau_2} depict Astrocyte responses to a common
spike pattern, with different values for $\alpha_{ip3}=\alpha_{k+}$. The
curves in these plots are scale multiples of each-other, showing that
$\alpha_{ip3/k+}$ does not effect timing, but only scale of the Astrocyte
response. The figures also indicate that small values of
$\tau_{ip3}=\tau_{k+}$ can lead to a response that doesn't reach a
steady-state within a reasonable window, such as with $\tau_{ip3/k+} = 10$.
%% TODO: add more detail about steady-state


\section{Signed Proportional Astrocyte State Change}

The first variant of the \emph{Update(ip3, k+)} function considers the
magnitude of input pathway state variables $ip3$ and $k+$ as well as which
variable is larger (the order). This gives a magnitude and sign to each
update to the Astrocyte state variable $u$. The value of U eventually leading to LTP or
LTD, which is explored in Chapter \ref{chapter:obj2}. This approach is outlined
by Equations \ref{eq:astro:rate-u} and \ref{eq:astro:u-reset}, which define the
change in \ca

\begin{align}
  T_{diff} = ip3 - k^+ \\
  dca_{delta} = (H(T_{diff} - thr_{ltp}) - H(-T_{diff} - thr_{ltd})) * |ip3*k^+| \\
  dca_{decay} = -ca * \tau_{ca} * d_t \\
  dca = dca_{delta} + dca_{decay} \label{eq:astro:rate-u} \\
  ca = ca * (1 - H(ca - ca_{thr})) \label{eq:astro:u-reset}
\end{align}

Where H is the Heaviside step function, $T_{diff}$ is the difference between
$ip3$ and $K^+$ traces at some time $t$. $thr_{ltp}$ and $thr_{ltd}$ determine
the valid ranges of $T_{diff}$ for an update to \ca to occur. $dca_{delta}$
describes the change in \ca at a given time from external influence, while
$dca_{decay}$ accounts for the leaking of \ca. Note that the threshold values
determine the sign of $dca_{delta}$, and provide the ability to shift the \ca
response, as shown in Figure \ref{fig:astro_ltd_shift}. Increasing the distance
between the $ltp$ and $ltd$ thresholds also allow for varying degrees of
tolerance to transient activity that would otherwise lead to a response by the
Astrocyte. Simulations use the baseline parameter values outlined in Table
\ref{table:ordered_prop_params}, with any variations or parameter sweeps
outlined in graphs.

\begin{table}[!htp]\centering
  \caption{Signed Proportional Astrocyte Parameters} \label{table:ordered_prop_params}
  \scriptsize
  \begin{tabular}{lrrrrrr}\toprule
    dt &Alphas &Tau IP3 &Tau K+ <P Threshold <D Threshold \\\midrule
    0.001 s &1 &100 1/s &100 1/s &0 &0 \\
    \bottomrule
  \end{tabular}
\end{table}

\asvgf{figures/artifacts/obj1/sweep_u_tau_p0.1.svg}{Response of \ca to
Poisson Distributed spikes applied to Pre and Post-Synaptic
Pathways}{fig:astro:sweep_alpha_u_tau_1}{0.8}

\asvgf{figures/artifacts/obj1/sweep_u_tau_p0.7.svg}{Response of \ca to
Poisson Distributed spikes applied to Pre and Post-Synaptic
Pathways}{fig:astro:sweep_alpha_u_tau_2}{0.8}

Figures \ref{fig:astro:sweep_alpha_u_tau_1} and
\ref{fig:astro:sweep_alpha_u_tau_2} depict the response of \ca for
different values of $tau_{ca}$ and different spiking events. In general,
larger values of $\tau_{ca}$ result in smaller deviations in \ca overall, due
to the value decaying more quickly. With smaller $\tau_{ca}$ more of the
previous activity is integrated into the value of \ca and the response is
more broad.

Expanding on the simpler timeline plots, Figure
\ref{fig:astro:u_thr_heat_rate_vs_tau_u} shows the number of times \ca crosses a
threshold of $2.5$ for a given $\tau_{ca}$ and $r$ spiking rate for the Poisson
spike-train inputs. This graph can be used to tune either $thr_{ca}$ or $\tau_{ca}$
for a given application, and gives an impression of the excitability of the
model.

\asvgf{figures/artifacts/obj1/heatmap_spike-rate_tau_u.svg}{Rate of
Astrocyte Effect on Synaptic Weight Given Different input Poisson spike
rates, and different values of $\tau_{ca}$}{fig:astro:u_thr_heat_rate_vs_tau_u}{0.3}

\subsection{Classic-STDP and Variants} \label{sec:ordered_prop:stdp}

Using the Signed Proportional update rule the Astrocyte Plasticity Model is
capable of implementing classic STDP, along with several variations.  Figure
\ref{fig:astro:classic_stdp} shows the model's response to spike pairs with
a variety of inter-spike-interval times (X-axis). This is a typical STDP
curve.

\asvgf{figures/artifacts/obj1/1n1s1a_tp_m-plasticity_u-stdp_w-thr_dwdt_classic_stdp.svg}{Astrocyte
  and Neuron Model Response to Pulse Pairs:
  Baseline}{fig:astro:classic_stdp}{0.4}

The model has additional parameters beyond what a traditional STDP rule
would have, and this offers flexibility. By providing asymmetric
$\tau_{ip3}$ and $\tau_{k+}$ the model can be bias toward LTD/LTP. Figure
\ref{fig:astro:stdp_ltd_bias} shows this effect for LTD.

\asvgf{figures/artifacts/obj1/1n1s1a_tp_m-plasticity_u-stdp_w-thr_dwdt_ltd_bias.svg}{Astrocyte-Neuron
Model Response to Pulse Pairs: LTD Bias}{fig:astro:stdp_ltd_bias}{0.4}

In addition, threshold parameters $thr_{ltp}$ and $thr_{ltd}$ can be used to
shift activity from the LTP to LTD region, or vise versa. Figure
\ref{fig:astro:stdp_ltd_shift} shows this shifting behavior.

\asvgf{figures/artifacts/obj1/1n1s1a_tp_m-plasticity_u-stdp_w-thr_dwdt_ltd_dt_shift.svg}{Astrocyte-Neuron
Model Response to Pulse Pairs: LTD Shift}{fig:astro:stdp_ltd_shift}{0.4}

%% Additional figures can be found in Appendix \ref{appendix:astro_figures}.

One of the main features of this model is it's ability to generalize STDP,
and capture a variety of variants that have been proposed an explored in the
literature. Table \ref{table:astro_varient_params} maps specific values of
model parameters to the corresponding STDP behavior. If that
variant of STDP has a name from other research it is included.

\begin{table}[!htp]\centering
\caption{Model Parameters Associated with STDP Variants} \label{table:astro_varient_params}
\scriptsize
\begin{tabular}{lrrrrrrrrrrr}\toprule
STDP Variant &Name from Literature &$alpha_{ip3}$ &$tau_{ip3}$ &$alpha_{k+}$ &$tau_{k+}$ &$tau_u$ &$ltp_{thr}$ &$ltd_{thr}$ &$reset_{ip3}$ &$reset_{k+}$ \\\midrule
Triplet STDP &Triplet STDP &1 &100 &1 &100 &10000 &0 &0 &Yes &Yes \\
Classic & &1 &100 &1 &100 &10000 &0 &0 &No &No \\
LTD Bias & &1 &100 &1 &30 &10000 &0 &0 &No &No \\
LTP Bias & &1 &30 &1 &100 &10000 &0 &0 &No &No \\
LTD Shift &Mirror STDP &1 &100 &1 &100 &10000 &0.5 &0.5 &No &No \\
LTP Shift & &1 &100 &1 &100 &10000 &-0.5 &-0.5 &No &No \\
Anti-STDP &Anti-STDP, Rev-STDP &-1 &100 &-1 &100 &10000 &0 &0 &No &No \\
\bottomrule
\end{tabular}
\end{table}

\section{Rate-based Response} \label{sec:rate_response}

When input spiking activity is sparse in time, the model behaves as outlined
in Section \ref{sec:ordered_prop:stdp} above. However, when more dense
inputs are provided, the relationship between $ip3$ and $k+$ shifts from
temporal, to activity-based. In the sparse case, $ip3 > k+$ implies that an
incoming spike happened more recently then a post-synaptic spike. When
incoming spikes are dense in time, $ip3 > k+$ would imply more activity on
the input side (on average) than the output. What follows, is a Rate-based
internal response, where \ca responds to a difference in average activity,
subject to time constants. Equation \ref{eq:astro:rate-u} above describes \ca
behavior. Equation \label{eq:astro:u-reset} outlines the reset of behavior \ca,
the value reset to $0$ when $|ca| > thr_{ca}$.

In Figure \ref{fig:astro:rp_many-w_tl}, the middle plot shows the internal
response of an Astrocyte to a random Poisson across different weight values. In
this case, there is significant response from the Astrocyte for weight values of
$2.0$, with less of a response for $4.0$. This is due to a lower firing rate out
of the LIF neuron compared to the input. Intuitively there should be a weight
value where the difference between pre and post-synaptic firing rate is
smallest. For the upper graph $w=10.0$ is close to that point, with only two
regions in the timeline with any activity.

\asvgf{figures/artifacts/obj1/astro_rp_many-w_tl.svg}{Astrocyte Response to Random
  Poission Spike Train for Various Synaptic Weights}{fig:astro:rp_many-w_tl}{0.4}

Looking at the upper plot in Figure \ref{fig:astro:rp_many-w_tl}, the effect of
$thr_{ltp}=1.75$ and $thr_{ltd}=-1.75$ can be seen. Those parameter values form
a hysteresis band, where the Astrocyte does not respond with increased \ca for
small deviations between \ipt and \kp. This leads to an increased range of
weight values resulting in stable behavior. Sweeping synaptic weight in Figure
\ref{fig:astro:rp_many-w_sweep} shows that weight values beyond $w \approx 3.0$
result in no \ca activity.

\asvgf{figures/artifacts/obj1/astro_rp_many-w_sweep.svg}{Astrocyte Response to Random
  Poission Spike Train for Various Synaptic Weights}{fig:astro:rp_w_sweep}{0.4}

%% One approach to mitigate the averaging effect would be lowering
%% $\tau_{ip3}$ and $\tau_{k+}$, though this would reduce sensitivity to pulse
%% pairs occurring on larger time-scales. A more elegant approach would be to
%% modify \emph{Update(ip3, k+)}, combining the input traces in a way that
%% preserves timing. $u$ can then be used to integrate pulse-pair events over
%% time and drive weight change. Equation \ref{eq:astro:temp-u} describes the
%% update for $u$.

\section{Integrated Spike Timing Response} \label{section:istp}
Next, this work considers an variant of \emph{Update(ip3, k+)} that is sensitive
to pre and post-synaptic spike timing, but not the rate of activity at each
terminal. The previous Signed Proportional model outlined in Section
\ref{sec:sign-prop} could exhibit sensitivity to spike timing. This was
demonstrated in Figure \ref{fig:astro:classic_stdp}, where the characteristic
STDP curve was reproduced by the Astrocyte model. As shown in Section
\ref{sec:rate_response} the sensitivity to spike timing only holds for sparse
inputs. With dense inputs, \ipt and \kp values represent average activity, and not
timing.

This new \emph{Update(ip3, k+)} function is define by Equation
\ref{eq:astro:temp-u}. $reset_{ip3}$ and $reset_{k+}$ parameters, determine if
additional spikes are added into the $ip3$ and $k+$ traces, or override
them. These parameters, along with the weight update behavior determine which
groupings of spike result in changes to $u$. Figure
\ref{fig:astro:spike_associate} outlines different possible spike association
behaviors.

%% TODO: Define reset behavior

\asvgf{figures/SpikeAssociation.svg}{A Variety of Spike Associations
Supported by the Neuron-Astrocyte model}{fig:astro:spike_associate}{0.5}

\begin{align}
T_{delta} = sign_{ltd} * z_{pre} * k^+ + sign_{ltp} * z_{post} * ip3 \\
dca = -ca * \tau_{ca} + d_t + T_{delta} \label{eq:astro:temp-u}
\end{align}

In this case, $z_{post}$ and $z_{pre}$ are pre and post-synaptic spikes that
the Astrocyte is sensing. The sign $sign_{ip3/k+}$ values may be functions
of $k+$ and $ip3$ as well as some constants. For some experiments they will
simply be $sign_{ltd}=-1.0$ and $sign_{ltp}=1.0$. Equations
\ref{eq:astro:temp-sign-1} and \ref{eq:astro:temp-sign-2} show definitions
of $sign_{ltd}$ and $sign_{ltp}$ that would allow a region around $dt=0$
where weight updates were 0.

\begin{align}
sign_{ltd} = -H(k+ + thr_{ltd}) \label{eq:astro:temp-sign-1}\\
sign_{ltp} = H(ip3 + thr_{ltd}) \label{eq:astro:temp-sign-2}
\end{align}

Unlike with the rate-based variation, updates to \ca happen only when spikes
occur on either pre or post-synaptic terminals, instead of continuously. As
in the rate-based model, these updates to \ca are integrated, and reset when \ca
exceeds a threshold $thr_{ca}$. The baseline parameter values for simulation are
outlined in Table \ref{table:istp_params}.

\begin{table}[!htp] \centering
  \caption{Integrated Spike Timing Astrocyte Parameters} \label{table:istp_params}
  \scriptsize
  \begin{tabular}{lrrrrr}\toprule
    dt &Alphas &Tau IP3 &Tau K+ &Ca Threshold \\\midrule
    0.001 s &1 &100 1/s &100 1/s &2.5 \\
    \bottomrule
  \end{tabular}
\end{table}

Figure \ref{fig:astro:tp_many_w_tl} shows the behavior of temporal
integration model when a series of spike pulses are presented to the
network. Each pulse is a fixed number of spikes, and the resulting \ca activity
is depicted for a given synaptic weight. This simulation depicts the averaging
effect of the model spike timing model, With an impulse of 4 spikes, the
neuron is driven enough to output a single spike, this happens around
300ms. With 4 pre-synaptic spikes and 1 post-synaptic spike, the Astrocyte's
internal state $u$ is increased, exceeding the threshold of $2.5$. Subsequent
spike impulses have $>4$ spikes, which leads to a drop in \ca, due to
anti-causal pairing (transiently) of pre-synaptic spikes. Regardless of these
transient drops, \ca consistently exceeds $thr_{ca}$ throughout the simulation.

Figure \ref{fig:astro:tp_many_w_sweep} graphs the number of times $thr_{ca}$ is
exceeded during a fixed length simulation, for a few different values of
$thr_{ca}$. For $Ca^{2+}=2.5$, regions of the graph around $0.8$ and $1.2$ show
weight values that result in a stable configuration, where $thr_{ca}$ is not
exceeded for the length of simulation. Figure \ref{fig:astro:tp_many_w_tl} shows
a timeline of Astrocyte activity for three different weight values, and
$thr_{ca}=2.5$.

%% TODO: generate weight sweep graphs, with fixed spiking impulse input

\asvgf{figures/artifacts/obj1/astro_tp_many-w_sweep.svg}{Aggregate Astrocyte
  Response to Input from a single synapse, with various synaptic weight and
  $thr_{ca}$}{fig:astro:tp_many_w_sweep}{0.5}

\asvgf{figures/artifacts/obj1/astro_tp_many-w_tl.svg}{Timeline of Astrocyte
  Activity With Input from a single synapse, given
  $thr_{ca}=2.5$}{fig:astro:tp_many_w_tl}{0.5}

\section{Summary}
This chapter introduces a computationally efficient and intuitive Astrocyte
model. This model was derived from the many works outlined in Chapter
\ref{chapter:backround}, leaning heavily towards Engineering applications,
while taking inspiration from the Neuroscience and many highly bio-realistic
models. A relatively small number of parameters are used to define the
Astrocyte model's behavior, and the effects of varying this behavior is
explored in two different configurations of the model, with an emphasis on
signals internal to the Astrocyte. Figure \ref{fig:1n1s1a_fn_diagram} shows
an ``Update FN'' block, which represents how the input traces \ipt and \kp
effect Astrocyte \ca. With \ca concentration driving the mechanisms
identified for the Astrocyte to influence things external to
itself. Considering a variant of the model where \ca responds to a
difference in the magnitude between \ipt and \kp, it was found that \ca
response varied widely given a random Poisson input for some weight values.
For other weight values, where a quick response from the neuron was
observed, \ca was relatively stable. Switching the input to be a pair
of pre and post-synaptic spikes with variable inter-spike-interval, a graph
very similar to the characteristic STDP curve was observed (Figure
\ref{fig:astro:classic_stdp}).

Since the first variation of the Astrocyte model exhibited sensitivity to
the general firing rates at pre and post-synaptic terminals, it was logical
to explore a variation sensitive to spike timing. This ``Integrated Spike-Timing''
model, similarly, exhibited an STDP-like \ca response given a single spiking
pair with variable timing. With dense inputs this model maintained sensitivity
to recent spike timing, and exhibited bi-modal stability with respect to
synaptic weights.

    \chapter{Synaptic Plasticity Using the Astrocyte Model in A Single Synapse
  Configuration} \label{chapter:obj2}

Astrocytes have suspected involvement in synaptic plasticity in both the
long \cite{min_2012} and short term \cite{pitta_2012}. A variety of
bio-inspired Astrocyte models have touched on short term plasticity, and
suggested modulation of classical Hebbian learning \cite{pitta_2016}. In
addition to the experimental evidence, Astrocytes are well placed to control
synaptic plasticity at many levels. Drilling down to the level of the
synapse, astrocytes monitor synaptic activity chemically, and respond
quickly with a variety of gliotransmitters, some of which are known to be
critical for \Gls{ltp}/\Gls{ltd} \cite{min_2012}. This fast local activity, which takes
the form of \ca transients in Astrocyte end-foot processes is integrated
into a cell-level response, and can travel to distant synapses.

The goal in Chapter \ref{chapter:obj2} is to explore Astrocyte mediated synaptic
plasticity in a single synapse configuration. This will close the loop with the
experiments in Chapter \ref{chapter:obj1}, by incorporating parameterized
synaptic weight changes, directed by the Astrocyte. Focus will be shifted from the
Astrocyte's internal response, to the learning process, progression of weight
changes. Novel features of the full astrocyte plasticity model, will be
identified and the effect of parameters on these features explored.

%% with a single LIF neuron. Existing
%% literature outlines many ways that Astrocytes are shown to, are suspected to, or
%% have been modeled to affect their external environment. Table
%% \ref{able:obj2_astro_features} outlines these features of Astrocyte and
%% Neuron-Astrocyte interactions that were common threads in background and related 
%% work (see Chapter \ref{chapter:background}). Within this chapter a number of
%% these features have equivalent configurations in the developed Astrocyte model,
%% and the effects of that configuration are explored further.

%% %If the table is too wide, replace \begin{table}[!htp]...\end{table} with
%% \begin{adjustwidth}{-2.5 cm}{-2.5 cm}\centering\begin{threeparttable}[!htb]
%% %% \begin{table}[!htp]\centering
%% \caption{Generated by Spread-LaTeX}\label{table:obj2_astro_features}
%% \scriptsize
%% \begin{tabular}{lrrrrrr}\toprule
%% Astrocyte Feature &Primary Author &References &Astrocyte Model &Functional Model in Primary Work &Adaption in This Work \\\midrule
%% Ca Mediated increase in Synaptic Current &Wade &Wade2011, Wade2012 &Lin and Rizel or Variation &\ipt = L(incoming spikes), d\_ca = L(\ipt) + L(Ca) - NL(Ca) &i = z*w + i\_astro \\
%% Ca Mediated increase in Synaptic Weight &De Pitta &DePitta2016 &Lin and Rizel or Variation &Ca -> \Gls{glu} -> increased PR &Threshold on Ca + constant multiplier to drive weight increase. Weight incrase is proportional to Ca, subject to a maximum change and scale factor \\
%% Modulation of Ca Based STDP Curve &De Pitta &DePitta2016 &Lin and Rizel or Variation &Shifting and modulation of Ca based two-threshold STDP &Differential Tau paramters on \ipt and \kp to achieve a similar result \\
%% Gating of \Gls{ltp}/\Gls{ltd} With D-Serine release &No primary Author & & &D-serine release opens the gate for synaptic plasticity through other mechanisms &The astrocyte drives plasticity, based on internal state including pesudo ip3, k+ and Ca \\
%% Working Memory &Gordleeva &Gordleeva2021 &Lin and Rizel or Variation &Astrocyte Calcium persists, and changes network sensitivity to prompt inputs & \\
%% \bottomrule
%% \end{tabular}
%% %% \end{table}
%% \end{threeparttable}\end{adjustwidth}

% TODO: Come up with a more generic name for Lin and Rizel model, and talk about
% how other models follow that basic approach, but that this work need not make
% a distinction between them.

\section{\ca Threshold Triggered Weight Update}
One key Astrocyte property highlighted in Chapter \ref{chapter:background} is
the concept of \Gls{cicr}. Essentially, Astrocytes integrate synaptic activity into a
\ca response, once that \ca concentration exceeds a threshold, there is a
cascade of events resulting in both \ca propagation to the cell body, and the
release of glio-transmitters into the synaptic cleft. This concept of \Gls{cicr}, fits
well into the current model, and will be explored in this chapter as the
mechanism that drives weight change.

In the context of this work's Astrocyte model, as \ca values increase and exceed
a threshold $thr_{ca}$, a weight change will occur at the synapse, and the \ca
value degraded to zero, representing the re-uptake of \ca observed with \Gls{cicr} in
biology. Tuning this threshold value, should lead to some improvements for both
weight convergence given a set of inputs, and tolerance to overall noise in the
system.

\begin{align}
  d_w = w * (\alpha_{ltp} * H(ca - thr_{ca}) + \alpha_{ltd} * H(-ca - thr_{ca}))  \label{eq:dw-thr}
\end{align}

Equation \ref{eq:dw-thr} describes the baseline concept for weight
change. $\alpha_{ltp/ltd}$ parameters server as learning rates, and can be tuned
based on the type of inputs or Astrocyte performance.


\section{Synchronization of Firing Rate}
While STDP, and spike-timing based learning is the main focus of this work, the
proportional difference astrocyte configuration outlined in Section
\ref{sec:rate_response} shows a novel feature that has ties to Neuroscience
research. One of the possible functional roles of Astrocytes in nature is
synchronization of synaptic firing (see Chapter \ref{chapter:background}). This
is thought to support computation in the brain, and drive detectable patterns we
call brain waves. It is interesting then, that a variation on the Astrocyte
model in this work, is able to exhibit synchronization on a smaller scale.

Again, closing the loop from Chapter \ref{chapter:obj1}, the Astrocyte directs
synaptic plasticity as outlined by Equation \ref{eq:dw-thr}. Simulation with a
random Poisson spike train input, and $thr_{ca}=0$, shows that weight values
increase from their starting point, and approach a value where post-synaptic
spike track pre-synaptic.

%% \begin{align}
%%   d_w = ca / thr_{ca}  \label{eq:dw-thr-classic}
%% \end{align}

\asvgf{figures/artifacts/obj2/snn_1n1s1a_rp_thr_0.svg}{
  Astrocyte response to random Poisson input with rate-based plasticity
  rule. $thr_{ltp}=-thr_{ltd}=0.8$}{fig:astro:rp}{0.5}

Figure \ref{fig:astro:rp_thr} show the effect of tuning $thr_{ca}$, which
requires a significant \ca response before synaptic weights can change. The
result is a more optimal convergence, which is stable, with inputs and outputs
exhibiting a high degree of synchronization.

\asvgf{figures/artifacts/obj2/snn_1n1s1a_rp_0.svg}{
  Astrocyte response to random Poisson input with rate-based plasticity
  rule. $thr_{ltp}=-thr_{ltd}=1.5$}{fig:astro:rp_thr}{0.5}

\section{Spike-Timing Integration Plasticity}

One of the key features of the Astrocyte model developed in chapter
\ref{chapter:obj1}, and an overarching goal of this thesis is the support for
detection and integration of coincident spikes, and subsequent synaptic
plasticity in an STDP-like fashion. In Chapter \ref{chapter:obj1} it is shown
that the Astrocyte model is capable of
implementing the classic STDP curve, common variations on STDP, and modulating
of the standard STDP curve. This STDP-like response however, was not a
plasticity response, but an internal \ca response. Moving on to this chapter,
that \ca response is being translated into a weight update. Given the definition
of $dw$ in Equation \ref{eq:dw-classic}, weight updates will track the \ca
response, and an end-to-end implementation of STDP results. Equation
\ref{eq:dw-classic} is then used as a baseline during simulation, to compare
with features of the Spike-timing integration approach.

\begin{align}
  d_w = ca / thr_{ca}  \label{eq:dw-classic}
\end{align}

This ``pulse-pair integration'' approach differs from classic STDP in the
definition of $dw$, with the latter using Equation \ref{eq:obj2:dw}, and in the
general flexibility of the Astrocyte model, which isn't typical of STDP in the
literature. Functionally, this makes spike-timing integrated plasticity different
in a few key ways. Firstly, weight updates are delayed, and
only occur when sufficiently many coincident spike occur to drive \ca
$>thr_{ca}$ in a short time window (dependent on time constant $\tau_{ca}$). Any
weight update will depend on a series of 
causal or anti-causal spiking events. The need for multiple events leads to 
increased confidence in each weight change, and can filter out some random spike
associations due to noise.  Furthermore, the approach has support from existing
Neuroscience and Engineering literature, and is similar to \Gls{cicr}, where calcium
levels crossing some threshold lead to a release of gliotransmitters locally. In
the case of this work, that Gliotransmitter release is modeled as a change in
synaptic weight.

Equations \ref{eq:obj2:dw} and \ref{eq:obj2:ca_reset} fully
describe the weight update step. To evaluate the Astrocyte
model in this configuration a series of spike impulses are used. These are
groups of regularly firing inputs in a $1010101$ pattern, with sufficient space
between these groups for Astrocyte and Neuron states to settle. This input was
chosen as a benchmark, and to represent a burst of input, which is common in
biological systems.

\begin{align}
  d_w = H(ca - ca_{thr}) * \alpha_{ltp} + H(-ca - ca_{thr}) *
  \alpha_{ltd} \label{eq:obj2:dw} \\
  ca = ca * (1 - H(ca - ca_{thr}) +  H(-ca - ca_{thr})) \label{eq:obj2:ca_reset}
\end{align}

As discussed before A slight variation allows the Astrocyte model to implement
classic STDP, outlined in Equations \ref{eq:dw-classic}. In this case \ca
effects weight values and is reset every time-step. Since the \ca response was
already shown to mimic Classic STDP in \ref{chapter:obj1} Equation
\ref{eq:dw-classic} will as well.

\subsection{Impulse Response}
Figure \ref{fig:ppi:impulse_istp} depicts the Astrocyte plasticity model with
$Ca_{thr}=2.5$, $\alpha_{ltp}=1.05$ and $\alpha_{ltd}=0.95$. In contrast, Figure
\ref{fig:ppi:impulse_stp} shows the behavior of a classic STDP (implemented by
the Astrocyte) with the same inputs and time-constants. The inputs themselves
consist of successive spikes, with 1ms between each. These spikes form groups
with enough time between for neuron and Astrocyte states to decay. Each group
has an additional spike compared to the last.

Comparing these graphs, it is clear that integrating
behavior of the Astrocyte model effects the learning process
significantly. Resulting in more stable behavior, and increasing synaptic weight
due to causal, correlated inputs. The key effect here can be seen considering
what happens when there are additional input spikes after a post-synaptic
spike. In this case classic STDP would decrease the weight, while the Astrocyte
integrates this event into \ca. Those input spikes are eventually associated
with an additional down-stream spike, forming a causal relationship and \ca is
increased. Overall this leads to \Gls{ltp}, with the potential \Gls{ltd} events being
smoothed out.

\asvgf{figures/artifacts/obj2/snn_1n1s1a_tp_pulse_astro_0.svg}{
  Astrocyte response to successive spikes, displaying temporal pulse-pair
  integration}{fig:ppi:impulse_istp}{0.5}

\asvgf{figures/artifacts/obj2/snn_1n1s1a_tp_pulse_classic_0.svg}{
  Classic STDP response to successive spikes}{fig:ppi:impulse_stp}{0.4}

It is important to understand the situations under which the Astrocyte model
will converge, and compare that to classic STDP. Figures
\ref{fig:ppi:impulse_fixed_istp} and \ref{fig:ppi:impulse_fixed_stp} depict
responses to a similar series of spike impulses, but in this case each group of
spikes is the same length. There is a significant difference between the
behavior in this case, with the Astrocyte model quickly converging, while
classic STDP appears to overshoot a stable configuration, oscillating back and
forth. Figure \ref{fig:ppi:impulse_fixed_stp_diverge} shows the same simulation
on a longer timescale, showing these oscillations continue and remain periodic.

% Astrocyte response to fixed length impulse inputs
\asvgf{figures/artifacts/obj2/snn_1n1s1a_tp_pulse_const_astro_0.svg}{Astrocyte
  response to a series of fixed length spike impulse inputs showing
  convergence}{fig:ppi:impulse_fixed_istp}{0.4}

% Classic STDP with fixed length impulse inputs
\asvgf{figures/artifacts/obj2/snn_1n1s1a_tp_pulse_const_astro_0.svg}{Classic
  STDP response to a series of fixed length spike impulse
  inputs}{fig:ppi:impulse_fixed_stp}{0.4}

% Classic STDP again, but with a longer simulation and only showing weights
\asvgf{figures/artifacts/obj2/snn_1n1s1a_tp_pulse_const_diverge_classic_0.svg}{Classic
  STDP response to fixed length spikes showing divergent
  behavior}{fig:ppi:impulse_fixed_stp_diverge}{0.4}

% Comparison with noisy inputs
The smoothing effect observed with the Astrocyte plasticity model should work
just as well at smoothing out noise, as it does transient causal or anti-causal
spike pairings. Testing this, Figures \ref{fig:ppi:impulse_noise_stp} and
\ref{fig:ppi:impulse_noise_istp} show the results of similar simulations, with
the addition of uniformly random noise in both cases. The probability specified,
is the probability that a noisy spike is received on at the pre-synaptic
terminal during any single time-step. Observing the weight updates associated
with the Astrocyte model, there isn't a significant impact from the noise on the
progression of weight modulation. In the case of classic STDP, weight values
fluctuate even further, adding to the already unstable behavior.

% STDP response to impulse + uniform noise
\asvgf{figures/artifacts/obj2/snn_1n1s1a_tp_pulse_noise_classic_0.svg}{Classic
  STD response to impulse inputs with uniformly random
  noise}{fig:ppi:impulse_noise_stp}{0.4}

% Astro response to impulse + uniform noise
\asvgf{figures/artifacts/obj2/snn_1n1s1a_tp_pulse_noise_astro_0.svg}{Astrocyte
  response to impulse inputs with uniformly random
  noise}{fig:ppi:impulse_noise_istp}{0.4}

% Exploration of Weight update behavior with different parameters
The established baseline thus far, is that given this repeating (possibly
ramping up) spike impulse as an input, results in stable convergent weight
change from the Astrocyte model. In contrast, classic STDP shows instability and
lack of convergence. This is only a single configuration however, and there are many
parameters that could be modified. An investigation into the following
parameters can help determine if the convergent/noisy behavior is stable,
and what other behaviors of the model may arise.

\begin{itemize}
\item Initial weight
\item Spike association
\item Time constants
\item $dw$ factors, $\alpha_{ltp}$ and $\alpha_{ltd}$
\item Neuron threshold
\end{itemize}

Figure \ref{fig:sweep:mu} shows a sweep of initial weight values, and
demonstrates the effect on weight convergence. In both cases, if weight values
are sufficiently low, the general lack of post-synaptic activity leads to little
or no weight update. With higher initial weight values the Astrocyte model was
able to converge as with previous simulations, while the STDP diverged towards a
weight value of zero. The Astrocyte model converged to a weight value as a
function of the initial weight, specifically the next stable configuration from
Figure \ref{fig:astro:tp_many_w_sweep} that can be reached by \Gls{ltp}. This is a key
property of the weight update mechanics, in that the system seeks a stable
configuration.

\asvgf{figures/artifacts/obj2/snn_1n1s1a_tp_mu_sweep_0.svg}{Classic STDP and
  Astrocyte Response to Initial Weight}{fig:sweep:mu}{0.4}

%% \asvgf{figures/artifacts/obj2/snn_1n1s1a_tp_sweep_association_0.svg}{Classic
%%   STDP and Astrocyte Response to Different Spike Associations}{fig:sweep:assoc}{0.4}

Figure \ref{fig:sweep:astro_dw_factor} shows the effect of varying the weight
update factor $\alpha_{ltp}=\alpha_{ltd}$. In general, a very low factor results
in a lack of movement in the synaptic weight, whereas a high weight factor leads
to some stable regions being skipped over, but in each case some stable
configuration is found.

\asvgf{figures/artifacts/obj2/snn_1n1s1a_tp_sweep_dw_factor_astro_0.svg}{Astrocyte
Response to Different Values for Weight Update
Factor}{fig:sweep:astro_dw_factor}{0.4}

\asvgf{figures/artifacts/obj2/snn_1n1s1a_tp_sweep_dw_factor_classic_0.svg}{Classic
STDP Response to Different Values for Weight Update Factor}{fig:sweep:stdp_dw_factor}{0.4}

Figure \ref{fig:sweep:stdp_dw_factor} shows the response of Classic STDP to
different $\alpha_{ltp}=\alpha_{ltd}$. In general the response mimics the form
of the astrocyte weight progression, but exhibits significant instability, and
does not converge to a stable configuration. Figure \ref{fig:sweep:v_th} depicts
synaptic weight progression given different values for LIF neuron membrane
voltage threshold $v_{th}$. This experiment, meant to rule out a potential
avenue for increasing stability of Classic STDP further supports the unique
stability offered by an Astrocyte plasticity approach.

\asvgf{figures/artifacts/obj2/snn_1n1s1a_tp_sweep_lif_v_th_0.svg}{Classic STDP
  and Astrocyte Response to Different LIF Neuron Voltage Threshold Values}{fig:sweep:v_th}{0.4}

Figures \ref{fig:sweep:astro_tau} and \ref{fig:sweep:stdp_tau} show various
weight progressions across three different subplots, with the Figures
depicting Astrocyte and Classic STDP behavior respectively. In general,
modifying the tau parameters, whether $\tau_{ip3}$, $\tau_{k+}$, or both leads
to the synaptic weight settling on the stable configuration around $w=1.1$
instead of $w=0.81$. In the case of Classic STDP, some modified $\tau_*$
configurations lead to divergence to the maximum weight value ($20$), or
exhibited similar instability to what has been observed thus far.

\asvgf{figures/artifacts/obj2/snn_1n1s1a_tp_sweep_tau_astro_0.svg}{Astrocyte
  Response to Different Values for \kp and \ipt Pathway $\tau$}{fig:sweep:astro_tau}{0.4}

\asvgf{figures/artifacts/obj2/snn_1n1s1a_tp_sweep_tau_classic_0.svg}{Classic
  STDP Response to Different Values for Pre and Post-synaptic Trace
  $\tau$}{fig:sweep:stdp_tau}{0.4}

% TODO: graph of noise response vs. noise percent and possibly type

\section{Summary}

This Chapter builds upon the Astrocyte-LIF models from the previous chapter,
adding a plasticity step, which translates \ca levels to a change
in synaptic weight. In order to implement STDP, \ca concentration can be directly
translated into a weight change. This is the baseline used for comparison with
other variations on weight update dynamics.

Temporal integration is explored next, by defining a simple threshold on
\ca to drive plasticity. This means (in general) that multiple plasticity events
must occur on a given synapse before any change in weight occurs. A variety of
parameter values are explored, in both the ``Spike Timing'' and ``Rate-based''
Astrocyte models. Compared to classic STDP, the threshold approach exhibits
convergence where classic STDP diverges, or exhibits oscillation. A threshold
update rule is able to consistently drive synaptic weights into one of the
Astrocyte model's stable configurations. In addition, and learning and
convergence are stable in the presence of Gaussian noise.

The observed improvements over classic STDP were demonstrated across a variety
of parameter values, including varying $tau_{ip3}$, $tau_{k+}$, LIF $V_{th}$,
and the weight update multiplier. In each case where learning did occur, the
improvements over classic STDP were observed.


    \chapter{Extension Astrocyte Model To Multi-Synapse
  Configurations} \label{chapter:obj3}

In biology, astrocytes are observed influencing many synapses, in general,
hundreds of synapse across a single digit number of neurons. This spatial
integration is a fundamental property of astrocytes, and the response to
activity from these connected synapses is observed as a cell-level \ca
response. Taking this concept and applying it to the astrocyte model
in this work, a coordinated approach to synaptic plasticity is developed. This
approach is employed in a learning task where a multi-synapse view is critical.

%% \section{Multi-Synapse Cell-Level Response}
%% What is the ultimage goal here... to support some learning on some specific set
%% of inputs, that would not otherwise be achievable w/ STDP. What is the task?
%% what is an interesting task that involves spike timing?

%% What are some things I've got in my notes about this?

%% Gating of plasticity
%%   - Only allow plasticity on synchronous inputs
%%   - The opposite of ^
  
%% Orchestration between Synapse
%%   - Don't allow multiple synapse to learn from the same post-synaptic spike (WTA
%%   Pairing)
%%   - Global learning ``pool'' Each thr event on any synapse depletes the pool,
%%   which then refills
%%   - Relative strength of plasticity

\section{Synaptic Coupling}
This chapter introduces the concept of synaptic coupling. This is an
intermediate between the strictly local response that has been explored thus
far, and the slower (seconds timescale) cell-level responses directly observed
in biology. With synaptic coupling, there is a regional response to a hand-full
of synapses.  This response is fast (unlike the cell-level \ca responses),
operating at the speed of pre and post-synaptic spikes. Some in the
Computational and Neuroscience communities believe similar intermediate
astrocyte responses occur in nature, but that researchers don't have the
capabilities to observe them yet. In any case, synaptic coupling fits the common
theme of multi-level integration, and is shown to be critical in an example
learning task. Figure \ref{fig:astro:syn_coupling} shows the progression of the
astrocyte model from exhibiting a strictly local responses, to support
coupling. There are three major changes to the model from previous iterations.

\begin{enumerate}
\item The \ca response generated local to a synapse propagates up to the
  regional level. This is consistent with the observation of \gls{cicr} explored in
  Chapter \ref{chapter:background}.
\item At the regional level, there is some function implemented by the
  astrocyte, which directs plasticity across any connected synapses.
\item At the regional level, two additional internal signals \dser and
  \serca are introduced. These can be though of as chemical signals which can
  propagate from the regional level, down to a given synapse. They are
  responsible for triggering a weight change locally, or causing
  degredataion/re-uptake of \ipt, \kp, and \ca.
\end{enumerate}

\asvgf{figures/local-multi-compare.svg}{Astrocyte functional diagram for
  multi-synapse synaptic coupling}{fig:astro:syn_coupling}{0.5}

\section{Synaptic AND Coupling}
The first function explored for multi-synapse astrocyte plasticity is logical
AND. This is implemented by a single astrocyte, influencing N synapses of a
single \gls{lif} neuron (1nNs1a). The expected behavior of an \gls{lif} neuron implementing
AND can be defined in terms of single spikes as shown in Figure
\ref{fig:logical-and-coupling}.

\asvgf{figures/AstroAndCoupling.svg}{Logical Definition for AND
  Coupling}{fig:logical-and-coupling}{0.5}

In Figure \ref{fig:logical-and-coupling} each of the blue or orange boxes
represents a spike, either pre-synaptic, or post-synaptic. Figure
\ref{fig:global-v-local-and-coupling} shows the difference between the local
astrocyte response, vs. the global response required to implement logical
AND. Without any context of the response from other synapses, a strictly local
rule isn't able to correctly converge to an AND function. In most situations
outlined, the regional control logic will need to signal the correct response to
one or more synapses.

To implement this, first the typical activity local to a synapse is considered,
see Figure \ref{fig:local-ca-global-response}. These \ca responses propagate,
and are readily available at the global/regional level. Using the \ca transients
from each synapse, the global logic determines if weights should change on a
given synapse, and if so, in what direction. To communicate the proper response
\dser (which triggers a weight change at a synapse) and \serca, (which triggers
messenger degradation and re-uptake) signals are passed from the regional
control level, to individual synapses.

%% Define equations which outline the logic of AND coupling

\asvgf{figures/AstroAndCoupling_implementation.svg}{Comparison of astrocyte
  response locally, and the desired and coupling
  response}{fig:global-v-local-and-coupling}{0.5}

\asvgf{figures/AstroAndCoupling_signals.svg}{Logical definition for and
  coupling}{fig:local-ca-global-response}{0.5}

In essence, all spiking activity on N synapses (for the purpose of implementing
AND) fall into one of the following categories.
\begin{itemize}
\item AND - The neuron behaved correctly, this is either $pre_0, pre_1, ...,
  pre_n \implies post_o$ or $others -> \neg post_0$ : Reset all synapses
\item Other-influence - This situation occurs when $post_o$ comes before all
  $pre_i$ inputs. This implies something happening outside the control of
  associated synapses. It doesn't make sense to change the weight in this case,
  as it is unknown why there was a $post_0$ spike. In addition, the \gls{lif} neuron
  may be in a refractory period, and it wouldn't make sense to try and update
  weights during that: Reset all synapses.
\item Early-Spike - $post_0$ occurs before some spiking inputs, but after
  others. This would indicate one or more of the input synapses have weights
  that are too high: \gls{ltd} on synapses that received inputs.
\item LPT - If input spikes arrive at $pre_0, pre_1, ..., pre_N$ and there is no
  post-synaptic spike $post_0$, this implies weight values are too low: trigger
  \gls{ltp} on all synapses.
\end{itemize}

Equations (\ref{eq:ca_and_ltp}) through (\ref{eq:serca_cases}) formalized the behavior outlined
in Figure \ref{fig:local-ca-global-response} and the list above. In the context
of this chapter, resetting of a synapse (in response to a \serca signal) means
to reset all of the variables (\ca, \ipt, \kp) to zero, which is analogous to
the breakdown and re-uptake that occurs in biological astrocytes in response to
\gls{cicr}.

\begin{align}
  ca^i_{ltp} &= ca_i > thr_{and} \label{eq:ca_and_ltp} \\
  ca^i_{ltd} &= ca_i < -thr_{and} \\
  ca^i_{ip3} &= ca_i > thr_{ltp} \\
  ca_{ltp} &= \Sigma_{i=1}^N ca_{ltp}^i \\
  ca_{ltd} &= \Sigma_{i=1}^N ca_{ltd}^i \\
  ca_{ip3} &= \Sigma_{i=1}^N ca_{ip3}^i \\
  dser &=
  \begin{cases} 
    -1.0, ca_{ltp} > 0 \land ca_{ltp} < N \\
    1.0, ca_{ip3} > 0 \land ca_{ip3} < N \\
    0.0, otherwise
  \end{cases} \label{eq:dser_cases} \\
  serca &=
  \begin{cases}
    1.0, ca_{ltp} == N \\
    1.0, ca_{ltp} == 0 \land ca_{ltd} > 0 \\
    0.0, otherwise 
  \end{cases} \label{eq:serca_cases}
\end{align}

Where $N$ is the total number of synapses for AND coupling, $ca_i$ is the
calcium concentration of synapse $i$ that has diffused from local to regional
level. \dser and \serca are the signals from regional to local level, gating
plasticity and controlling chemical re-uptake.

Which of these categories in Figure \ref{fig:logical-and-coupling} a given set
of inputs and response belongs to, depends on some model parameters. Namely,
the timing constants and thresholds $thr_{and}$ and $thr_{ltp}$. Given a
set of parameters, there is a deterministic window of time spikes must fall into
in order to be considered for a particular response. Outside this window, from
the perspective of the astrocyte, those spikes never happened. This window is
dependent on many parameters however, and possibly changes given previous
activity. Instead of solving for the exact point in time a spikes falls out of
the time window, a lower bound is defined, $\delta_{ptp}=10ms$. The astrocyte
was tuned to guarantee proper behavior if spikes fall within this
window. This implies that the time between the first pre-synaptic spike, to the
first (possibly only) post-synaptic spike must be no more than this value. With
this timing constraint a variety of pre-pre-post spiking events can be generated
in a two-synapse configuration. This experiment will remove the influence of
synaptic weights and \gls{lif} neuron dynamics, focusing solely on astrocyte
response \ca.

Figure \ref{fig:2n1a_and_response} Shows the response of a single astrocyte to
two pre-synaptic spiking inputs, and single post-synaptic spiking outputs. Spikes
can arrive in any order, or not at all. If spikes do occur, they abide by the
$\delta_{ptp}=10ms$ constraint. Though the figure shows the spiking bouts
continuous in time, the astrocyte's internal state is reset between each.

\asvgf{figures/artifacts/obj3/astro_2s1a_and.svg}{Astrocyte with AND coupling
  response to two pre-synaptic spiking inputs, and a single post-synaptic spike
  randomly generated}{fig:2n1a_and_response}{0.4}

For the limited number of examples in Figure \ref{fig:2n1a_and_response}, the
astrocyte performs correct computations. This can be observed by comparing the
ground truth categorization of inputs (bottom plot) with the \dser and \serca
activity. For example: when the categorization is ``Early Spike'', \dser is $-1.0$ for a
single time-step, indicating \gls{ltd} on that synapse. Automating this comparison
process, an error rate can be computed, where each 10ms bout is labeled as a
match, or mis-match. A more exhaustive search, consisting
of 10,000 10ms bouts, showed zero mis-matches from the expected result,
\eq{eq:astro_2syn_exaust} shows that 10,000 iterations should be sufficient to
cover the entire input space for two synapses. Figures
\ref{fig:3n1a_and_response} and \ref{fig:4n1a_and_response} show the same success
for three and four synapses, which is supported by the 10k simulation test.

\begin{align}
  \textrm{possible AND spikes} &= \binom{11}{1}^3 = 1331 \label{eq:astro_2syn_exaust}
\end{align}

\eq{eq:astro_2syn_exaust} solves the counting problem ``How many
possible spiking configurations are'' for three synapses and a 10ms (10 discrete
1ms time steps) window. The $11^{th}$ possibility in this case, represents that a
spike does not occur on a synapse.

\asvgf{figures/artifacts/obj3/astro_3s1a_and.svg}{Astrocyte with AND coupling
  response to three pre-synaptic spiking inputs, and a single post-synaptic spike
  randomly generated}{fig:3n1a_and_response}{0.4}

\asvgf{figures/artifacts/obj3/astro_4s1a_and.svg}{Astrocyte with AND coupling
  response to four pre-synaptic spiking inputs, and a single post-synaptic spike
  randomly generated}{fig:4n1a_and_response}{0.4}

Though simulations thus far are a good indication of a performant astrocyte
model; Real-world applications will involve a set of continuous inputs, where
astrocyte and neuron states are maintained, not reset every 10ms. A continuous
simulation is explored next, with inputs generated by concatenating a number of
10ms bouts in time, without any resetting of astrocyte or Neuron state. In this
configuration, there are some situations where the astrocyte does not respond as
expected (a mis-match from ground truth), in general this is due to a \ca
deviation from zero, at the boundary where generated 10ms bouts are
joined. These situations made up $\approx 14\%$ of inputs, across 10,000 10ms
bouts. Figure \ref{fig:astro_2s1a_and_cont} shows some of the failure cases
encountered.

\asvgf{figures/artifacts/obj3/astro_2s1a_and_cont_invalid.svg}{Astrocyte with AND coupling
  response to a continuous timeline of two pre-synaptic spiking inputs, and a
  single post-synaptic spike randomly generated}{fig:astro_2s1a_and_cont}{0.4}

When considering $n=3$ and $n=4$ synapses with continuous randomly generated
spiking inputs, the error rate observed is consistent with that of $n=2$. For
$n=3$ there are 283 mismatches out of 2000 10ms bouts (14.15 \% error). For
$n=4$, there are 416 mismatches out of 3000 (13.9 \% error).

By default there are some cases present in the randomly generated spike train
that are impossible in the current configuration. Namely, cases where a
post-synaptic spike precedes all pre-synaptic spikes. This would be possible
with influence from other synapses outside of the AND-coupling domain, or
through the introduction of noise into the neuron model, but those cases aren't
considered here. With the removal of such cases, the error rate increases to
23.3 \% for 2 synapses, 22.0 \% for 3 synapses, and 18.0 \% for 4 synapses. The
increased error rate indicates that they astrocyte model benefits from sparsity
in activity needing a response.

Initial simulations which have isolated the astrocyte and provided random
inputs, indicate that in a majority of cases, the astrocyte is capable of
directing plasticity correctly to implement an AND function. The
\dser and \serca signaling reflects the activity of the global AND logic, and
the polarity of \dser matches the desired weight update. A fairly consistent
error rate of between 14 \% and 24 \% is observed, which is theorized to be low
enough for proper convergence to implement AND.

Closing the loop fully, the weight update behavior outlined in equation
\ref{eq:astro_and_dw} is implemented. With that addition, an astrocyte-\gls{lif} (1n2s1a)
configuration was simulated, given the same continuous inputs generated by
concatenating 10ms bouts, but only of pre-synaptic spikes for $n=2$
synapses. Where before, post-synaptic spikes were also randomly generated, now
any post-synaptic activity is driven by an \gls{lif} neuron with synaptic weights.
The default synaptic weight is $\approx 0.7$.

\begin{align}
  dw_{syn} = ca_{syn} lr_{ltp} \textrm{ if } ca_{syn} >
  0 \label{eq:astro_and_dw} \\
  dw_{syn} = ca_{syn} lr_{ltd} \textrm{ if } ca_{syn} < 0 \\
\end{align}

\asvgf{figures/artifacts/obj3/snn_1n2s1a_and_xlim.svg}{Simulation of astrocyte/\gls{lif}
  network in the 1n2s1a configuration, with the astrocyte directing synaptic
  plasticity}{fig:snn_2s1a_and}{0.4}

\asvgf{figures/artifacts/obj3/snn_1n2s1a_and.svg}{Weight change in the 2s1n1a
  configuration, with the astrocyte directing synaptic
  plasticity}{fig:snn_2s1a_and_w_dw}{0.6}

Looking at Figure \ref{fig:snn_2s1a_and}: Initially, synaptic weight values are
too high, and the neuron fires in response to a single pre-synaptic spike. As
expected, the astrocyte traces show \dser signals arriving at the offending
synapses, triggering \gls{ltd}. Completing the link, and allowing \ca to drive
plasticity, showed weight updates corresponding to the \dser signals and \ca
concentration. As the simulation progresses, weight values decrease, and the \gls{lif}
neuron begins to fire in response to two pre-synaptic spikes. There was a slight
over-correction, and some \gls{ltp} events occur, bringing the weights back
up. Eventually, the synaptic weights converge and offer a good approximation of
an AND function over the inputs. Looking at the bottom plot, there are two error
rate traces, labeled $+\Delta_w$ and $+\Delta_w$. These indicate example inputs
where the neuron responded incorrectly, and one or more synaptic weights neeeded
to be increased, or decreased respectively. Summing these two error rates give
an overall error rate.

Under the limited test conditions, Figure \ref{fig:snn_2s1a_and} demonstrates
success for coordinated astrocyte plasticity. This convergence to an AND
function needs further exploration. Since the coordinated plasticity was
developed to work with N synapses, it is important to evaluate other
configurations besides $n=2$.

\asvgf{figures/artifacts/obj3/snn_1n3s1a_and.svg}{Weight change in the 3s1n1a
  configuration, with the astrocyte directing synaptic
  plasticity}{fig:snn_3s1a_and_w_dw}{0.6}

\asvgf{figures/artifacts/obj3/snn_1n4s1a_and.svg}{Weight change in the 4s1n1a
  configuration, with the astrocyte directing synaptic
  plasticity}{fig:snn_4s1a_and_w_dw}{0.6}

Figures \ref{fig:snn_3s1a_and_w_dw} and \ref{fig:snn_4s1a_and_w_dw} show the
progression of learning in 3 and 4 synapse configurations respectively. In each
case, the addition of a synapse beyond 2 results in increased noise in the
convergence, towards the end of simulation. Exploring this convergence further,
initial weight values of each synapse are uniformly distributed from $0.0$ to
$2.0$. Figure \ref{fig:snn_4s1a_and_w_dw_w} shows the results, which indicate
the initial weight value does have some effect on convergence, but that overall,
a stable solution is reached given a range of initial values.

\asvgf{figures/artifacts/obj3/snn_4s1a_and_w.svg}{Weight change in the 1n4s1a
  configuration, with the astrocyte directing synaptic
  plasticity}{fig:snn_4s1a_and_w_dw_w}{0.6}

Though the learning rule converges to a degree, there is still a fair amount of
noise in the weight values as $n$ increases, and the mean error rate is upwards
of $20\%$. In order to better explain what is going on, observe the two error
rates, in the lower plots of Figures \ref{fig:snn_3s1a_and_w_dw} and
\ref{fig:snn_4s1a_and_w_dw}. There is an oscillation, between the majority of
errors involving $+\Delta W$ and $-\Delta W$. This indicates that optimal weight
values on all of the synapses, are being under and overshot repeatedly. This is
a common problem in learning rules, and is not a fundamental
drawback. Additional discussion and next steps were outlined in chapter
\ref{chapter:conclusion}.

\section{Summary}
%% Though there is a lack of general consensus, much of the Neuroscience literature
%% suggests that Astrocytes have a fast local response to synaptic activity, which
%% is then integrated to a cell-level response. Taking inspiration from this
%% multi-level integration approach, the Astrocyte model is extended. At the local
%% level, \ipt, \kp, and \ca dynamics remain mostly the same compared to previous
%% chapters. The \ca signals are generated local to the synapse, but now, in
%% addition they propagate up to a regional level. At this multi-synapse, regional
%% level \ca responses from multiple synapses are integrated and a response
%% generated. This response then travels back down via internal (to the astrocyte)
%% chemical signals to one or more synapses, affecting local plasticity. The two
%% regional functions are explored in this chapter, AND and NAND. In general, the
%% Astrocyte will drive weight values towards implementing the desired function,
%% detecting behavior that is not in alignment and updating weights
%% accordingly. For these functions in particular, having a multi-synapse view of
%% activity is critical. It is shown that in most cases, the locally controlled
%% plasticity explored in chapter two would result in incorrect weight
%% updates. With the regional logic mostly in control of when, and in what
%% direction weights move, local dynamics are responsible for the magnitude of a
%% weight change. In this way, global and local dynamics work together to implement
%% a coordinated learning model.

%% With this approach, it was shown that convergence to implementation of an AND
%% function could be achieved with 2, 3, and 4 synapse configurations. Each
%% consisting of a single \gls{lif} neuron, single Astrocyte. It was difficult to
%% determine a common set of parameters which gave optimal convergence across the
%% different number of synapses, but parameters were found that provided
%% convergence in each case.

Though there is a lack of general consensus, much of the Neuroscience literature
suggests that astrocytes have a fast local response to synaptic activity, which
is then integrated to a cell-level response. Taking inspiration from this
multi-level integration approach, the astrocyte model is extended. At the local
level, \ipt, \kp, and \ca dynamics remain mostly the same compared to previous
chapters. The \ca signals are generated local to the synapse, but now, in
addition they propagate up to a regional level. At this multi-synapse, regional
level \ca responses from multiple synapses are integrated and a response
generated. This response then travels back down via internal (to the astrocyte)
chemical signals to one or more synapses, affecting local plasticity. The two
regional functions are explored in this chapter, AND and NAND. In general, the
astrocyte will drive weight values towards implementing the desired function,
detecting behavior that is not in alignment and updating weights
accordingly. For these functions in particular, having a multi-synapse view of
activity is critical. It is shown that in most cases, the locally controlled
plasticity explored in chapter two would result in incorrect weight
updates. With the regional logic mostly in control of when, and in what
direction weights move, local dynamics are responsible for the magnitude of a
weight change. In this way, global and local dynamics work together to implement
a coordinated learning model.

With this approach, it was shown that convergence to implementation of an AND
function could be achieved with 2, 3, and 4 synapse configurations. Each
consisting of a single \gls{lif} neuron, single astrocyte. It was difficult to
determine a common set of parameters which gave optimal convergence across the
different number of synapses, but parameters were found that provided
convergence in each case.

    % Author       : Jacob Kiggins
% Contact      : jmk1154@g.rit.edu
% Date Created : 01/31/2021
%

\chapter{Future Work} \label{chapter:future-work}
This work covers a broad range of topics, and explores work in fields that are
still largely in their infancy. Plasticity, while well-defined in some areas of
machine learning, isn't as clear-cut in spiking neural networks. Further,
Astrocyte-Neuron interactions, aren't well understood in the Neuroscience
community, and the mathematical model representing them vary widely as a
result. The goal of this work was to extract some useful features, given what is
known about Astrocyte-Neuron interactions, using a simple model. This was
achieved, but the path taken leaves many interesting questions unanswered, and
opens the door to future work.

\section{Astrocyte Local Learning}
Local learning in this work was limited to two different types of inputs, and
the Astrocyte wasn't evaluated against any particular learning objective with
those input. Some interesting properties were teased out, such as the tendency of
the Astrocyte to change weights, seeking a state with low \ca response to
inputs. This is good, as that low \ca response configuration can be leveraged,
to direct an Astrocyte to a particular goal. The threshold function
\emph{Dw(\ca)} is fairly simple, more complicated functions should be explored
here, to determine what temporal features an Astrocyte can learn.

\section{Multi Synapse Plasticity}
This work just scratches the surface of what should be explored in a
multi-synapse configuration. One of the best next steps however, is to explore
input and learning objective pairings that allow for local and global dynamics
to work together with plasticity. A harmonious relationship between global and
local dynamics would allow a global rule to operate on a longer time-scale, and
implement a higher-level function.

%% \section{Multi Neuron and Synapse Coordination}

\section{Supervised Learning}

Astrocytes offer an interesting opportunity to work in tandom with a supervisory
learning signal. In may cases, especially with a longer running task-based
learning objective, feedback from a supervisory signal is sparse in
time. Astrocytes are in a fantastic position to bridge this gap. They can
maintain \ca concentrations over longer time-scales, and use their connections
and internal states to direct plasticity intelligently, given a global and
non-specific learning signal.

\chapter{Conclusion} \label{chapter:conclusion}

The main goal of this work was to identify key properties of Astrocyte-neuron
interactions in biology, that could be leveraged in a simplified computational
model, when paired with \Gls{lif} neurons. This initial question lead to the
identification of a variety of function roles, both concrete and theorized
surrounding Astrocytes in the Neuroscience literature. These include: working
memory, modulation of synaptic plasticity, synchronization of neural firing, and
long-range signaling across sub-networks of neurons. In paralell, common
chemical signaling pathways were teased from the same body of
research. The intersection of signaling pathways and functional role that best
fit this work was determined to be synapic plasticity, with a focus on
multi-level integration, across time and across synapses.

A computational Astrocyte model was developed to implement classic STDP
on a single synapse, using simplified versions of common Astrocyte chemical
pathways. Varying the parameters, it was shown that this Astrocyte model could
implement common variations on STDP, as well as shift and bias the standard
\Gls{stdp} weight update curve. Mimicing the \Gls{cicr} behavior observed in
Astrocytes, a threshold was defined and used to gate plasticity. This lead to
temporal integration of synaptic plasticity. Using this model, weight updates
were more stable when compared to \Gls{stdp} for the test input, and showed
convergence in a variety of configurations, where \Gls{stdp} either oscillated
or diverged. Exploring further, it was determined that the Astrocyte model was
driving weights to values that resulted in a lower \ca response. This feedback
mechanism allowed for flexibility in weight convergence, and opened the door for
implementing additional learning rules based on temporal integration.

Astrocytes have not been observed influencing only a single synapse, and based
on existing literature, their computational roles and benefits come from having
a multi-synapse view. To mirror this in the developed computational model, the
concept of synaptic coupling is introduced. Instead of strictly local dynamics
and weight updates, \ca local to a synapse is able to propagate to a regional
level, where some function is implemented across the participating synapses. At
this regional level, chemical signals are sent back to the local synapse level
depending on the function and synaptic activity. The function explored in this
work specifically, was logical AND. While simple, this function requires a
multi-synapse view to correctly implement, and can be scaled to an arbitrary
number of synapses easily. Results showed that an Astrocyte can implement a
robust learning rule capable of converging to weight values that implement AND
with 2 synapses. With 3 and 4 synapses, the Astrocyte still exhibits
convergence, but is only able to roughly arrive at a solution.

This work lays the groundwork for an Astrocyte-like approach to synaptic
plasticity, providing a computationally simple, bio-inspired model. This model
is shown to generalize classic \Gls{stdp}, and can improve upon it in a variety
of situations. Breaking away from the strictly local paradigm, the Astrocyte
model demonstrates coordinated plasticity in a learning task that would be
otherwise impossible with a strictly local view. Equally as important, this work
offers an opportunity for organization and coordination, when melding local and
global learning signals.

    % Author       : Jacob Kiggins
% Contact      : jmk1154@g.rit.edu
% Date Created : 01/31/2021
%

\chapter{Appendices} \label{chapter:astro_plasticity_model}

\section{Appendix A: Astrocyte Operating Mode Figures} \label{appendix:astro_figures}

\asvgf{figures/artifacts/obj1/astro_probe_dwdt-classic_stdp_rp.svg}{Astrocyte-Neuron Model Response to Pulse Pairs: Baseline}{}{0.4}
\asvgf{figures/artifacts/obj1/astro_probe_dwdt-anti_stdp_rp.svg}{Astrocyte-Neuron Model Response to Pulse Pairs: Anti-STDP}{}{0.4}
\asvgf{figures/artifacts/obj1/astro_probe_dwdt-ltd_bias_rp.svg}{Astrocyte-Neuron Model Response to Pulse Pairs: LTD Bias}{}{0.4}
\asvgf{figures/artifacts/obj1/astro_probe_dwdt-ltd_dt_shift_rp.svg}{Astrocyte-Neuron Model Response to Pulse Pairs: LTD Shift}{}{0.4}
\asvgf{figures/artifacts/obj1/astro_probe_dwdt-ltp_bias_rp.svg}{Astrocyte-Neuron Model Response to Pulse Pairs: LTP Bias}{}{0.4}
\asvgf{figures/artifacts/obj1/astro_probe_dwdt-ltp_dt_shift_rp.svg}{Astrocyte-Neuron Model Response to Pulse Pairs: LTP Shift}{}{0.4}

	
    % Bibliography file
    \printbibliography

    % Glossary
    \printglossary[type=main]

% End the document
\end{document} 
