\chapter{Synaptic Plasticity Using the Astrocyte Model in A Single Synapse
  Configuration} \label{chapter:obj2}

Astrocytes have suspected involvement in synaptic plasticity in both the
long \cite{min_2012} and short term \cite{pitta_2012}. A variety of
bio-inspired astrocyte models have touched on short term plasticity, and
suggested modulation of classical Hebbian learning \cite{pitta_2016}. In
addition to the experimental evidence, astrocytes are well placed to control
synaptic plasticity at many levels. Drilling down to the level of the
synapse, astrocytes monitor synaptic activity chemically, and respond
quickly with a variety of gliotransmitters, some of which are known to be
critical for \gls{ltp}/\gls{ltd} \cite{min_2012}. This fast local activity, which takes
the form of \ca transients in astrocyte end-foot processes is integrated
into a cell-level response, and can travel to distant synapses.

The goal in Chapter \ref{chapter:obj2} is to explore astrocyte mediated synaptic
plasticity in a single synapse configuration. This will close the loop with the
experiments in Chapter \ref{chapter:obj1}, by incorporating parameterized
synaptic weight changes, directed by the astrocyte. Focus was shifted from the
astrocyte's internal response, to the learning process, progression of weight
changes. Novel features of the full astrocyte plasticity model were
identified and the effect of parameters on these features explored.

%% with a single \gls{lif} neuron. Existing
%% literature outlines many ways that Astrocytes are shown to, are suspected to, or
%% have been modeled to affect their external environment. Table
%% \ref{able:obj2_astro_features} outlines these features of Astrocyte and
%% Neuron-Astrocyte interactions that were common threads in background and related 
%% work (see Chapter \ref{chapter:background}). Within this chapter a number of
%% these features have equivalent configurations in the developed Astrocyte model,
%% and the effects of that configuration are explored further.

%% %If the table is too wide, replace \begin{table}[!htp]...\end{table} with
%% \begin{adjustwidth}{-2.5 cm}{-2.5 cm}\centering\begin{threeparttable}[!htb]
%% %% \begin{table}[!htp]\centering
%% \caption{Generated by Spread-LaTeX}\label{table:obj2_astro_features}
%% \scriptsize
%% \begin{tabular}{lrrrrrr}\toprule
%% Astrocyte Feature &Primary Author &References &Astrocyte Model &Functional Model in Primary Work &Adaption in This Work \\\midrule
%% Ca Mediated increase in Synaptic Current &Wade &Wade2011, Wade2012 &Lin and Rizel or Variation &\ipt = L(incoming spikes), d\_ca = L(\ipt) + L(Ca) - NL(Ca) &i = z*w + i\_astro \\
%% Ca Mediated increase in Synaptic Weight &De Pitta &DePitta2016 &Lin and Rizel or Variation &Ca -> \gls{glu} -> increased PR &Threshold on Ca + constant multiplier to drive weight increase. Weight incrase is proportional to Ca, subject to a maximum change and scale factor \\
%% Modulation of Ca Based \gls{stdp} Curve &De Pitta &DePitta2016 &Lin and Rizel or Variation &Shifting and modulation of Ca based two-threshold \gls{stdp} &Differential Tau paramters on \ipt and \kp to achieve a similar result \\
%% Gating of \gls{ltp}/\gls{ltd} With D-Serine release &No primary Author & & &D-serine release opens the gate for synaptic plasticity through other mechanisms &The astrocyte drives plasticity, based on internal state including pesudo ip3, k+ and Ca \\
%% Working Memory &Gordleeva &Gordleeva2021 &Lin and Rizel or Variation &Astrocyte Calcium persists, and changes network sensitivity to prompt inputs & \\
%% \bottomrule
%% \end{tabular}
%% %% \end{table}
%% \end{threeparttable}\end{adjustwidth}

% TODO: Come up with a more generic name for Lin and Rizel model, and talk about
% how other models follow that basic approach, but that this work need not make
% a distinction between them.

\section{Calcium Threshold Triggered Weight Update}
One key astrocyte property highlighted in Chapter \ref{chapter:background} is
the concept of \gls{cicr}. Essentially, astrocytes integrate synaptic activity into a
\ca response, once that \ca concentration exceeds a threshold, there is a
cascade of events resulting in both \ca propagation to the cell body, and the
release of glio-transmitters into the synaptic cleft. This concept of \gls{cicr}, fits
well into the current model, and was explored in this chapter as the
mechanism that drives weight change.

In the context of this work's astrocyte model, as \ca values increase and exceed
a threshold $thr_{ca}$, a weight change will occur at the synapse, and the \ca
value degraded to zero, representing the re-uptake of \ca observed with \gls{cicr} in
biology. Tuning this threshold value, should lead to some improvements for both
weight convergence given a set of inputs, and tolerance to overall noise in the
system.

\begin{align}
  d_w = w (\alpha_{ltp} H(\cam - thr_{ca}) + \alpha_{ltd} H(-\cam - thr_{ca}))  \label{eq:dw-thr}
\end{align}

\eq{eq:dw-thr} describes the baseline concept for weight
change. $\alpha_{ltp/ltd}$ parameters serve as learning rates, and can be tuned
based on the type of inputs or astrocyte performance.


\section{Synchronization of Firing Rate}
While \gls{stdp}, and spike-timing based learning is the main focus of this work, the
proportional difference astrocyte configuration outlined in Section
\ref{sec:rate_response} shows a novel feature that has ties to Neuroscience
research. One of the possible functional roles of astrocytes in nature is
synchronization of synaptic firing (see Chapter \ref{chapter:background}). This
is thought to support computation in the brain, and drive detectable patterns we
call brain waves. It is interesting then, that a variation on the astrocyte
model in this work, is able to exhibit synchronization on a smaller scale.

Again, closing the loop from Chapter \ref{chapter:obj1}, the astrocyte directs
synaptic plasticity as outlined by \eq{eq:dw-thr}. Simulation with a
random Poisson spike train input, and $thr_{ca}=0$, shows that weight values
increase from their starting point, and approach a value where post-synaptic
spike track pre-synaptic.

%% \begin{align}
%%   d_w = ca / thr_{ca}  \label{eq:dw-thr-classic}
%% \end{align}

\asvgf{figures/artifacts/obj2/snn_1n1s1a_rp_thr_0.svg}{
  Astrocyte response to random poisson input with rate-based plasticity
  rule. $thr_{ltp}=-thr_{ltd}=0.8$}{fig:astro:rp}{0.5}

Figure \ref{fig:astro:rp_thr} show the effect of tuning $thr_{ca}$, which
requires a significant \ca response before synaptic weights can change. The
result is a more optimal convergence, which is stable, with inputs and outputs
exhibiting a high degree of synchronization.

\asvgf{figures/artifacts/obj2/snn_1n1s1a_rp_0.svg}{
  Astrocyte response to random poisson input with rate-based plasticity
  rule. $thr_{ltp}=-thr_{ltd}=1.5$}{fig:astro:rp_thr}{0.5}

\section{Spike-Timing Integration Plasticity}

One of the key features of the astrocyte model developed in chapter
\ref{chapter:obj1}, and an overarching goal of this thesis is the support for
detection and integration of coincident spikes, and subsequent synaptic
plasticity in an \gls{stdp}-like fashion. In Chapter \ref{chapter:obj1} it is shown
that the astrocyte model is capable of
implementing the classic \gls{stdp} curve, common variations on \gls{stdp}, and modulating
of the standard \gls{stdp} curve. This \gls{stdp}-like response however, was not a
plasticity response, but an internal \ca response. Moving on to this chapter,
that \ca response is being translated into a weight update. Given the definition
of $dw$ in \eq{eq:dw-classic}, weight updates will track the \ca
response, and an end-to-end implementation of \gls{stdp}
results. \eq{eq:dw-classic} is then used as a baseline during simulation, to
compare with features of the Spike-timing integration approach.

\begin{align}
  d_w = \cam / thr_{ca}  \label{eq:dw-classic}
\end{align}

This ``pulse-pair integration'' approach differs from classic \gls{stdp} in the
definition of $dw$, with the latter using \eq{eq:obj2:dw}, and in the
general flexibility of the astrocyte model, which isn't typical of \gls{stdp} in the
literature. Functionally, this makes spike-timing integrated plasticity different
in a few key ways. Firstly, weight updates are delayed, and
only occur when sufficiently many coincident spike occur to drive \ca
$>thr_{ca}$ in a short time window (dependent on time constant $\tau_{ca}$). Any
weight update will depend on a series of 
causal or anti-causal spiking events. The need for multiple events leads to 
increased confidence in each weight change, and can filter out some random spike
associations due to noise.  Furthermore, the approach has support from existing
Neuroscience and Engineering literature, and is similar to \gls{cicr}, where calcium
levels crossing some threshold lead to a release of gliotransmitters locally. In
the case of this work, that Gliotransmitter release is modeled as a change in
synaptic weight.

Equations (\ref{eq:obj2:dw}) and (\ref{eq:obj2:ca_reset}) fully
describe the weight update step. To evaluate the astrocyte
model in this configuration a series of spike impulses are used. These are
groups of regularly firing inputs in a $1010101$ pattern, with sufficient space
between these groups for astrocyte and Neuron states to settle. This input was
chosen as a benchmark, and to represent a burst of input, which is common in
biological systems.

\begin{align}
  d_w = H(\cam - thr_{ca}) \alpha_{ltp} + H(-\cam - thr_{ca})
  \alpha_{ltd} \label{eq:obj2:dw} \\
  \cam = \cam (1 - H(\cam - ca_{thr}) +  H(-ca - ca_{thr})) \label{eq:obj2:ca_reset}
\end{align}

As discussed before a slight variation allows the astrocyte model to implement
classic \gls{stdp}, outlined in \eq{eq:dw-classic}. In this case \ca
effects weight values and is reset every time-step. Since the \ca response was
already shown to mimic Classic \gls{stdp} in Chapter \ref{chapter:obj1},
\eq{eq:dw-classic} will as well.

\subsection{Impulse Response}
Figure \ref{fig:ppi:impulse_istp} depicts the astrocyte plasticity model with
$Ca_{thr}=2.5$, $\alpha_{ltp}=1.05$ and $\alpha_{ltd}=0.95$. In contrast, Figure
\ref{fig:ppi:impulse_stp} shows the behavior of a classic \gls{stdp} (implemented by
the astrocyte) with the same inputs and time-constants. The inputs themselves
consist of successive spikes, with 1ms between each. These spikes form groups
with enough time between for neuron and astrocyte states to decay. Each group
has an additional spike compared to the last.

Comparing these graphs, it is clear that integrating
behavior of the astrocyte model effects the learning process
significantly. Resulting in more stable behavior, and increasing synaptic weight
due to causal, correlated inputs. The key effect here can be seen considering
what happens when there are additional input spikes after a post-synaptic
spike. In this case classic \gls{stdp} would decrease the weight, while the astrocyte
integrates this event into \ca. Those input spikes are eventually associated
with an additional down-stream spike, forming a causal relationship and \ca is
increased. Overall this leads to \gls{ltp}, with the potential \gls{ltd} events being
smoothed out.

\asvgf{figures/artifacts/obj2/snn_1n1s1a_tp_pulse_astro_0.svg}{
  Astrocyte response to successive spikes, displaying temporal pulse-pair
  integration}{fig:ppi:impulse_istp}{0.6}

\asvgf{figures/artifacts/obj2/snn_1n1s1a_tp_pulse_classic_0.svg}{
  Classic \gls{stdp} response to successive spikes}{fig:ppi:impulse_stp}{0.6}

It is important to understand the situations under which the astrocyte model
will converge, and compare that to classic \gls{stdp}. Figures
\ref{fig:ppi:impulse_fixed_istp} and \ref{fig:ppi:impulse_fixed_stp} depict
responses to a similar series of spike impulses, but in this case each group of
spikes is the same length. There is a significant difference between the
behaviors in this case, with the astrocyte model quickly converging, while
classic \gls{stdp} appears to overshoot a stable configuration, oscillating back and
forth. Figure \ref{fig:ppi:impulse_fixed_stp_diverge} shows the same simulation
on a longer timescale, showing these oscillations continue and remain periodic.

% Astrocyte response to fixed length impulse inputs
\asvgf{figures/artifacts/obj2/snn_1n1s1a_tp_pulse_const_astro_0.svg}{Astrocyte
  response to a series of fixed length spike impulse inputs showing
  convergence}{fig:ppi:impulse_fixed_istp}{0.6}

% Classic \gls{stdp} with fixed length impulse inputs
\asvgf{figures/artifacts/obj2/snn_1n1s1a_tp_pulse_const_astro_0.svg}{Classic
  \gls{stdp} response to a series of fixed length spike impulse
  inputs}{fig:ppi:impulse_fixed_stp}{0.6}

% Classic STDP again, but with a longer simulation and only showing weights
\asvgf{figures/artifacts/obj2/snn_1n1s1a_tp_pulse_const_diverge_classic_0.svg}{Classic
  \gls{stdp} response to fixed length spikes showing divergent
  behavior}{fig:ppi:impulse_fixed_stp_diverge}{0.6}

% Comparison with noisy inputs
The smoothing effect observed with the astrocyte plasticity model should work
just as well at smoothing out noise, as it does transient causal or anti-causal
spike pairings. Testing this, Figures \ref{fig:ppi:impulse_noise_stp} and
\ref{fig:ppi:impulse_noise_istp} show the results of similar simulations, with
the addition of uniformly random noise in both cases. The probability specified,
is the probability that a noisy spike is received at the pre-synaptic
terminal during any single time-step. Observing the weight updates associated
with the astrocyte model, there isn't a significant impact from the noise on the
progression of weight modulation. In the case of classic \gls{stdp}, weight values
fluctuate even further, adding to the already unstable behavior.

% STDP response to impulse + uniform noise
\asvgf{figures/artifacts/obj2/snn_1n1s1a_tp_pulse_const_noise_classic_0.svg}{Classic
  \gls{stdp} response to impulse inputs with uniformly random
  noise}{fig:ppi:impulse_noise_stp}{0.6}

% Astro response to impulse + uniform noise
\asvgf{figures/artifacts/obj2/snn_1n1s1a_tp_pulse_const_noise_astro_0.svg}{Astrocyte
  response to impulse inputs with uniformly random
  noise}{fig:ppi:impulse_noise_istp}{0.6}

% Exploration of Weight update behavior with different parameters
The established baseline thus far, is that given this repeating (possibly
ramping up) spike impulse as an input, results in stable convergent weight
change from the astrocyte model. In contrast, classic \gls{stdp} shows instability and
lack of convergence. This is only a single configuration however, and there are many
parameters that could be modified. An investigation into the following
parameters can help determine if the convergent/noisy behavior is stable,
and what other behaviors of the model may arise.

\begin{itemize}
\item Initial weight
\item Spike association
\item Time constants
\item $dw$ factors, $\alpha_{ltp}$ and $\alpha_{ltd}$
\item Neuron threshold
\end{itemize}

Figure \ref{fig:sweep:mu} shows a sweep of initial weight values, and
demonstrates the effect on weight convergence. In both cases, if weight values
are sufficiently low, the general lack of post-synaptic activity leads to little
or no weight update. With higher initial weight values the astrocyte model was
able to converge as with previous simulations, while the \gls{stdp} diverged towards a
weight value of zero. The astrocyte model converged to a weight value as a
function of the initial weight, specifically the next stable configuration from
Figure \ref{fig:astro:tp_many_w_sweep} that can be reached by \gls{ltp}. This is a key
property of the weight update mechanics, in that the system seeks a stable
configuration.

\asvgf{figures/artifacts/obj2/snn_1n1s1a_tp_mu_sweep_0.svg}{Classic \gls{stdp} and
  astrocyte response to initial weight}{fig:sweep:mu}{0.4}

%% \asvgf{figures/artifacts/obj2/snn_1n1s1a_tp_sweep_association_0.svg}{Classic
%%   \gls{stdp} and Astrocyte Response to Different Spike Associations}{fig:sweep:assoc}{0.4}

Figure \ref{fig:sweep:astro_dw_factor} shows the effect of varying the weight
update factor $\alpha_{ltp}=\alpha_{ltd}$. In general, a very low factor results
in a lack of movement in the synaptic weight, whereas a high weight factor leads
to some stable regions being skipped over, but in each case some stable
configuration is found.

\asvgf{figures/artifacts/obj2/snn_1n1s1a_tp_sweep_dw_factor_astro_0.svg}{Astrocyte
response to different values for weight update
factor}{fig:sweep:astro_dw_factor}{0.4}

\asvgf{figures/artifacts/obj2/snn_1n1s1a_tp_sweep_dw_factor_classic_0.svg}{Classic
\gls{stdp} response to different values for weight update factor}{fig:sweep:stdp_dw_factor}{0.4}

Figure \ref{fig:sweep:stdp_dw_factor} shows the response of Classic \gls{stdp} to
different $\alpha_{ltp}=\alpha_{ltd}$. In general the response mimics the form
of the astrocyte weight progression, but exhibits significant instability, and
does not converge to a stable configuration. Figure \ref{fig:sweep:v_th} depicts
synaptic weight progression given different values for \gls{lif} neuron membrane
voltage threshold $v_{th}$. This experiment, meant to rule out a potential
avenue for increasing stability of Classic \gls{stdp} further supports the unique
stability offered by an astrocyte plasticity approach.

\asvgf{figures/artifacts/obj2/snn_1n1s1a_tp_sweep_lif_v_th_0.svg}{Classic \gls{stdp}
  and astrocyte response to different \gls{lif} neuron voltage threshold values}{fig:sweep:v_th}{0.4}

Figures \ref{fig:sweep:astro_tau} and \ref{fig:sweep:stdp_tau} show various
weight progressions across three different subplots, with the Figures
depicting astrocyte and Classic \gls{stdp} behavior respectively. In general,
modifying $\tau_*$ parameters, whether $\tau_{ip3}$, $\tau_{k+}$, or both leads
to the synaptic weight settling on the stable configuration around $w=1.1$
instead of $w=0.81$. In the case of Classic \gls{stdp}, some modified $\tau_*$
configurations lead to divergence to the maximum weight value ($20$), or
exhibited similar instability to what has been observed thus far.

\asvgf{figures/artifacts/obj2/snn_1n1s1a_tp_sweep_tau_astro_0.svg}{Astrocyte
  response to different values for $\tau_{ip3}$ and
  $\tau_{k+}$}{fig:sweep:astro_tau}{0.4}

\asvgf{figures/artifacts/obj2/snn_1n1s1a_tp_sweep_tau_classic_0.svg}{Classic
  \gls{stdp} response to different values for $\tau_{ip3}$ and
  $\tau_{k+}$}{fig:sweep:stdp_tau}{0.4}

\section{Summary}

This Chapter builds upon the astrocyte \gls{lif} models from the previous chapter,
adding a plasticity step, which translates \ca levels to a change
in synaptic weight. In order to implement \gls{stdp}, \ca concentration can be directly
translated into a weight change. This is the baseline used for comparison with
other variations on weight update dynamics.

Temporal integration is explored next, by defining a simple threshold on
\ca to drive plasticity. This means (in general) that multiple plasticity events
must occur on a given synapse before any change in weight occurs. A variety of
parameter values are explored, in both the ``Spike Timing'' and ``Rate-based''
astrocyte models. Compared to classic \gls{stdp}, the threshold approach exhibits
convergence where classic \gls{stdp} diverges, or exhibits oscillation. A threshold
update rule is able to consistently drive synaptic weights into one of the
astrocyte model's stable configurations. In addition, and learning and
convergence are stable in the presence of random input spikes following a
uniform distribution.

The observed improvements over classic \gls{stdp} were demonstrated across a variety
of parameter values, including varying $\tau_{ip3}$, $\tau_{k+}$, \gls{lif} $V_{th}$,
and the weight update multiplier. In each case where learning did occur, the
improvements over classic \gls{stdp} were observed.

