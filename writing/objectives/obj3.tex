\chapter{Extension Astrocyte Model To Multi-Synapse
  Configurations} \label{chapter:obj3}

In biology, Astrocytes are observed influencing many synapses, in general,
hundreds of synapse across a single digit number of neurons. This spatial
integration is a fundamental property of Astrocytes, and the response to
activity from these connected synapses is observed as a cell-level \ca
response. Taking this concept and applying it to the Astrocyte model
in this work, a coordinated approach to synaptic plasticity is developed. This
approach is employed in a learning task where a multi-synapse view is critical.

%% \section{Multi-Synapse Cell-Level Response}
%% What is the ultimage goal here... to support some learning on some specific set
%% of inputs, that would not otherwise be achievable w/ STDP. What is the task?
%% what is an interesting task that involves spike timing?

%% What are some things I've got in my notes about this?

%% Gating of plasticity
%%   - Only allow plasticity on synchronous inputs
%%   - The opposite of ^
  
%% Orchestration between Synapse
%%   - Don't allow multiple synapse to learn from the same post-synaptic spike (WTA
%%   Pairing)
%%   - Global learning ``pool'' Each thr event on any synapse depletes the pool,
%%   which then refills
%%   - Relative strength of plasticity


\section{Synaptic Coupling}
This chapter introduces the concept of synaptic coupling. This addition to the
Astrocyte model is inspired by the cell-level \ca responses Astrocytes
exhibit. Figure \ref{fig:astro:syn_coupling} shows the progression of the
Astrocyte model to support coupling. Synaptic coupling works by defining a
regional control area of the astrocyte, which is connected to a subset of the
total synapses influenced by an Astrocyte. At this regional level, the Astrocyte
implements some function across N synapses. \ca local to each synapse propagates
up to the regional level. To facilitate this function, additional chemical
signals pass down from regional level, to local.

\asvgf{figures/1nNs1a_diagram_obj3.svg}{Astrocyte Functional Diagram for
  Multi-Synapse Synaptic Coupling}{fig:astro:syn_coupling}{0.5}

\subsection{AND Coupling}
The first function explored for multi-synapse Astrocyte plasticity is logical
AND. This is implemented by a single Astrocyte, influencing N synapses of a
single LIF neuron (Nn1s1a). AND coupling can be defined in terms of single
spikes as shown in Figure \ref{fig:logical-and-coupling}.

\asvgf{figures/AstroAndCoupling.svg}{Logical Definition for AND
  Coupling}{fig:logical-and-coupling}{0.5}

In Figure \ref{fig:logical-and-coupling} each of the blue or orange boxes
represents a spike, either pre-synaptic, or post-synaptic. Figure
\ref{fig:global-v-local-and-coupling'} shows the difference between the local
Astrocyte response, vs. the global response required to implement logical
AND. In most situations, the regional control logic will need to provide input
to the local level, in order for weight updates to converge on an AND
function. To implement this, first the typical activity local to a
synapse is considered, see Figure \ref{fig:local-ca-global-response}. These \ca
responses propagate, and are readily available at the global/regional
level. Using the \ca transients from each synapse, the global logic determines
if weights should change on a given synapse, and if so, in what direction. To
communicate the proper response, two internal chemical signals are
employed. \dser which triggers a weight change at a synapse and \serca, which
triggers a reset, or re-uptake of \ca, \ipt, and \kp.

\asvgf{figures/AstroAndCoupling_implementation.svg}{Comparison of Astrocyte
  Response Locally, and the Desired AND Coupling
  Response}{fig:global-v-local-and-coupling}{0.5}

\asvgf{figures/AstroAndCoupling_signals.svg}{Logical Definition for AND
  Coupling}{fig:logical-and-coupling}{0.5}

In order to test the Astrocyte response and tune parameters, some timing
constraints must be defined. Initially, a single timing constraint
$\delta_{ptp}=10ms$ will be used. This implies that the time between the
first pre-synaptic spike, to the first (possibly only) post-synaptic spike must
be no more than this value to ensure proper behavior. With this timing
constraint a variety of pre-pre-post spiking events can be generated in a
two-synapse configuration. This experiment will remove the influence of synaptic
weights and LIF neuron dynamics, focusing solely on Astrocyte response \ca.

Figure \ref{fig:2n1a_and_response} Shows the response of a single Astrocyte to
two pre-synaptic spiking inputs, and single post-synaptic spiking outputs. Spike
can arrive in any order, or not at all. If spikes do occur, they abide by the
$\delta_{ptp}=10ms$ constraint. Though the figure shows the spiking bouts
continuous in time, the Astrocyte's internal state is reset between each.

\asvgf{figures/artifacts/obj3/astro_2s1a_and.svg}{Astrocyte with AND coupling
  response to two pre-synaptic spiking inputs, and a single post-synaptic spike
  randomly generated}{0.5}

For the limited number of examples in Figure \ref{fig:2n1a_and_response} it is
shown that the Astrocyte performs the correct computation. A more exhaustive
search, consisting of 10,000 10ms bouts, showed zero mismatches from the expected
response outlined in Figure \ref{fig:logical-and-coupling}. Though this is a
good indication, real-world inputs won't present in 
nice 10ms packets, or at least the goal is not to be constrained in such a
way. A continuous set of inputs is explored next, with inputs generated by
concatenating a number of 10ms bouts in time. In this configuration, there are
some situations where the Astrocyte does not respond as expected, in general
this is due to a \ca deviation from zero, at the boundary where generated 10ms
bouts are joined. These situations made up $\approx 12\%$ of possible inputs. Figure
\ref{fig:astro_2s1a_and_cont} shows some of the failure cases encountered across
10,000 10ms bouts.

\asvgf{figures/artifacts/obj3/astro_2s1a_and_cont.svg}{Astrocyte with AND coupling
  response to a continuous timeline of two pre-synaptic spiking inputs, and a
  single post-synaptic spike randomly generated}{0.5}

Initial simulations which have isolated the Astrocyte and provided random
inputs, indicate that in a majority of cases, the Astrocyte is capable of
directing plasticity correctly to implement an AND function. The
\dser and \serca signaling reflects the activity of the global AND logic, and
the polarity of \dser matches the desired weight update. In the case of Figure
\ref{fig:astro_2s1a_and_cont}, in only 12\% of 10ms bouts does the observed
Astrocyte response deviate from the Ground Truth.

Closing the loop fully, the weight update behavior outlined in equaion
\ref{eq:astro_and_dw} is implemented. With that addition, an Astrocyte-LIF (1n2s1a)
configuration was simulated, given the same continuous inputs generated by
concatenating 10ms bouts, but only of pre-synaptic spikes for $n=2$
synapses. Where before, post-synaptic spikes were also randomly generated, now
any post-synaptic activity is driven by an LIF neuron with synaptic weights.
The default synaptic weight is $\approx 0.7$. Figure \ref{fig:snn_2s1a_and}
shows

\asvgf{figures/artifacts/obj3/snn_2s1a_and.svg}{Simulation of Astrocyte/LIF
  network in the 2s1n1a configuration, with the Astrocyte directing synaptic
  plasticity}{0.5}

Initially, 
neuron to fire. As expected, the Astrocyte traces showed \dser signals
arriving at both synapses for all inputs that satisfied an AND
condition. Completing the link, and allowing \ca to drive plasticity, showed
weight updates corresponding to the \dser signals and \ca
concentration. Overall, with these simple inputs, 

\section{Summary}
Though there is a lack of general consensus, much of the Neuroscience literature
suggests that Astrocytes have a fast local response to synaptic activity, which
is then integrated to a cell-level response. Taking inspiration from this
multi-level integration approach, the Astrocyte model is extended. At the local
level, \ipt, \kp, and \ca dynamics remain mostly the same compared to previous
chapters. The \ca signals are generated local to the synapse, but now, in
addition they propagate up to a regional level. At this multi-synapse, regional
level \ca responses from multiple synapses are integrated and a response
generated. This response then travels back down via internal (to the astrocyte)
chemical signals to one or more synapses, affecting local plasticity. The two
regional functions are explored in this chapter, AND and NAND. In general, the
Astrocyte will drive weight values towards implementing the desired function,
detecting behavior that is not in alignment and updating weights
accordingly. For these functions in particular, having a multi-synapse view of
activity is critical. It is shown that in most cases, the locally controlled
plasticity explored in chapter two would result in incorrect weight
updates. With the regional logic mostly in control of when, and in what
direction weights move, local dynamics are responsible for the magnitude of a
weight change. In this way, global and local dynamics work together to implement
a coordinated learning model.

With this approach, it was shown that convergence to implementation of an AND
function could be achieved with 2, 3, and 4 synapse configurations. Each
consisting of a single LIF neuron, single Astrocyte. It was difficult to
determine a common set of parameters which gave optimal convergence across the
different number of synapses, but parameters were found that provided
convergence in each case.
