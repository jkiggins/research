% prologue.tex
%
% Author       : James Mnatzaganian
% Contact      : http://techtorials.me
% Date Created : 08/27/15
%
% Description  : The prologue used by "thesis.tex".
%
% Copyright (c) 2015 James Mnatzaganian

% NOTE: All filler text has "TODO" written. This must be removed in the final copy!

% Initialize starting pages to use Roman numerals
\frontmatter

% \begin{acknowledgments}
%%%%%%%%%%%%%%%%%%%%%%%%%%%%%%%%%%%%%%%%%%%%%%%%%%%%%%%%%%%%%%%%%%%%%%%%%%%%%%%
% Acknowledgments
%%%%%%%%%%%%%%%%%%%%%%%%%%%%%%%%%%%%%%%%%%%%%%%%%%%%%%%%%%%%%%%%%%%%%%%%%%%%%%%

\begin{acknowledgments}
	TODO
\end{acknowledgments}
% \end{acknowledgments}

% \begin{dedication}
%%%%%%%%%%%%%%%%%%%%%%%%%%%%%%%%%%%%%%%%%%%%%%%%%%%%%%%%%%%%%%%%%%%%%%%%%%%%%%%
% Dedication
%%%%%%%%%%%%%%%%%%%%%%%%%%%%%%%%%%%%%%%%%%%%%%%%%%%%%%%%%%%%%%%%%%%%%%%%%%%%%%%

\begin{dedication}
	TODO
\end{dedication}
% \end{dedication}

%\begin{abstract}
%%%%%%%%%%%%%%%%%%%%%%%%%%%%%%%%%%%%%%%%%%%%%%%%%%%%%%%%%%%%%%%%%%%%%%%%%%%%%%%
% Abstract
%%%%%%%%%%%%%%%%%%%%%%%%%%%%%%%%%%%%%%%%%%%%%%%%%%%%%%%%%%%%%%%%%%%%%%%%%%%%%%%

% Abstract
\begin{abstract}
	The mammalian brain is the most capable and complex computing entity known
	today. For many years there has been research focusing on reproducing the
	brain's processing capabilities. The first iteration of this process was
	Artificial Neural Networks. The use of sigmoid neurons and convolution
	operations have lead to significant success with many difficult
	problems. Object detection, style transfer, and segmentation of complex
	objects in noisy environments are examples. These systems can match and
	outperform humans in specific situations. Though in general, they require
	much more training, have higher power consumption, are more susceptible to
	noise, and can fail with adversarial inputs. In addition, state of the art
	models have a difficult time handling time-series data, and don't have any
	inherent memory, at least not with the same flexibility as humans
	display. Spiking neural networks are a logical next step to getting some of 
	the biological advantages on to computing platforms. This work explores a
	bio-plausible configuration of Spiking Neural Networks with Astrocytes,
	which displays desirable properties beyond what Spiking Networks are able
	to achieve alone, and bring the field a bit closer to leveraging some of the
	distinct advantages of biological neural networks.
\end{abstract}
%\end{abstract}

% \begin{introductory lists and tabels
%%%%%%%%%%%%%%%%%%%%%%%%%%%%%%%%%%%%%%%%%%%%%%%%%%%%%%%%%%%%%%%%%%%%%%%%%%%%%%%
% Introductory Lists and Tables
%%%%%%%%%%%%%%%%%%%%%%%%%%%%%%%%%%%%%%%%%%%%%%%%%%%%%%%%%%%%%%%%%%%%%%%%%%%%%%%

% Add TOC, list of figures, list of tables in that order
\makealllists

% Add the acronyms
\glsaddall
\printglossary[type=\acronymtype]

% Reset all acronyms
\glsresetall

% Start using Arabic numbers
\mainmatter
% \end{introductory lists and tabels}
