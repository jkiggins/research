% experiments.tex
%
% Author       : Jacob Kiggins
% Contact      : jmk1154@g.rit.edu
% Date Created : 05/23/2021
%

\chapter{Experiments}\label{chapter:miles_and_tasks}
\section{Evaluation and Testing}
It is necessary to Establish some basic criteria for tuning and evaluating these models. One such metric who's need arose early on is the order of magnitude for a single weight update. Initial simulations show STDP $\delta W$ on the order of $10^-9$, This didn't result in any significant changes to the model. In fact the loss graph was a straight line.

To establish a baseline, the original MNIST example, which used back-propagation as a training method was modified to track the weight changes across each batch. In general 
\section{Generating some Baselines}
In order to have a point of comparison, some baseline simulations are run. The first is

\section{A proposed Astrocyte Model}
It can be said with some fair level of certainty that astrocytes do the following things: Listen to activity at some number of synapses, respond with internal signals, affect spiking activity and plasticity at some regional level. In addition, the propagate signals internally, and hold state. The proposed model has all of these things, and presents them in a very general sense.

In order to have a usable implementation of this model a few details must be worked out. First are the neuron-astrocte and astrocyte-neuron interfaces. These will simply be voltage and current. If you think of the pre-synaptic signal as being a voltage spike, and the post-synaptic signal as being a current (making the weight a conductance) then these signals can be used directly as input to the astrocyte. More importantly, in order to affect the synapse, the astrocyte can simply inject current or voltage into the synapse. Next, we must determine what function $\delta Ca(pre_v, post_i)$ maps inputs to a change in internal state. For this, there are a variety of options. Recall the behavior of $Ca^{2+}$ within an astrocyte, a variety of external signaling can cause an increase in it (neurotransmitter uptake, ion uptake). In addition, there is an internal feedback loop involving the Endoplasmic Reticulum known as $Ca^{2+}$ mediated $Ca^{2+}$ release. Ignoring the equations governing $Ca^{2+}$ for now, it is reasonable to assume (due to the positive feedback) that the underlying behavior is exponential with respect to the inputs.

