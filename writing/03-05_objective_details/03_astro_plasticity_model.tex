% Author       : Jacob Kiggins
% Contact      : jmk1154@g.rit.edu
% Date Created : 01/31/2021
%

\chapter{Development of an Astrocyte Plasticity Model that Generalizes STDP} \label{chapter:astro_plasticity_modeal}
    The first objective of this thesis, is to develop an Astrocyte-synapse model
    that is capable of generalizing classic STDP along with common
    variants, with a common set of parameters.

    Astrocytes have suspected involvement in synaptic plasticity in both the
    long \cite{min_2012} and short term \cite{pitta_2012}. A variety of
    bio-inspired astrocyte models have touched on short term plasticity, and
    suggested modulation of classical Hebbian learning \cite{pitta_2016}. In
    addition to the experimental evidence, astrocytes are well placed to control
    synaptic plasticity at many levels. Drilling down to the level of the
    synapse, astrocytes monitor synaptic activity chemically, and respond
    quickly with a variety of gliotransmitters, some of which are known to be
    critical for LTP/LTD \cite{min_2012}. This fast local activity, which takes
    the form of Ca2+ transients in Astrocyte end-foot processes is integrated
    into a cell-level response, and can travel to distant synapses.

    Though the specifics of chemical signaling are still a mystery
    \cite{manninen_2018}. There is a common theme to what is known both
    chemically, and from bio-inspired models. This theme has potential to
    greatly improve upon existing STDP-like plasticity rules. Astrocytes behave
    as Master integrators of neural activity. They are capable of sensing
    activity across many thousands of synapses simultaneously. Integrating that
    activity quickly at the local level, and propagating signals to the entire
    cell body, integrating over a longer time-scale. Applying a similar approach
    to plasticity would allow synaptic strength updates to be based upon a
    timeline of recent activity, instead of a single event locally, and open the
    door for coordinated learning, with signals being integrated across many
    synapses, and signals traveling back to the local level.

    \section{Spiking Neuron Model}
    This work makes use of a Leaky Integrate and Fire (LIF) neuron model. LIF
    neurons are fairly simple computationally, relying on exponential functions
    and thresholds in the functional definition. While not the most
    bio-realistic (see Figure \ref{fig:sn_model_compare}), LIF neurons exhibit
    rich behavior with few parameters, and are easily extended. In addition, LIF
    neurons are the most widely used in SNN literature \cite{ponulak_2011}. The
    goal here, is to remain computationally simple, while still leveraging
    biology, which is the same approach used in the Astrocyte Model definition.

    Equations \ref{eq:lif:i}, and \ref{eq:lif:v} define the neuron current and
    voltage dynamics.
    
    \begin{align}
      d_{i} = -i(t) \tau_{syn} * d_t + z_{pre}(t) * w \label{eq:lif:i} \\
      d_{v} = -\tau_{mem} * (v(t) + i(t)) * d_t \label{eq:lif:v}
    \end{align}

    Here, $z_{pre}(t)$ represents the spiking activity on the synapse at time
    $t$ and has a value of either $0$ or $1$. $w$ is the synaptic
    weight. $\tau_{syn}$ and $\tau_{mem}$ are synapse and membrane decay
    time-constants.

    The neuron fires according to Equation \ref{eq:lif:fire}, where $H$ is the
    Heaviside step function.

    \begin{align}
      v'(t) = v(t) + dv \label{eq:lif:v_update} \\
      z_{post}(t) = H(v'(t) - thr_{mem}) \label{eq:lif:fire}
    \end{align}

    After firing the neuron membrane potential is reset to $v_{reset}$, which is
    generally a negative voltage $\leq thr_{mem}$, see Equation \ref{eq:lif:reset}.

    \begin{align}
      v(t) = v'(t) * (1 - z_{post}(t)) + z_{post}(t) * v_{reset} \label{eq:lif:reset}
    \end{align}


    Figure \ref{fig:1n1s1a_fn_diagram} depicts the LIF neuron described by
    equations above. Figures \ref{fig:lif:sample_1}, \ref{fig:lif:sample_2}, and
    \ref{fig:lif:sample_3} shows a sample of the LIF responses to a random
    Poisson input with various parameter values.

    \asvgf{figures/artifacts/obj1/lif_sample_mem-60.0_syn-500.0.svg}{LIF Neuron
      Response To Poisson Input. $tau_{mem}=60$ and
      $tau_{syn}=500$}{fig:lif:sample_1}{0.7}

    \asvgf{figures/artifacts/obj1/lif_sample_mem-30.0_syn-500.0.svg}{LIF Neuron
      Response To Poisson Input. $tau_{mem}=30$ and
      $tau_{syn}=500$}{fig:lif:sample_2}{0.7}

    \asvgf{figures/artifacts/obj1/lif_sample_mem-60.0_syn-250.0.svg}{LIF Neuron
      Response To Poisson Input. $tau_{mem}=60$ and
      $tau_{syn}=250$}{fig:lif:sample_3}{0.7}

    Comparing Figures \ref{fig:lif:sample_1} and \ref{fig:lif:sample_2}, the
    neuron seems more excitable with a higher $\tau_{mem}$. This is initially
    counter-intuitive, since a higher value would result in increased decay,
    however this $\tau_{mem}$ is also scales the effect of the input current on
    membrane voltage by the input current (see Equation \ref{eq:lif:v}). In
    addition, a larger decay factor results in decreased recovery time for the
    neuron from reset. In contrast a lower value for $\tau_{syn?}$ causes the
    neuron to fire more often with the same input, due to the input current
    decaying more slowly.


    %%%%%%%%%%%%% SGNN Architecture %%%%%%%%%%%%%
    \section{Model Development}
    
    Model development starts with a survey of current research, and
    understanding of the underlying biology of Astrocyte mediated
    plasticity. This portion of the process has been completed, see Section
    \ref{section:background}. Astrocyte models in the literature range from
    computationally simple models, which are generally paired with real-valued
    neurons. To complex mathematical models, which aim to mimic the calcium
    dynamics within an Astrocyte. The latter models are generally paired with
    spiking neural networks. From the neuroscience and engineering literature,
    it was possible to identify the input and output pathways within the
    tripartite synapse most likely to contribute to synaptic plasticity. These
    are outlined below.

    \begin{enumerate}
        \item Pre-synaptic Glu $\implies$ IP3 Pathway $\implies$ Ca pathway
        \item Post-synaptic depolarization K+ $\implies$ Ca pathway
        \item Ca $\implies$ D-serine release (more generally, plasticity modulation)
    \end{enumerate}

    \afigf{figures/1n1s1a_diagram.png}{Astrocyte and Neuron Model Diagram For An LIF Neuron with One Input}{fig:1n1s1a_fn_diagram}{}

    Figure \ref{fig:1n1s1a_fn_diagram} outlines the developed tripartite synapse
    model, as well as the internal dynamics of the LIF neuron. Some aspects of
    this model will be interchanged depending on the operating mode. These
    blocks are the function blocks in darker blue. The portions of the model
    which are consistent across operating modes are described first, followed by
    a deeper dive into the remaining blocks. The two input pathways, which drive
    the state values $ip3$ and $k+$ are described by Equations
    \ref{eq:astro:spike-ip3} and \ref{eq:astro:spike-k+}.

    \begin{align}
      d_{ip3} = d_t * (-ip3)\tau_{ip3} + \alpha z_{pre}(t) \label{eq:astro:spike-ip3} \\
      d_{k+} = d_t * (-k+)\tau_{k+} + \alpha z_{post}(t) \label{eq:astro:spike-k+}
    \end{align}

    Where $\tau_{ip3/k+}$ represents a time constant for a given pathway, $z_{pre}$ is
    1 if there is a pre-synaptic spike at time $t$ ($0$ otherwise), and
    $z_{post}$ is $1$ when the post-synaptic neuron fires ($0$
    otherwise).

    The \emph{Update(ip3, k+)} and \emph{Effect(u)} blocks define the behavior
    of the Astrocyte at a local level, given input from the internal state
    values, which are representative of external activity. There are two main
    operating modes of the Astrocyte that have been explored for this objective.

    In general, \emph{Effect(u)} represents $d_w$ and takes the form outlined in
    Equation \ref{astro:dw}. Whenever the Astrocyte state variable reaches the
    pre-determined threshold the associated weight is scaled by some factor
    $\alpha$, which may be different, depending on the direction of weight
    change.

    \begin{align}
      d_w = H(u - u_{thr}) * \alpha_{ltp} + H(-u - u_{thr}) * \alpha_{ltd} \label{astro:dw}
    \end{align}

    Here, $H$ is the heaviside step function, and $u_{thr}$ a constant threshold
    applied to $u$.
    
    \section{Exploring Input Pathways Responses}

    In order to get some general guidelines for valid parameter ranges, it is
    necessary to sweep parameters, and explore the model's response.
    
    \asvgf{figures/artifacts/obj1/sweep-alpha_tau-ip3-spike0.265.svg}{Model
      Response to Sweeping Input pathway $\alpha$, and
      $\tau_{ip3/k+}$}{fig:astro:sweep_alpha_ip3_tau_1}{0.6}

    \asvgf{figures/artifacts/obj1/sweep-alpha_tau-ip3-spike0.055.svg}{Model Response to
      Sweeping Input pathway $\alpha$, and
      $\tau_{ip3/k+}$}{fig:astro:sweep_alpha_ip3_tau_2}{0.6}

    The group of similar plots in Figures \ref{fig:astro:sweep_alpha_ip3_tau_1} and
    \ref{fig:astro:sweep_alpha_ip3_tau_2} depict Astrocyte responses to a common
    spike pattern, with different values for $\alpha_{ip3}=\alpha_{k+}$. The
    curves in these plots are scale multiples of each-other, showing that
    $\alpha_{ip3/k+}$ does not effect timing, but only scale of the Astrocyte
    response. The figures also indicate that small values of
    $\tau_{ip3}=\tau_{k+}$ can lead to a response that doesn't reach a
    steady-state within a reasonable window, such as with $\tau_{ip3/k+} = 10$.
    %% TODO: add more detail about steady-state

    
    \section{Ordered Proportional Astrocyte State Change}

    The first variant of the \emph{Update(ip3, k+)} function considers the
    magnitude of input pathway state variables $ip3$ and $k+$ as well as which
    variable is larger (the order). This gives a magnitude and sign to each
    update to the Astrocyte state variable $u$, eventually leading to LTP or
    LTD.

    \begin{align}
      T_{diff}(t) = ip3(t) - k^+(t) \\
      T_{delta}(t) = (H(T_{diff}(t) - thr_{ltp}) - H(-T_{diff}(t) - thr_{ltd})) * |T_{diff}| \\
      u(t+1) = -u(t) * \tau_{u} * d_t + T_{delta}(t) \label{eq:astro:rate-u} \\
      u(t+1) = u(t+1) * (1 - H(u - u_{thr})) \label{eq:astro:u-reset}
    \end{align}

    Where H is the Heaviside step function, $T_{diff}$ is the difference between
    $ip3$ and $K^+$ traces at some time $t$. $thr_{ltp}$ and $thr_{ltd}$ determine the valid
    ranges of $T_{diff}$ for depression or potentiation. $T_{delta}(t)$ describes
    the change in $u$ at a given time from external influence. Note that the
    threshold values determine the sign of $T_{diff}$, and provide the ability
    to shift LTP into negative regions, or LTD into positive. Increasing the
    distance between the $ltp$ and $ltd$ thresholds also allow for varying
    degrees of tolerance to transient activity that would otherwise lead to
    weight change.

    \asvgf{figures/artifacts/obj1/sweep_u_tau_p0.1.svg}{Response of $u$ to
      Poisson Distributed spikes applied to Pre and Post-Synaptic
      Pathways}{fig:astro:sweep_alpha_u_tau_1}{0.8}

    \asvgf{figures/artifacts/obj1/sweep_u_tau_p0.7.svg}{Response of $u$ to
      Poisson Distributed spikes applied to Pre and Post-Synaptic
      Pathways}{fig:astro:sweep_alpha_u_tau_2}{0.8}

    Figures \ref{fig:astro:sweep_alpha_u_tau_1} and
    \ref{fig:astro:sweep_alpha_u_tau_2} depict the response of $u$ for
    different values of $u_{tau}$ and different spiking events. In general,
    larger values of $\tau_u$ result in smaller deviations in $u$ overall, due
    to the value decaying more quickly. With smaller $\tau_u$ more of the
    previous activity is integrated into the value of $u$ and the response is
    more broad.

    Expanding on the simpler timeline plots, Figure
    \ref{fig:astro:u_thr_heat_rate_vs_tau_u} shows the number of times $u$ crosses its
    threshold for a given $\tau_u$ and $r$ spiking rate for the Poisson
    spike-train inputs. This graph can be used to tune either $u_{thr}$ or
    $\tau_u$ for a given application, and gives an impression of the
    excitability of the model.

    \asvgf{figures/artifacts/obj1/heatmap_spike-rate_tau_u.svg}{Rate of
      Astrocyte Effect on Synaptic Weight Given Different input Poisson spike
      rates, and different values of $\tau_u$}{fig:astro:u_thr_heat_rate_vs_tau_u}{0.3}

    \subsection{Classic-STDP and Variants} \label{sec:ordered_prop:stdp}
    
    Using the Ordered Proportional update rule the Astrocyte Plasticity Model is
    capable of implementing classic STDP, along with several variations.  Figure
    \ref{fig:astro:classic_stdp} shows the model's response to spike pairs with
    a variety of inter-spike-interval times (X-axis). This is a typical STDP
    curve.

    \asvgf{figures/artifacts/obj1/astro_probe_dwdt-classic_stdp_rp.svg}{Astrocyte and
      Neuron Model Response to Pulse Pairs: Baseline}{fig:astro:classic_stdp}{0.4}

    The model has additional parameters beyond what a traditional STDP rule
    would have, and this offers flexibility. By providing asymmetric
    $\tau_{ip3}$ and $\tau_{k+}$ the model can be bias toward LTD/LTP. Figure
    \ref{fig:astro:stdp_ltd_bias} shows this effect for LTD.

    \asvgf{figures/artifacts/obj1/astro_probe_dwdt-ltd_bias_rp.svg}{Astrocyte-Neuron
      Model Response to Pulse Pairs: LTD Bias}{fig:astro:stdp_ltd_bias}{0.4}

    In addition, threshold parameters $thr_{ltp}$ and $thr_{ltd}$ can be used to
    shift activity from the LTP to LTD region, or vise versa. Figure
    \ref{fig:astro:ltd_shift} shows this shifting behavior.
    
    \asvgf{figures/artifacts/obj1/astro_probe_dwdt-ltd_dt_shift_rp.svg}{Astrocyte-Neuron
      Model Response to Pulse Pairs: LTD Shift}{fig:astro:stdp_ltd_shift}{0.4}

    Additional figures can be found in Appendix \ref{appendix:astro_figures}.

    One of the main features of this model is it's ability to generalize STDP,
    and capture a variety of variants that have been proposed an explored in the
    literature. Table \ref{table:astro_varient_params} maps specific values of
    model parameters to the corresponding STDP behavior. If that
    variant of STDP has a name from other research it is included.

    \begin{table}[!htp]\centering
      \caption{Model Parameters Associated with STDP Variants} \label{table:astro_varient_params}
      \scriptsize
      \begin{tabular}{lrrrrrrrrrrr}\toprule
        STDP Variant &Name from Literature &$alpha_{ip3}$ &$tau_{ip3}$ &$alpha_{k+}$ &$tau_{k+}$ &$tau_u$ &$ltp_{thr}$ &$ltd_{thr}$ &$reset_{ip3}$ &$reset_{k+}$ \\\midrule
        Triplet STDP &Triplet STDP &1 &100 &1 &100 &10000 &0 &0 &Yes &Yes \\
        Classic & &1 &100 &1 &100 &10000 &0 &0 &No &No \\
        LTD Bias & &1 &100 &1 &30 &10000 &0 &0 &No &No \\
        LTP Bias & &1 &30 &1 &100 &10000 &0 &0 &No &No \\
        LTD Shift &Mirror STDP &1 &100 &1 &100 &10000 &0.5 &0.5 &No &No \\
        LTP Shift & &1 &100 &1 &100 &10000 &-0.5 &-0.5 &No &No \\
        Anti-STDP &Anti-STDP, Rev-STDP &-1 &100 &-1 &100 &10000 &0 &0 &No &No \\
        \bottomrule
      \end{tabular}
    \end{table}
    
    \section{Rate-based Plasticity}

    When input spiking activity is sparse in time, the model behaves as outlined
    in Section \ref{sec:ordered_prop:stdp} above. However, when more dense
    inputs are provided, the relationship between $ip3$ and $k+$ shifts from
    temporal, to activity-based. In the sparse case, $ip3 > k+$ implies that an
    incoming spike happened more recently then a post-synaptic spike. When
    incoming spikes are dense in time, $ip3 > k+$ would imply more activity on
    the input side (on average) than the output. What follows, is a Rate-based
    plasticity rule, where weights are updated to bring pre-synaptic, and
    post-synaptic firing rates to approximately the same value. Equations
    \ref{eq:astro:rate-u} and \ref{astro:dw} above describe u, and
    weight update behavior. Equation \label{eq:astro:u-reset} outlines the reset
    of behavior of $u$, the value reset to $0$ when $|u| > u_thr$.

    Figure \ref{fig:rate:lif_net_1} shows a timeline of activity for a single
    LIF neuron with a single synapse and single Astrocyte (1n1s1a
    configuration). Figure \ref{fig:rate:lif_net_2} shows a timeline for the
    same configuration, with $thr_{ltp}=1.5$ and $thr_{ltd}=-1.5$. The threshold
    values create a band from $-1.5$ to $1.5$ where differences in pre and
    post-synaptic activity are tolerated. This leniency results in fewer weight
    updates and prevents oscillations when compared to Figure
    \ref{fig:rate:lif_net_1}.

    \asvgf{figures/artifacts/obj1/snn_1n1s1a_rp_no-band_0.svg}{APM Response to
      Poisson Input With $thr_{ltp}=thr_{ltd}=0.0$}{fig:rate:lif_net_1}{0.6}
    
    \asvgf{figures/artifacts/obj1/snn_1n1s1a_rp_band_0.svg}{APM response to
      Possion Input With
      $thr_{ltp}=1.5,thr_{ltd}=-1.5$}{fig:rate:lif_net_2}{0.6} 
    
 
    \subsection{Integrated Spike Timing Plasticity}
    
    In the rate-based operating mode, sparse activity results in classic STDP,
    or a variant depending on parameters. As activity levels increase though,
    the relationship between ip3 and k+ becomes one of average activity, and not
    timing. One approach to mitigate the averaging effect would be lowering
    $\tau_{ip3}$ and $\tau_{k+}$, though this would reduce sensitivity to pulse
    pairs occurring on larger time-scales. A more elegant approach would be to
    modify \emph{Update(ip3, k+)}, combining the input traces in a way that
    preserves timing. $u$ can then be used to integrate pulse-pair events over
    time and drive weight change. Equation \ref{eq:astro:temp-u} describes the
    update for $u$.

    $reset_{ip3}$ and $reset_{k+}$ parameters, determine if additional spikes
    are added into the $ip3$ and $k+$ traces, or override them. These
    parameters, along with the weight update behavior determine which groupings
    of spike result in changes to $u$. Figure \ref{fig:astro:spike_associate}
    outlines different possible spike association behaviors.

    \asvgf{figures/SpikeAssociation.svg}{A Variety of Spike Associations
      Supported by the Neuron-Astrocyte model}{fig:astro:spike_associate}{0.5}

    \begin{align}
      T_{delta} = sign_{ltd} * z_{pre}(t) * k^+ + sign_{ltp} * z_{post}(t) * ip3 \\
      du = -u(t) * \tau_u + d_t + T_{delta} \label{eq:astro:temp-u}
    \end{align}

    In this case, $z_{post}$ and $z_{pre}$ are pre and post-synaptic spikes that
    the Astrocyte is sensing. The sign $sign_{ip3/k+}$ values may be functions
    of $k+$ and $ip3$ as well as some constants. For some experiments they will
    simply be $sign_{ltd}=-1.0$ and $sign_{ltp}=1.0$. Equations
    \ref{eq:astro:temp-sign-1} and \ref{eq:astro:temp-sign-2} show definitions
    of $sign_{ltd}$ and $sign_{ltp}$ that would allow a region around $dt=0$
    where weight updates were 0.

    \begin{align}
      sign_{ltd} = -H(k+ + thr_{ltd}) \label{eq:astro:temp-sign-1}\\
      sign_{ltp} = H(ip3 + thr_{ltd}) \label{eq:astro:temp-sign-2}
    \end{align}

    Unlike with the rate-based variation, updates to $u$ happen only when spikes
    occur on either pre or post-synaptic terminals, instead of continuously. As
    in the rate-based model, these updates to $u$ are integrated, and weight
    change occurs when $u$ exceeds its threshold.

    Depending on requirements, ip3 or k+ traces may be reset when u is updated,
    or reset when the weight is updated. This is similar to the one-to-many,
    many-to-many, or one-to-one (nearest neighbor) configurations seen with
    STDP.

    Figure \ref{fig:astro:1n1s1a_impulse} shows the behavior of the temporal
    integration model when a series of spike pulses are presented to the
    network. Each pulse is a single spike longer than the last. This simulation
    depicts the averaging effect of the model. With an impulse of 4 spikes, the
    neuron is driven enough to output a single spike, this happens around
    300ms. With 4 pre-synaptic spikes and 1 post-synaptic spike, the Astrocyte's
    internal state $u$ is increased, exceeding the threshold of $2.5$, and the
    synaptic weight is increased. Subsequent spike impulses have $>4$ spikes,
    which leads to an LTD configuration. At each subsequent burst of input
    activity past 300ms, $u$ can be seen to decrease. With traditional STDP,
    this would result in a weight decrease, however since this model averages
    the pulse-pair events, and the majority are LTP, no decrease in weight is
    observed.

    \asvgf{figures/artifacts/obj1/snn_1n1s1a_tp_stdp_u_thr-2.5_0.svg}{Simulation of
      Neuron-Astrocyte Model In 1S1N1A Configuration With Spike Pulses as
      Inputs}{fig:astro:1n1s1a_impulse}{0.5}
    

    %%%%%%%%%%%%% Extending STDP w/ in 1N1S Case %%%%%%%%%%%%%
    %% \section{Computational benefits of Synapically-mediated plasticity: Single Neuron Single Synapse}

    %% One of the key challenges holding back full-scale deployment of SNNs in
    %% industry, or their use for more complex models, is the availability of
    %% end-to-end supervised training rules.
    
    %% \subsection{Overview of Learning Rules}
    %% Learning rules for SNNs fall into one of a few catagories.
    %% \begin{enumerate}
    %%     \item Unsupervised, local, temporal based rules
    %%     \item Supervised rules attempting to emulate back-propagation
    %%     \item Supervised teaching-signal rules, such as ReSuMe
    %%     \item Supervised, reward-signal based rules, usually coupled with a
    %%       local-learning rule, such as STDP.
    %% \end{enumerate}
    
    %% Temporal learning rules are generally variants on STDP, with the differences
    %% being in the specifics of the equations mapping a time delay (between pre
    %% and post-synaptic spikes) to a weight update. Research has been done to
    %% couple STDP with a reward signal, providing a supervised approach
    %% \cite{rstdp_mnist}. In this specific implementation, supervised learning was
    %% restricted to the last two layers, with others being unsupervised. One
    %% advantage of a reward-modulated STDP approach, is the inherit support for
    %% recurrent spiking neural networks. This approach, or one similar to this
    %% will be used as a starting point for this research.
    
    %% Back-propagation has shown wide success in CNNs, achieving high performance
    %% in a variety of tasks. It then makes sense to at least attempt to apply this
    %% approach to SNNs. In the context of spiking neural networks,
    %% back-propagation has a few fundamental problems, making a successful
    %% algorithm difficult. First, the activation function many spiking neurons is
    %% generally a sum of dirac delta functions, which aren't differentiable. Some
    %% frameworks, such as Multi-SpikeProp are able to get around this first issue
    %% by using a continuous activation function \cite{deep_spike}. Other SNN-BP
    %% algorithms exist, such as spatio-temporal back-propagaion
    %% \cite{snn_stbp}. In this approach, an approximation of membrane potential is
    %% used instead of spike (essentially ignoring the activation function). This
    %% membrane potential quickly decays, and increases due to incoming
    %% spikes. Equation \ref{eq:snn_stbp:eq2} governs this behavior. A set of
    %% iterative equations (which take the place of a continuous solution to the
    %% O.D.E) govern membrane potential updates, only needing to be computed when a
    %% spike is generated. Using equations like this greatly reduces complexity
    %% when deriving back-propagation.
    
    %% \begin{align}
    %%     u(t) = u(t_{i-1})e^{\frac{t_{i-1} - 1}{\tau}} + I(t) \label{eq:snn_stbp:eq2}
    %% \end{align}
    
    %% A third approach to training was considered, called ReSuMe (remote
    %% supervised learning). This approach and its derivatives make use of a
    %% teaching signal, and update weights based on the difference between real
    %% output and the teaching signal. Unfortunately, all of the methods reviewed
    %% restrict neuron outputs to having 0 or 1 spikes within some time-interval,
    %% and received input from many-spiking neurons. This is too restricting for
    %% the goals of this research \cite{deep_spike}.
    
    %% Many of the gradient based learning approaches, while able to achieve good
    %% performance under specific conditions, and can circumvent the obvious
    %% roadblocks, still have significant drawbacks. First, they don't scale well
    %% to many-layer architectures, meaning these learning approaches couldn't
    %% train networks to match or exceed traditional CNN performance. Second, the
    %% learning approaches generally don't lend themselves to an feasible hardware
    %% implementation, removing the possibility for on-chip learning
    %% \cite{bp_stdp}. BP-STDP takes traditional back-propagation, and shows how it
    %% can be implemented in spiking networks as local STDP updates. Using some
    %% teaching signal, an error function is defined at the output, this error is
    %% then propagated backward through the network, using the forward weights to
    %% scale STDP (or anti-STDP) weight updates.
    
    %% \subsection{Selection of Bio-Inspired Rules}
    %% From the considered approaches two will be used, and adapted for use in this
    %% research. First, is the reward-modulated STDP. This approach has a few
    %% distinct advantages. It is independent of a specific neuron model, and can
    %% be used in conjunction with recurrent connections. Its simplicity provides
    %% flexibility for integration with an Astrocyte element, where Astrocyte
    %% dynamics can exist along side the learning rule. In addition, this learning
    %% rule is independent of neuron model, allowing for fast prototyping when
    %% selecting a neuron model. The rule will be adapted to allow for full-network
    %% training, likely using spike history to correlate an output at one time,
    %% with activity within the network in the past.
    
    %% Second the spatio-temporal update back-propagation will be evaluated and
    %% adapted for this research. The main advantage here is precise control over
    %% the activity of the network, since weights are updated according to spike
    %% timing throughout the network. This approach also supports end-to-end
    %% training as is. The approach will need to be adapted to support a more
    %% bio-inspired neuron model, if such a model is desired. This approach seemed
    %% the best of the back-propagation with SNN implementations, and can be used
    %% as a comparison to local, time-based learning rules such as R-STDP.
    
    %% \section{Develop the Foundation of a Scale-able FPGA Implementation}
    %% In an attempt to package the conclusions of this research into a usable
    %% implementation, both for industry and future research, an FPGA
    %% implementation will be developed. The goal is to develop the building blocks
    %% for an SGNN along with a small example proving these elements can work
    %% together. A large-scale implementation may not be possible given hardware
    %% constraints, but calculations showing scalability either on FPGA or
    %% neuromorphic hardware will be included instead.
    
    %% There are many examples in the literature of spiking networks being
    %% implemented in FPGA, however many of these employ simple LIF neurons, and
    %% don't have the added complexity of an Astrocyte element. Fortunately, there
    %% is at least one example for which to reference in attempting a digital
    %% implementation of complex SGNN dynamics. \cite{fpga_sgnn} develops a digital
    %% implementation of an Izhikevich spiking neuron, modeling the complex
    %% dynamics using piece-wise linear equations. Calcium dynamics within the
    %% Astrocyte are modeled similarly. Their use of the Izhikevich is especially
    %% helpful, since that is the upper limit for complexity as far as neuron
    %% models are concerned, that will likely be employed in this research. If it
    %% can be implemented in FPGA in an efficient way, then an efficient
    %% implementation of the neuron model which is eventually chosen is guaranteed
    %% to be feasible on FPGA.
    
    %% One of the key building blocks for an FPGA implementation will be the
    %% piece-wise-linear approximations of complex neuron and neuron-astrocyte
    %% dynamics. The approach in \cite{fpga_sgnn} can be generalized to any
    %% non-linear function of a single variable. The approach searches a space of
    %% possible linear approximations, consisting of 3 - 7 linear functions. For
    %% each of 3 - 7, 2 - 6 points are chosen, dividing the function into 3 - 7
    %% sections. For each section Least-Squares is used to find a linear
    %% best-fit. Combining error values from each of the sections provides an
    %% overall approximation error. Numerical methods are used to find the best
    %% points to divide up the function (for a given segment count). Comparing MSE
    %% from each of the specified segment counts, an optimum solution is chosen. A
    %% complex system is modeled this way, by approximating the fundamental pieces.
    
    %% As an example, consider Equations \ref{eq:fpga_sgnn_15} and
    %% \ref{eq:fpga_sgnn_16}.
    %% \begin{align}
    %%   0.04v^2 + 5v + 140 - u = 0 \label{eq:fpga_sgnn_15} \\
    %%   0.02(0.2v - u) = 0 \label{eq:fpga_sgnn_16}
    %% \end{align}
    
    %% Using the linear approximation steps above \cite{fpga_sgnn} presented
    %% approximations
    
    %% \begin{align*}
    %%   v = 0.5|v+68| + 0.5|v+57| - 22 - u + I \\
    %%   u = 0.02(0.2v - u)
    %% \end{align*}
    
    %% These equations are then discretized for simulation on FPGA using the Euler
    %% method, with $h=\frac{1}{2^6}$.
    
    %% \begin{align*}
    %%     v[n+1] = (0.5 * |v[n] + 68| + 0.5 * |v[n] + 57| - 22 - u[n] + I)h + v[n] \\
    %%     u[n+1] = (0.02(0.2v[n] - u[n]))h + u[n]
    %% \end{align*}
    
    %% Choosing h as a power of 2, and approximating 0.02 and 0.2 as powers of 2,
    %% the equations can be implemented on FPGA without a single multiplier, using
    %% only shifters and adders. This general approach, linear approximation then
    %% discritization with powers of 2 will be used to develop efficient scaleable
    %% neuron and Astrocyte implementations that can be used at scale.
