% AuThor       : Jacob Kiggins
% Contact      : jmk1154@g.rit.edu
% Date Created : 01/31/2021
%

\chapter{Future Work} \label{chapter:future-work}
This work covers a broad range of topics, and explores work in fields that are
still largely in their infancy. Plasticity, while well-defined in some areas of
machine learning, isn't as clear-cut in spiking neural networks. Further,
astrocyte-neuron interactions, aren't well understood in the neuroscience
community, and the mathematical model representing them vary widely as a
result. The goal of this work was to extract some useful features, given what is
known about astrocyte-Neuron interactions, using a simple model. This was
achieved, but the path taken leaves many interesting questions unanswered, and
opens the door to future work.

\section{Astrocyte Local Learning}
Local learning in this work was limited to two different types of inputs, and
the astrocyte wasn't evaluated against any particular learning objective with
those input. Some interesting properties were teased out, such as the tendency of
the astrocyte to change weights, seeking a state with low \ca response to
inputs. This is good, as that low \ca response configuration can be leveraged,
to direct an astrocyte to a particular goal. The threshold function
\emph{Dw(\ca)} is fairly simple, more complicated functions should be explored
here, to determine what temporal features an astrocyte can learn.

\section{Multi Synapse Plasticity}
This work just scratches the surface of what should be explored in a
multi-synapse configuration. One of the best next steps however, is to explore
\ca build-up at the regional level, on a time-scale slower than that of local
dynamics. This regional \ca would then determine when plasticity occurs, with
the hypothesis being that improvements in convergence would result, given the
additional temporal integration. 
%% \section{Multi Neuron and Synapse Coordination}

\section{Supervised Learning}

Astrocytes offer an interesting opportunity to work in tandom with a supervisory
learning signal. In may cases, especially with a longer running task-based
learning objective, feedback from a supervisory signal is sparse in
time. Astrocytes are in a fantastic position to bridge this gap. They can
maintain \ca concentrations over longer time-scales, and use their connections
and internal states to direct plasticity intelligently, given a global and
non-specific learning signal.

\chapter{Conclusion} \label{chapter:conclusion}

The main goal of this work was to identify key properties of astrocyte-neuron
interactions in biology, that could be leveraged in a simplified computational
model, when paired with \gls{lif} neurons. This initial question led to the
identification of a variety of function roles, both concrete and theorized
surrounding astrocytes in the neuroscience literature. These include: working
memory, modulation of synaptic plasticity, synchronization of neural firing, and
long-range signaling across sub-networks of neurons. In paralell, common
chemical signaling pathways were teased from the same body of
research. The intersection of signaling pathways and functional role that best
fit this work was determined to be synapic plasticity, with a focus on
multi-level integration, across time and across synapses.

A computational astrocyte model was developed to implement classic STDP
on a single synapse, using simplified versions of common astrocyte chemical
pathways. Varying the parameters, it was shown that this astrocyte model could
implement common variations on STDP, as well as shift and bias the standard
\gls{stdp} weight update curve. Mimicing the \gls{cicr} behavior observed in
astrocytes, a threshold was defined and used to gate plasticity. This led to
temporal integration of synaptic plasticity. Using this model, weight updates
were more stable when compared to \gls{stdp} for the test input, and showed
convergence in a variety of configurations, where \gls{stdp} either oscillated
or diverged. Exploring further, it was determined that the astrocyte model was
driving weights to values that resulted in a lower \ca response. This feedback
mechanism allowed for flexibility in weight convergence, and opened the door for
implementing additional learning rules based on temporal integration.

Astrocytes have not been observed influencing only a single synapse, and based
on existing literature, their computational roles and benefits come from having
a multi-synapse view. To mirror this in the developed computational model, the
concept of synaptic coupling is introduced. Instead of strictly local dynamics
and weight updates, \ca local to a synapse is able to propagate to a regional
level, where some function is implemented across the participating synapses. At
this regional level, chemical signals are sent back to the local synapse level
depending on the function and synaptic activity. The function explored in this
work specifically, was logical AND. While simple, this function requires a
multi-synapse view to correctly implement, and can be scaled to an arbitrary
number of synapses easily. Results showed that an astrocyte can implement a
robust learning rule capable of converging to weight values that implement AND
with 2 synapses. With 3 and 4 synapses, the astrocyte still exhibits
convergence, but is only able to roughly arrive at a solution.

This work lays the groundwork for an astrocyte-like approach to synaptic
plasticity, providing a computationally simple, bio-inspired model. This model
is shown to generalize classic \gls{stdp}, and can improve upon it in a variety
of situations. Breaking away from the strictly local paradigm, the astrocyte
model demonstrates coordinated plasticity in a learning task that would be
otherwise impossible with a strictly local view. Equally as important, this work
offers an opportunity for organization and coordination, when melding local and
global learning signals.
