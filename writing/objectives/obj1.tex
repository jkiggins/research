% Author       : Jacob Kiggins
% Contact      : jmk1154@g.rit.edu
% Date Created : 01/31/2021
%

\chapter{Development of an Astrocyte Model Sensitive to Spike Timing} \label{chapter:obj1}
The first objective of this thesis, is to develop an astrocyte-synapse model
that is capable of responding to pre and post-synaptic spike timing, and
explore how internal and external configuration affects this response. These
internal signals will eventually be used to drive synaptic plasticity, and
generalize \gls{stdp}. That theme will guide the experiments in this chapter.

Though many details of chemical signaling within astrocytes are still unknown
\parencite{manninen_2018}, there is a common theme to what is known both chemically,
and from bio-inspired models. This theme has potential to greatly improve upon
existing \gls{stdp}-like plasticity rules. Astrocytes behave as Master integrators of
neural activity. They are capable of sensing activity across many thousands of
synapses simultaneously. Integrating that activity quickly at the local level,
and propagating signals to the entire cell body, integrating over a longer
time-scale. Applying this multi-level integration approach to synaptic
plasticity leads to many interesting questions and potential benefits beyond
existing local learning rules to explore.

\section{Spiking Neuron Model}
This work makes use of a Leaky Integrate and Fire (\gls{lif}) neuron model. \Gls{lif}
neurons are fairly simple computationally, relying on exponential functions
and thresholds in the functional definition. While not the most
bio-realistic (see Figure \ref{fig:sn_model_compare}), \gls{lif} neurons exhibit
rich behavior with few parameters, and are easily extended. In addition, \gls{lif}
neurons are the most widely used in SNN literature \parencite{ponulak_2011}. The
goal here, is to remain computationally simple, while still leveraging unique
features found in biology. This same approach used in the astrocyte model
definition.

\eq{eq:lif:psp}, and (\ref{eq:lif:v}) define the differential equations
that govern neuron post-synaptic potential and membrane voltage dynamics.

\begin{align}
d_{vsyn} = -v_{syn}(t) \tau_{syn} d_t + z_{pre}(t) w \label{eq:lif:psp} \\
d_{vmem} = -\tau_{mem} (v_{mem} + v_{syn}) d_t \label{eq:lif:v}
\end{align}

Here, $z_{pre}$ represents the spiking activity on the synapse at time
$t$ and has a value of either $0$ or $1$. $w$ is the synaptic
weight. $\tau_{syn}$ and $\tau_{mem}$ are synapse and membrane timing
constants.

The neuron fires according to \eq{eq:lif:fire}, where $H$ is the
Heaviside step function, and $thr_{mem}$ is the $v_{mem}$ threshold that
triggers the neuron to fire.

\begin{align}
z_{post}(t) = H(v_{mem}(t) - thr_{mem}) \label{eq:lif:fire}
\end{align}

After firing the neuron membrane potential is reset to $v_{reset}$, which is
generally a negative voltage $\leq thr_{mem}$, see \eq{eq:lif:reset}.

\begin{align}
v_{mem}(t+1) = v_{mem}(t)(1 - z_{post}(t)) + z_{post}(t)v_{reset} \label{eq:lif:reset}
\end{align}

Figure \ref{fig:1n1s1a_fn_diagram} depicts the \gls{lif} neuron described by
equations above. Figures \ref{fig:lif:sample_1}, \ref{fig:lif:sample_2}, and
\ref{fig:lif:sample_3} shows a sample of the \gls{lif} responses to a random
Poisson input with various parameter values. Table \ref{table:lif_params} shows
the baseline parameter values, with any differences being outlined in the
figures. In each figure, blue traces represent the post-synaptic potential
voltage wave, while purple represents the membrane voltage.

\begin{table}[!htp]\centering
  \caption{\Gls{lif} Neuron baseline parameters} \label{table:lif_params}
  \scriptsize
  \begin{tabular}{lrrrr}\toprule
    Reset Voltage &Membrane Threshold &Membrane Tau &Synaptic Tau \\\midrule
    -0.2 v &0.2 v &60 1/s &500 1/s \\
    \bottomrule
  \end{tabular}
\end{table}

\asvgf{figures/artifacts/obj1/lif_sample_mem-60.0_syn-500.0.svg}{\Gls{lif} Neuron
response to poisson input. $\tau_{mem}=60$ and
$\tau_{syn}=500$}{fig:lif:sample_1}{0.7}

\asvgf{figures/artifacts/obj1/lif_sample_mem-30.0_syn-500.0.svg}{\Gls{lif} Neuron
response to poisson input. $\tau_{mem}=30$ and
$\tau_{syn}=500$}{fig:lif:sample_2}{0.7}

\asvgf{figures/artifacts/obj1/lif_sample_mem-60.0_syn-250.0.svg}{\Gls{lif} Neuron
response to poisson input. $\tau_{mem}=60$ and
$\tau_{syn}=250$}{fig:lif:sample_3}{0.7}

Comparing Figures \ref{fig:lif:sample_1} and \ref{fig:lif:sample_2}, the
neuron seems more excitable with a higher $\tau_{mem}$. This is initially
counter-intuitive, since a higher value would result in increased decay,
however this $\tau_{mem}$ also scales the effect of the input current on
membrane voltage by the input current (see \eq{eq:lif:v}). In
addition, a larger decay factor results in decreased recovery time for the
neuron from reset. In contrast a lower value for $\tau_{syn}$ causes the
neuron to fire more often with the same input, due to the input current
decaying more slowly.

\section{Classic STDP} \label{obj1:sec:classic_stdp}
    
The proposed astrocyte model, in the single-synapse configuration aims to
generalize \gls{stdp}, and offer some additional effects and features that are
distinct from \gls{stdp}. In order to have a point of comparison, an \gls{stdp}
implementation must be chosen to use in this capacity. The chosen model,
referenced by \parencite{song_2000} is outlined by \eq{eq:song_classic_stdp}.

\begin{align}
  F(\delta t) =
  \begin{cases} 
    Ae^{- \frac{|t_{pre}-t_{post}|}{\tau}} & t_{pre} - t_{post} \leq 0, A > 0
    \\ Ae^{- \frac{|t_{pre}-t_{post}|}{\tau}} & t_{pre} - t_{post} > 0, B < 0
  \end{cases} \label{eq:song_classic_stdp}
\end{align}

This equation is evaluated for various $\delta t$ between $-50ms$ and
$50ms$. The resulting curve is outlined in Figure \ref{fig:stdp_dw_dt}. This
functional definition is not directly implemented since there is not timeline of
spikes to work with, instead, the exponential function is approximated by
integrating Equations (\ref{eq:song_impl_pre}), (\ref{eq:song_impl_post}), and
\ref{eq:song_impl_dw} over time. Table \ref{table:classic_stdp_params} shows the
parameters used during simulation.

\begin{table}[!htp] \centering
  \caption{Parameters for classic \gls{stdp}} \label{table:classic_stdp_params}
  \scriptsize
  \begin{tabular}{lrr}\toprule
    A &Tau \\\midrule
    1 &100 1/s \\
    \bottomrule
  \end{tabular}
\end{table}

\asvgf{figures/artifacts/obj1/closed_form_stdp.svg}{Simulation of charicteristic
  stdp equation}{fig:stdp_dw_dt}{0.4}

\begin{align}
  \delta trace_{pre} = -H(Z_{pre})(\tau trace_{pre}) +
  A Z_{pre} \label{eq:song_impl_pre} \\ 
  \delta trace_{post} = -H(Z_{post})(\tau trace_{post}) +
  A Z_{post} \label{eq:song_impl_post} \\
  F = H(Z_{pre}) trace_{post} + H(Z_{post}) trace_{pre} \label{eq:song_impl_dw}
\end{align}

Depending on the desired behavior, the $H(Z_*)$ term may be dropped from the
update Equations, this is related to the spike associations outlined in Figure
\ref{fig:astro:spike_associate}.

Figure \ref{fig:stdp_dw_dt_impl} shows a sweep of spike pairs as before from
$-50ms$ to $50ms$. This figure matches Figure \ref{fig:stdp_dw_dt}, showing that
the implementation matches the theoretical equations.


\asvgf{figures/artifacts/obj1/1n1s1a_tp_l-stdp_w-dw_mult_dwdt_astro_plasticity.svg}{Simulation
  of classic \gls{stdp} implementation, equations \ref{eq:song_impl_pre} -
  \ref{eq:song_impl_dw}}{fig:stdp_dw_dt_impl}{0.5}


%%%% Model Development Section %%%%
\section{Model Development}
Astrocyte model development starts with a survey of current research, and
understanding of the underlying biology of astrocyte mediated
plasticity. Astrocyte models in the literature range from computationally simple
models, which are generally paired with perceptron neurons to complex
mathematical models, which aim to mimic the calcium dynamics precisely. The
latter models are generally paired with spiking neural networks. From the
Neuroscience and engineering literature, it was possible to identify the common
input and output pathways within the tripartite synapse. 

\begin{enumerate}
  \item Pre-synaptic \gls{glu} $\implies$ \ipt Pathway $\implies$ \ca pathway
  \item Post-synaptic depolarization \kp $\implies$ \ca pathway
  \item \ca $\implies$ D-serine release (more generally, plasticity modulation)
\end{enumerate}

\asvgf{figures/1n1s1a_diagram.svg}{Astrocyte and neuron model diagram for an \gls{lif}
  neuron with one input}{fig:1n1s1a_fn_diagram}{0.5}

Figure \ref{fig:1n1s1a_fn_diagram} outlines the developed tripartite synapse
model, as well as the internal dynamics of the \gls{lif} neuron. Some aspects of this
model are left undefined at first, and different configurations were explored
experimentally. These gaps are the function blocks in darker blue. The
portions of the model which are consistent throughout this work are described
first, followed by a deeper dive into the remaining functionality. The two input
pathways, which drive the state values $ip3$ and $k+$ are described by Equations
(\ref{eq:astro:spike-ip3}) and (\ref{eq:astro:spike-k+}). It should be noted, that
the astrocyte response pathways are similar in form to an \gls{lif} neuron's membrane
voltage. This decision is driven in part by a simplification on some
bio-realistic models, but also to demonstrate that key properties of astrocytes
are dependent on the pathways and relative time-constants, less then the specifics of
\ca, \ipt, or \kp dynamics.

\begin{align}
  d_{ip3} = d_t (-IP_3)\tau_{ip3} + \alpha_{ip3} z_{pre}(t) \label{eq:astro:spike-ip3} \\
  d_{k+} = d_t (-K^+)\tau_{k+} + \alpha_{k+} z_{post}(t) \label{eq:astro:spike-k+}
\end{align}

Where $\tau_{ip3/k+}$ represents a time constant for a given pathway, $z_{pre}$
is $1$ if there is a pre-synaptic spike at time $t$ ($0$ otherwise), and
$z_{post}$ is $1$ when the post-synaptic neuron fires ($0$ otherwise). For all
simulations (unless otherwise specified) $d_t = 0.001$ or $1ms$.

The \emph{Ca(\ipt, \kp)} and \emph{Dw(\ca)} blocks define the behavior
of the astrocyte at a synapse local level, given input from the internal state values,
which are representative of external activity. In this chapter only the internal
state of the astrocyte was considered so \emph{Dw(\ca)} is a
No-Op. For many of the astrocyte model variations explored in this chapter, a
key property is stability of the response given a set of weight
values. Stability, in this case, refers to either a lack of \ca response, or
sub-threshold \ca response. This was an important property when considering
a learning bout, as stable points represent places that learning will converge,
with weight values remaining the same.

%% TODO: put this somewhere in Obj2
%% In general, \emph{Effect(u)} represents $d_w$ and takes the form outlined in
%% \eq{astro:dw}. Whenever the astrocyte state variable reaches the
%% pre-determined threshold the associated weight is scaled by some factor
%% $\alpha$, which may be different, depending on the direction of weight
%% change.

%% \begin{align}
%%   d_w = H(u - u_{thr}) * \alpha_{ltp} + H(-u - u_{thr}) * \alpha_{ltd} \label{astro:dw}
%% \end{align}

%% Here, $H$ is the heaviside step function, and $u_{thr}$ a constant threshold
%% applied to $u$.
    
\section{Exploring Input Pathways Responses}

In order to get some general guidelines for valid parameter ranges, it is
necessary to sweep parameters, and explore the model's response. The parameters
used during simulation are outlined in Table \ref{table:astro_in_path_params},
with any deviations mentioned in the graph.

\begin{table}[!htp] \centering
  \caption{Common astrocyte input pathway parameters} \label{table:astro_in_path_params}
  \scriptsize
  \begin{tabular}{lrrrrr}\toprule
    dt &Alpha \ipt &Tau \ipt &Alpha \kp &Tau \kp \\\midrule
    0.001 s &1 &100 1/s &1 &100 1/s \\
    \bottomrule
  \end{tabular}
\end{table}

\asvgf{figures/artifacts/obj1/sweep-alpha_tau-ip3-spike0.225.svg}{Model
response to sweeping input pathway $\alpha$, and
$\tau_{ip3/k+}$}{fig:astro:sweep_alpha_ip3_tau_1}{0.5}

\asvgf{figures/artifacts/obj1/sweep-alpha_tau-ip3-spike0.055.svg}{Model response to
sweeping input pathway $\alpha$, and
$\tau_{ip3/k+}$}{fig:astro:sweep_alpha_ip3_tau_2}{0.5}

The group of similar plots in Figures \ref{fig:astro:sweep_alpha_ip3_tau_1} and
\ref{fig:astro:sweep_alpha_ip3_tau_2} depict astrocyte responses to a common
spike pattern, with different values for $\alpha_{ip3}=\alpha_{k+}$. The
curves in these plots are scale multiples of each-other, showing that
$\alpha_{ip3/k+}$ does not effect timing, but only the scale of the astrocyte
response. The figures also indicate that small values of
$\tau_{ip3}=\tau_{k+}$ can lead to a response that doesn't reach a
steady-state within a reasonable window, such as with $\tau_{ip3/k+} = 10$.
%% TODO: add more detail about steady-state


\section{Signed Proportional \ca Integration} \label{sec:sign_prop}

To support a plasticity rule with \gls{stdp}-like characteristics, an astrocyte model
needs to be sensitive to the relative spike timing of pre and post-synaptic
spikes. One approach to achieve this, is to specify a \emph{Ca(ip3, k+)}
function that considers the magnitude
of input pathway state variables \ipt and \kp as well as which variable is
larger (the order). This gives a magnitude and sign to each update to the
astrocyte state variable \ca. The value of \ca will eventually leading to \gls{ltp} or
\gls{ltd} when explored in Chapter \ref{chapter:obj2}. This approach is outlined by Equations
(\ref{eq:astro:rate-u}) and (\ref{eq:astro:u-reset}), which define the change in
\ca.

\begin{align}
  T_{diff} = ip3 - k^+ \\
  dca_{delta} = (H(T_{diff} - thr_{ltp}) - H(-T_{diff} - thr_{ltd})) |IP_3 K^+| \\
  dca_{decay} = -ca \tau_{ca} d_t \\
  dca = dca_{delta} + dca_{decay} \label{eq:astro:rate-u} \\
  \cam = \cam (1 - H(\cam - thr_{ca})) \label{eq:astro:u-reset}
\end{align}

Where H is the Heaviside step function, $T_{diff}$ is the difference between
\ipt and \kp traces at some time $t$. $thr_{ltp}$ and $thr_{ltd}$ determine
the valid ranges of $T_{diff}$ for an update to \ca to occur. $dca_{delta}$
describes the change in \ca at a given time from external influence, while
$dca_{decay}$ accounts for the leaking of \ca. Note that the threshold values
determine the sign of $dca_{delta}$, and provide the ability to shift the \ca
response, as shown in Figure \ref{fig:astro:stdp_ltd_shift}. Increasing the distance
between the \gls{ltp} and \gls{ltd} thresholds also allow for varying degrees of
tolerance to transient activity that would otherwise lead to a response by the
astrocyte. Simulations use the baseline parameter values outlined in Table
\ref{table:ordered_prop_params}, with any variations or parameter sweeps
outlined in graphs.

\begin{table}[!htp]\centering
  \caption{Signed proportional astrocyte parameters} \label{table:ordered_prop_params}
  \scriptsize
  \begin{tabular}{lrrrrrr}\toprule
    dt &Alphas &Tau \ipt &Tau \kp <P Threshold <D Threshold \\\midrule
    0.001 s &1 &100 1/s &100 1/s &0 &0 \\
    \bottomrule
  \end{tabular}
\end{table}

\asvgf{figures/artifacts/obj1/sweep_u_tau_p0.1.svg}{Response of \ca to
poisson distributed spikes applied to pre and postsynaptic
pathways}{fig:astro:sweep_alpha_u_tau_1}{0.4}

\asvgf{figures/artifacts/obj1/sweep_u_tau_p0.7.svg}{Response of \ca to
poisson distributed spikes applied to pre and postsynaptic
pathways}{fig:astro:sweep_alpha_u_tau_2}{0.4}

Figures \ref{fig:astro:sweep_alpha_u_tau_1} and
\ref{fig:astro:sweep_alpha_u_tau_2} depict the response of \ca for
different values of $\tau_{ca}$ and different spiking events. In general,
larger values of $\tau_{ca}$ result in smaller deviations in \ca overall, due
to the value decaying more quickly. With smaller $\tau_{ca}$ more of the
previous activity is integrated into the value of \ca and the response is
more broad.

Expanding on the simpler timeline plots, Figure
\ref{fig:astro:u_thr_heat_rate_vs_tau_u} shows the number of times \ca crosses a
threshold of $2.5$ for a given $\tau_{ca}$ and $r$ spiking rate for the Poisson
spike-train inputs. This graph can be used to tune either $thr_{ca}$ or $\tau_{ca}$
for a given application, and gives an impression of the excitability of the
model.

\asvgf{figures/artifacts/obj1/heatmap_spike-rate_tau_ca.svg}{Rate of
astrocyte effect on synaptic weight given different input poisson spike
rates, and different values of $\tau_{ca}$}{fig:astro:u_thr_heat_rate_vs_tau_u}{0.3}

\subsection{Classic-STDP and Variants} \label{sec:ordered_prop:stdp}

Using the Signed Proportional update rule the astrocyte Plasticity Model is
capable of implementing classic \gls{stdp}, along with several variations. Figure
\ref{fig:astro:classic_stdp} shows the model's response to spike pairs with
a variety of inter-spike-interval times (X-axis). This is a typical \gls{stdp}
curve.

\asvgf{figures/artifacts/obj1/1n1s1a_tp_l-stdp_w-dw_mult_dwdt_astro_plasticity.svg}{Astrocyte
  and neuron model response to pulse pairs:
  baseline}{fig:astro:classic_stdp}{0.5}

The model has additional parameters beyond what a traditional \gls{stdp} rule
would have, and this offers flexibility. By providing asymmetric
$\tau_{ip3}$ and $\tau_{k+}$ the model can be bias toward \gls{ltd}/\gls{ltp}. Figure
\ref{fig:astro:stdp_ltd_bias} shows this effect for \gls{ltd}.

\asvgf{figures/artifacts/obj1/1n1s1a_tp_l-stdp_w-dw_mult_dwdt_ltd_bias.svg}{Astrocyte-neuron
model response to pulse pairs: \gls{ltd} bias}{fig:astro:stdp_ltd_bias}{0.5}

In addition, threshold parameters $thr_{ltp}$ and $thr_{ltd}$ can be used to
shift activity from the \gls{ltp} to \gls{ltd} region, or vise versa. Figure
\ref{fig:astro:stdp_ltd_shift} shows this shifting behavior.

\asvgf{figures/artifacts/obj1/1n1s1a_tp_l-stdp_w-dw_mult_dwdt_ltd_dt_shift.svg}{Astrocyte-neuron
model response to pulse pairs: \gls{ltd} shift}{fig:astro:stdp_ltd_shift}{0.5}

Additional figures can be found in Appendix \ref{section:appendix-a}.

One of the main features of this model is it's ability to generalize \gls{stdp},
and capture a variety of variants that have been proposed and explored in the
literature. Table \ref{table:astro_varient_params} maps specific values of
model parameters to the corresponding \gls{stdp} behavior. If that
variant of \gls{stdp} has a name from other research it is included.

\begin{table}[!htp]\centering
\caption{Model Parameters associated with \gls{stdp} variants} \label{table:astro_varient_params}
\scriptsize
\begin{tabular}{lrrrrrrrrrrr}\toprule
STDP Variant &Name &$\alpha_{ip3}$ &$\tau_{ip3}$ &$\alpha_{k+}$ &$\tau_{k+}$ &$\tau_u$ &$thr_{ltp}$ &$thr_{ltd}$ &$reset_{ip3}$ &$reset_{k+}$ \\\midrule
Triplet STDP &Triplet STDP &1 &100 &1 &100 &10000 &0 &0 &Yes &Yes \\
Classic & &1 &100 &1 &100 &10000 &0 &0 &No &No \\
\Gls{ltd} Bias & &1 &100 &1 &30 &10000 &0 &0 &No &No \\
\Gls{ltp} Bias & &1 &30 &1 &100 &10000 &0 &0 &No &No \\
\Gls{ltd} Shift &Mirror STDP &1 &100 &1 &100 &10000 &0.5 &0.5 &No &No \\
\Gls{ltp} Shift & &1 &100 &1 &100 &10000 &-0.5 &-0.5 &No &No \\
Anti-STDP &Anti-STDP &-1 &100 &-1 &100 &10000 &0 &0 &No &No \\
\bottomrule
\end{tabular}
\end{table}

\subsection{Rate-based Response} \label{sec:rate_response}

When input spiking activity is sparse in time, the model behaves as outlined
in Section \ref{sec:ordered_prop:stdp} above. However, when more dense
inputs are provided, the relationship between \ipt and \kp shifts from
temporal, to activity-based. In the sparse case, \ipt $>$ \kp implies that an
incoming spike happened more recently then a post-synaptic spike. When
incoming spikes are dense in time, \ipt $>$ \kp would imply more activity on
the input side (on average) than the output. What follows, is a Rate-based
internal response, where \ca responds to a difference in average activity,
subject to time constants. \eq{eq:astro:rate-u} above describes \ca
behavior. \eq{eq:astro:u-reset} outlines the reset of behavior \ca,
the value reset to $0$ when $|\cam| > thr_{ca}$.

In Figure \ref{fig:astro:rp_many-w_tl}, the middle plot shows the internal
response of an astrocyte to a random Poisson across different weight values. In
this case, there is significant response from the astrocyte for weight values of
$2.0$, with less of a response for $4.0$. This is due to a lower firing rate out
of the \gls{lif} neuron compared to the input. Intuitively there should be a weight
value where the difference between pre and post-synaptic firing rate is
smallest. For the upper graph $w=10.0$ is close to that point, with only two
regions in the timeline with any activity.

\asvgf{figures/artifacts/obj1/astro_rp_many-w_tl.svg}{Astrocyte response to random
  poission spike train for various synaptic weights}{fig:astro:rp_many-w_tl}{0.5}

Looking at the upper plot in Figure \ref{fig:astro:rp_many-w_tl}, the effect of
$thr_{ltp}=1.75$ and $thr_{ltd}=-1.75$ can be seen. Those parameter values form
a hysteresis band, where the astrocyte does not respond with increased \ca for
small deviations between \ipt and \kp. This leads to an increased range of
weight values resulting in stable behavior. Sweeping synaptic weight in Figure
\ref{fig:astro:rp_many_w_sweep} shows that weight values beyond $w \approx 3.0$
result in no \ca activity.

\asvgf{figures/artifacts/obj1/astro_rp_many-w_sweep.svg}{Astrocyte response to random
  poission spike train for various synaptic weights}{fig:astro:rp_many_w_sweep}{0.4}

While this effect with denser inputs is novel, and likely very useful in some
cases, the breakdown of sensitivity to spike-timing is not always a desirable
property. There may be some mitigations, involving tighter time constants
$\tau_{ip3}$ and $\tau_{k+}$, but initial simulations showed other issues with
this. In the next chapter, a modified \emph{Ca(\ipt, \kp)} is introduced, which
exhibits spike-timing sensitivity with sparse and dense inputs.

\section{Integrated Spike Timing Response} \label{section:istp}
 The previous Signed Proportional model outlined in Section
\ref{sec:sign_prop} could exhibit sensitivity to spike timing. This was
demonstrated in Figure \ref{fig:astro:classic_stdp}, where the characteristic
\gls{stdp} curve was reproduced by the astrocyte model. As shown in Section
\ref{sec:rate_response} the sensitivity to spike timing only holds for sparse
inputs.

This \emph{Ca(\ipt, \kp)} function is define by \eq{eq:astro:temp-u}. $reset_{ip3}$ and $reset_{k+}$ parameters, determine if
additional spikes are added into the \ipt and \kp traces, or override
them. These parameters, along with the weight update behavior determine which
groupings of spikes result in changes to \ca. Figure
\ref{fig:astro:spike_associate} outlines possible spike association
behaviors, with the boxes representing which spikes are grouped into a \ca
response. Green boxes represent an increase in \ca (which may eventually lead to
\gls{ltp}) and red a decrease (which may lead to \gls{ltd}).

%% TODO: Define reset behavior

\asvgf{figures/SpikeAssociation.svg}{A variety of spike associations
supported by the neuron-astrocyte model}{fig:astro:spike_associate}{0.5}

\begin{align}
T_{delta} = -\alpha_{ltd} z_{pre} K^+ + \alpha_{ltp} z_{post} IP_3\\
d_{ca} = -\cam \tau_{ca} d_t + T_{delta} \label{eq:astro:temp-u}
\end{align}

In this case, $z_{post}$ and $z_{pre}$ are pre and post-synaptic spikes that
the astrocyte is sensing on a single synapse. The $\alpha_{ip3/k+}$ parameters
are used to control the magnitude of influence \ipt and \kp have on astrocyte
\ca. In addition $\alpha_{ltd/ltp}$ can override default \gls{ltp}/\gls{ltd} behavior,
implementing anti-stdp, or biasing weight updates towards \gls{ltp}/\gls{ltd}. Note also,
that \eq{eq:astro:temp-u} implies astrocyte \ca will decay with each
time step, but only be influenced by \ipt and \kp when spikes occur, and
not continuously. If enough \ca accumulates from coordinated spiking events, a
threshold $thr_{ca}$ is reached and \ca reset. The baseline parameter values
for simulation are outlined in Table \ref{table:istp_params}.

%%  Equations
%% \ref{eq:astro:temp-sign-1} and \ref{eq:astro:temp-sign-2} show definitions
%% of $sign_{ltd}$ and $sign_{ltp}$ that would allow a region around $dt=0$
%% where weight updates were 0.

%% \begin{align}
%% sign_{ltd} = -(1 - H(k+ + thr_{ltd})) \label{eq:astro:temp-sign-1}\\
%% sign_{ltp} = 1 - H(ip3 + thr_{ltp}) \label{eq:astro:temp-sign-2}
%% \end{align}

\begin{table}[!htp] \centering
  \caption{Integrated spike timing astrocyte parameters} \label{table:istp_params}
  \scriptsize
  \begin{tabular}{lrrrrr}\toprule
    dt &$\alpha$-s &$\tau$ \ipt & $\tau_{k+}$ & $thr_{ca}$ \\\midrule
    0.001 s &1 &100 1/s &100 1/s &2.5 \\
    \bottomrule
  \end{tabular}
\end{table}

The changes to \emph{Ca(\ipt, \kp)} introduced in this section did not diminish
the astrocyte's ability to generalize \gls{stdp} and variations. Figures that can be
found in Appendix \ref{appendix:astro_figures} show the full set of \gls{stdp}-like
simulations for both astrocyte configurations in this chapter.

A common theme when discussing astrocytes and their functional roles is one of
time-scale. Spike can occur at any time-step, and last only a single
time-step. When pairing with classic \gls{stdp}, learning can occur just as
quickly, with influence from only a single pair of spikes. This property of
\gls{stdp}, in some cases, leads to instability. As a spike pattern is presented
to a neuron, and that neuron fires, it is possible, and very likely with some
inputs, that \gls{stdp} will associate spikes incorrectly. It is a hasty
learning approach, and naturally can't predict the future. An astrocyte
plasticity model is in a good position to mitigate this then. Defining a
threshold, $thr_{ca}$ which gates any external response by the astrocyte (namely
a weight update) leads to a more patient learning process.

Figure \ref{fig:astro:ramp_impulse} shows the behavior of temporal integration
model when a series of spike impulses are presented to the network. Each
successive pulse has an additional spike from the last and the resulting \ca
activity is depicted for a given synaptic weight. This simulation exposes the
averaging effect of the spike timing model. With an impulse of 4 spikes, the
neuron is driven enough to output a single spike, this happens around
300ms. With 4 pre-synaptic spikes and 1 post-synaptic spike, the astrocyte's \ca
is increased, exceeding the threshold of $2.5$. Subsequent spike impulses have
$>4$ spikes, which leads to a drop in \ca, due to anti-causal pairing
(transiently) of pre-synaptic spikes. Regardless of these transient drops, \ca
consistently exceeds $thr_{ca}$ throughout the
simulation.

\asvgf{figures/artifacts/obj1/astro_tp_impulse.svg}{Astrocyte response to
  ramping impulse input: demonstrating pulse-pair averaging
  feature}{fig:astro:ramp_impulse}{0.5}

Figure \ref{fig:astro:tp_many_w_sweep} graphs the number of times $thr_{ca}$ is
exceeded during a fixed length simulation, for a few different values of
$thr_{ca}$. For $\cam=2.5$, regions of the graph around $0.8$ and $1.2$ show
weight values that result in a stable configuration, where all \ca activity is
sub-threshold. Figure \ref{fig:astro:tp_many_w_tl} shows
a timeline of astrocyte activity for three different weight values, and
$thr_{ca}=2.5$.

\asvgf{figures/artifacts/obj1/astro_tp_many-w_sweep.svg}{Aggregate astrocyte
  response to input from a single synapse, with various synaptic weight and
  $thr_{ca}$}{fig:astro:tp_many_w_sweep}{0.6}

\asvgf{figures/artifacts/obj1/astro_tp_many-w_xlim_tl.svg}{Timeline of astrocyte
  activity with input from a single synapse, given
  $thr_{ca}=2.5$}{fig:astro:tp_many_w_tl}{0.6}

This concept of sub-threshold oscillations is important, as it provides a
mechanism for astrocyte-mediated plasticity to reach a convergence point given a
set of inputs. This convergence point is dependent on the input pattern in
question, and the $\Delta \cam$ vs $\Delta t$ curve characterizing the \ca
response. Considering Figure \ref{fig:astro:tp_many_w_tl} and recalling the
shape of Figure \ref{fig:astro:classic_stdp}: The three output spikes align with
input spikes, which maximizes $\Delta t$ between the output spikes, and
neighboring presynaptic spikes in any direction. This maximal spacing leads to a
minimum \ca response. Note that spikes with $\Delta t=0$ do not elicit a \ca
response. Independent of the weight value \ca remains sensitive to activity at a
synapse. Changes in the input firing pattern may result in \ca oscillations
exceeding the threshold once more, and trigger another round of weight updates.

%% Using values $thr_{ltd}=thr_{ltp}=1.0$ (see \eq{eq:astro:temp-sign-1})
%% results in a slight change in the model behavior, which adds the condition in
%% \eq{eq:dca_band}, resulting in different behavior, outlined in Figure
%% \ref{fig:astro:tp_many_w_sweep}. In general, there are greater regions of
%% stability when compared to the previous simulation without the
%% change. For $thr_{ca}=3.0$ there are no threshold events within the range of
%% synaptic weights used. Essentially, this change increases the tolerance for
%% Astrocyte response, and for what is considered synchronous activity.

%% \asvgf{figures/artifacts/obj1/astro_tp_many-w_dca_sweep.svg}{Number of \ca
%%   events with different synaptic weights.}{fig:astro:tp_many_w_sweep}{0.5}

\section{Summary}

This chapter introduces a computationally simple and efficient astrocyte
model. This model was developed with inspiration from the many works outlined in
Chapter \ref{chapter:background}, and focuses on an astrocyte's role in synaptic
plasticity. These include Neuroscience works, which outline observed and
theorized bio-chemical pathways. As well as works which present highly realistic
mathematical models of astrocyte behavior. For the astrocyte model developed in
this work, an emphasis was placed on computational efficiency, while maintaining
the common pathways pulled from the body of background research. Compared to
some of the more bio-realistic astrocyte models, a relatively small number of
parameters are used to define this model's behavior.

The literature shows considerable diversity on the main functional role(s) of an
astrocyte, as well as how a given function is achieved chemically. In this work,
with synaptic plasticity as a main functional role, a framework for astrocyte
local dynamics was developed, and is outlined in Figure
\ref{fig:1n1s1a_fn_diagram}. \emph{Ca(ip3, k+)} and \emph{Dw(ca)} are generic
functional blocks which represent the gap in existing literature.

It is assumed the \ca will drive synaptic plasticity, though the exact mechanism
isn't explored in this chapter. With that in mind, two different approaches are
considered for \ca dynamics, both of which can reproduce the characteristic
\gls{stdp} curve in their \ca response, as well as common variation. It is discussed
how this \ca response could extend \gls{stdp} as a learning rule. These
extensions include transformations on the standard \gls{stdp} weight update curve, as
well as introducing a third factor to the learning rule. One approach, which
considers the \ca response to a difference in the magnitude between \ipt and
\kp, was shown to generalize and \gls{stdp}-like response (Figure
\ref{fig:astro:classic_stdp}), and support common variations on
\gls{stdp} with one model. In addition, in the presence of dense spiking input the
astrocyte showed a response to the difference between pre and post-synaptic
average firing rates. The second approach instead uses an \emph{Ca(ip3, k+)} that
results in sensitivity to pre and post-synaptic spike timing. This model
generalizes \gls{stdp} in both sparse and dense inputs, supporting common \gls{stdp}
variants as well.

One of the major drawbacks to \gls{stdp} is its inherent instability. It is well known
that unchecked, \gls{stdp} will drive synaptic weights to either minimum or maximum
limits \parencite{legenstein_2005}. One of the key features of astrocyte plasticity
is an inherent stability at non-maximal weight values. This stability was
demonstrated in both configurations outlined above, which was exhibited by
minimal and sub-threshold \ca activity.

Overall, this objective introduces a framework for astrocyte-\gls{lif} Neuron
computation, and explores two specific models geared towards synaptic
plasticity. For each of these models, synaptic weight values of an \gls{lif} neuron
(coupled with an astrocyte) are swept, and stable configurations are identified,
which represent convergence points in a potential learning rule.

%% This chapter introduces a computationally simple and efficient Astrocyte
%% model. This model was developed with inspiration from the many works outlined in
%% Chapter \ref{chapter:backround}, and focuses on an Astrocyte's role in synaptic
%% plasticity. These include Neuroscience works, which outline observed and
%% theorized bio-chemical pathways. As well as works which present highly realistic
%% mathematical models of Astrocyte behavior. For the Astrocyte model developed in
%% this work, an emphasis was placed on computational efficiency, while maintaining
%% the common pathways pulled from the body of background research.  Compared to
%% some of the more bio-realistic Astrocyte models, A relatively small number of
%% parameters are used to define this model's behavior.

%% The literature shows considerable debate on the main functional role of an
%% Astrocyte, as well as how a given functional role is achieved chemically. In
%% this work, with synaptic plasticity as a main functional role, a framework for
%% Astrocyte local dynamics was developed, and is outlined in Figure
%% \ref{fig:1n1s1a_fn_diagram}. ``Update FN'' is a generic functional block which
%% represents the gap in existing literature.

%% It is assumed the \ca will drive synaptic plasticity, though the exact mechanism
%% isn't explored in this chapter. With that in mind, two different approaches
%% are considered, both of which are capable of generalizing STDP, and can extend
%% it in different ways. One, which considers the \ca response to a difference in
%% the magnitude between \ipt and \kp, was shown to generalize and STDP-like
%% response (Figure \ref{fig:astro:classic_stdp}), and extend it, supporting common
%% variations on STDP with one model. In addition, in the presence of dense spiking
%% input the Astrocyte showed a response to the difference between pre and
%% post-synaptic average firing rates. Sweeping weight values showed a stable
%% configuration to the model, where \ca response was minimal, or non-existent.

%% The second approach instead uses an ``Update FN'' that results in sensitivity to
%% pre and post-synaptic spike timing, instead of firing rate. This model
%% generalizes STDP in both sparse and dense inputs, as well as supporting common
%% STDP variants. Sweeping weight values for particular parameters lead to stable
%% configurations with minimal \ca response as well.

%% Overall, this objective introduces a framework for Astrocyte-\gls{lif} Neuron
%% computation, and explores two specific models geared towards synaptic
%% plasticity. For each of these models, stable configurations are explored, which
%% represent convergence points in a potential learning rule.
