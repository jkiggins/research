% introduction.tex
%
% Author       : James Mnatzaganian
% Contact      : http://techtorials.me
% Date Created : 08/27/15
%
% Description  : Introduction chapter used by "thesis.tex".
%
% Copyright (c) 2015 James Mnatzaganian
%
% Version 2 ::
% Author       : Andres Kwasinski
% Contact      : https://people.rit.edu/axkeec/
% Date Created : 01/09/2020
%

% NOTE: All filler text has "TODO" written. This must be removed in the final copy!

\chapter{Introduction}\label{section:introduction}
\section{Motivation}

The human brain exhibits may features that current approaches to machine
learning and intelligence are unable to mimic. Certainly, deep learning has
achieved fantastic results in a variety of specialized tasks, but this success
comes with a cost. Increasingly higher parameters counts are needed to
achieve improvements, and with each parameter comes additional time and energy
for training. Once trained, computing a deep network's decision in real-time
requires specialized, power-hungry hardware. The human brain on the other hand,
exhibits a much higher level of connectivity, is able to handle a wide variety
of complex tasks, and has many advantages in the context of learning, and
tolerance to adversarial inputs. Where a deep learning model requires many
diverse examples to approach generalization, the human brain is able to
surpass this with only a few examples \cite{tsimenidis_2020}. Even more
impressive, the brain can accomplish these feats with $\approx$ 20 W of
power. At a high level, this work explores \Glspl{snn} and Astrocytes in an
effort to explain, and realized some of these features.

Beyond the high-level advantages, both \gls{snn} literature and research
surrounding computational astrocyte models have gaps. \glspl{snn} are difficult
to train, especially at a network level. Overall, training is holding back
the success of \gls{snn} despite their rich dynamics and advantages with
time-series data \cite{tavanaei_2019}. Considering astrocytes, there is a clear
role for them in computation, learning, and overall cognition
\cite{mederos_2018}. Many existing computational astrocyte models fail to clearly
define a functional role and works that do, don't focus on aspects that solve key
problems in \glspl{snn} \cite{manninen_2019}. This leaves a clear need for an
computational astrocyte model that focuses on extracting key features from
biology, and advances the functionality of \glspl{snn}.

\section{Thesis Objectives}
The overarching goal of this thesis was to develop a bio-inspired astrocyte
model, and more broadly an astrocyte-neuron interaction model. This model was
designed to be computationally simple, scalable, and captures common themes
within Neuroscience literature. To direct this research, the developed astrocyte
element drove synaptic plasticity; First generalizing, then diverging from the
widely used STDP rule. Benefits of Astrocyte-like control of synaptic plasticity
was demonstrated in the single synapse, single neuron configuration (1S1N) as
well as select multi-synapse configurations. Inputs consisted of procedurally
generated spiking inputs, including Poisson rate-coded spike trains, spiking
inputs representing Boolean variables, and temporal inputs that followed
specific patterns. To evaluate each experiment different intrinsic properties
were observed, such as convergence/divergence of weights, speed of convergence,
stable points across a parameter space, and lastly performance on any task
driven experiments. Statistical analysis of various intermediate signals were
used to gain insight.

More specific objectives.
\begin{enumerate}
  %% Specify -> developing a novel astrocyte model. Stress what is new and value-added
\item Develop an Astrocyte \Gls{lif}-Neuron model based on key bio-chemical
  pathways that are common in background literature. This Astrocyte should
  respond internally to pre and post-synaptic spikes In a way that can mimic
  STDP, but is more generalized and flexible.

  \item Show, in the case of a single spiking neuron with one input, how the
    developed Astrocyte-\Gls{lif} model can drive synaptic plasticity in a way that
    generalizes, and extends STDP.

  \item Extend the Developed Astrocyte-Synapse Model to include coordination of
    plasticity rules across multiple synapses, and show how this coordination is
    critical to a learning task.

\end{enumerate}
