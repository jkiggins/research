\chapter{Deployment of The Astrocyte Plasticity Model in A Single Synapse
  Configuration} \label{chapter:obj2}

\section{Pulse-Pair Integration}
One unique property of the Astrocyte Plasticity model is the integration of
plasticity signals over time. This differs from classic STDP, where weight
updates are immediate in response to activity. The Astrocyte model can easily
accomplish this integration effect through it's $Ca^{2+}$ state
variable. Instead of driving a weight change in response to a pair of pre and
post-synaptic spikes, the Astrocyte increases or decreases $Ca^{2+}$. Only once Ca2+
reaches some threshold does the weight update occur. This is the behavior
described in \ref{section:istp}.

\subsection{Impulse Response}
To test this behavior, Figure \ref{fig:ppi:impulse_istp} depicts the Astrocyte
plasticity model with $Ca_{thr}=2.5$. In contrast, Figure
\ref{fig:ppi:impuse_stp} shows the behavior of a classic STDP rule with the same
inputs and configuration. The inputs themselves consist of successive spikes,
with 1ms between each. These spikes form groups with enough time between for
neuron and Astrocyte states to decay. Each group has an additional spike
compared to the last.

Comparing these graphs, it is clear that integrating
behavior of the Astrocyte model effects the learning process significantly. In
this case, resulting in more stable behavior, and increasing synaptic weight due
to causal inputs. The key effect here, can be seen considering what happens when
there are additional input spikes after a post-synaptic spike. In this case
classic STDP would decrease the weight, the Astrocyte rule integrates this event
into Ca2+. Those input spikes are eventually associated with an additional
down-stream spike (showing those spikes were causal after all) and Ca2+ is
increased. Eventually this leads to LTP, with the potential LTD events being
smoothed out.

%% TODO: Add ISTP Paths

It would be equally interesting to explore the networks response to a group of
spikes of the same length across a similar timescale. Figures
\ref{fig:impulse_ss_stdp} and \ref{fig:impulse_ss_istp} outline the plasticity
behavior for classic STDP and the Astrocyte model.

%% TODO: is a steady-state reached



Another situation where the Astrocyte model could have some functional benefit
is reward-modulated STDP (R-STDP). Intuitively, integrating reward signals could
help balance weight updates across multiple input, gt pairs, where the learning
rule needs to distinguish between similar inputs.
